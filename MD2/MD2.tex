\documentclass{report}

\input{../preamble.tex}
\input{../letterfonts.tex}

\title{\Huge{Matemática Discreta 2}}
\author{\huge{Santiago Sierra}}
\date{}

\begin{document}

\maketitle
\newpage% or \cleardoublepage
% \pdfbookmark[<level>]{<title>}{<dest>}
\pdfbookmark[section]{\contentsname}{toc}
\tableofcontents
\pagebreak

\chapter{Divisibilidad}
\section{Introducción}
\thm{Teorema de División Entera}{Dados $a,\ b\in\mathbb{Z}$, con $b\neq 0$, existen únicos $q,\ r\in\mathbb{Z}$ con $0\le r<|b|$ y $a=bq+r$.
	\begin{enumerate}
		\item A $q$ se le llama el cociente, y a $r$ el resto de dividir $a$ entre $b$.
		\item Basta con suponer que $b>0$, ya que si $a=bq+r$ entonces $a=(-b)(-q)+r$.
		\item Basta con suponer que $a\ge 0$, ya que si $a=bq+r$ (con $b>0$ y $0\le r<b$) entonces $-a=-bq-r$, pero aquí si $r\neq 0$ no obtuvimos un resto positivo. Sumando y restando $b$, tenemos que: $-a=b(-q)-b+b-r=b(-q-1)+(b-r)$ y si $r\neq 0$ al ser $0<r<b$, tenemos que $0<b-r<b$.
	\end{enumerate}}
\begin{myproof}
	Vamos a suponer que $a\ge 0$ y $b>0$, veamos primero la existencia:
	\\Consideremos el conjunto $$S=\{s\in\mathbb{N}:\ s=a-bx\text{ para algún }x\in\mathbb{Z}\}$$
	Entonces, como $a\ge 0$ tomando $x=0$, tenemos que $a\in S$ y, por lo tanto $\emptyset\neq S\subset\mathbb{N}$. Como todo conjunto de naturales no vació tiene mínimo, llamamos $r=min\ S$. Así que por la definición de $S$ tenemos que $r\ge 0$ y que existe un $q\in\mathbb{Z}$ con $r=a-bq$ y, por lo tanto $a=bq+r$. Entonces solo queda probar qué $r<b$.\\\\
	Supongamos lo contrario, que $r\ge b$; en este caso tendríamos que $r=b+s$ con $0\le s<r$. Pero en este caso tendríamos que $s=r-b=a-bq-b=a-b(q+1)$ y tendríamos que $s\in S$ lo cual es absurdo pues $s<r=min\ S$.\\\\
	Veamos la unicidad: supongamos que $a=bq_1+r_1$ y $a=bq_2+r_2$ con $0\le r_1,r_2<b$, entonces $bq_1+r_1=bq_2+r_2$, por lo tanto $r_2=b(q_1-q_2)+r_1$.\\
	Si $q_1-q_2\ge 1$ tendríamos que $r_2\ge b$, y si $q_1-q_2\le -1$ tendríamos que $r_2<0$ (pues $r_1<b$). Así que $q_1-q_2=0$, y sustituyendo, nos queda qué $r_1=r_2$.
\end{myproof}
\cor{}{
Sean $b\in\mathbb{N}$, con $b\ge 2$ y $x\in\mathbb{N}$, entonces existen $a_0,a_1,\dots,a_n$ enteros tales que podemos escribir a $x$ en base $b$ como $$x=b^na_n+b^{n-1}a_{n-1}+\dots+b^1a_1+b^0a_0=\sum_{i=0}^{n} b^ia_i,\text{ y } 0\le a_i<b,\ a_n\neq 0$$
}
\begin{myproof}
	Lo probamos por inducción en $x\in\mathbb{N}$. Sí $x=0$ es claro porque $x=b^0\times 0$.\\
	Sí $x>0$, por el teorema anterior existen $q$ y $r$ tales que $x=bq+r$ con $0\le r<b$. Como $q<x$ aplicamos la hipótesis inductiva para obtener $$q=\sum_{i=0}^{n} b^i a'_i$$
	con $0\le a'_i<b$. Entonces $$x=b\left( \sum_{i=0}^{n} b^i a'_i \right)+r=\left( \sum_{i=0}^{n} b^{i+1} a'_i\right) +r=\sum_{i=1}^{n+1} b^i a'_{i-1}+r=\sum_{i=0}^{n+1} b^ia_i $$
	con $a_0=r$ y $a_{i+1}=a'_i$ para $i=0,1,\dots,n$, demostrado así el corolario.
\end{myproof}
\ex{}{Escribamos $n=233$ en base $4$.
	\begin{align*}
		233 & =4\times 58+1                                                                                                     \\
		    & =4\times(4\times 14+2)+1                                                                                          \\
		    & =4\times(4\times(4\times 3+2)+2)+1                                                                                \\
		    & =4^3\times \textcolor{red}{3}+4^2\times\textcolor{red}{2}+4^1\times\textcolor{red}{2}+4^0\times\textcolor{red}{1} \\
		    & =(\textcolor{red}{3221})_4
	\end{align*}}
\dfn{}{Dados $n,m\in\mathbb{Z}$ decimos que $m$ divide a $n$ si existe $q\in\mathbb{Z}$ tal que $n=qm$. En este caso escribimos $m\mid n$, y en caso contrario escribiremos $m\nmid n$.}
\cor{}{\begin{enumerate}
		\item Tenemos que $m$ divide a $n$ si y solo si, el resto de dividir $n$ entre $m$ es cero.
		\item $\pm 1\mid a,\ \forall a\in\mathbb{Z}$. Ademas si un entero $x$ cumple que $x\mid a,\ \forall a\in\mathbb{Z}$, entonces $x=\pm 1$.
		\item $b\mid 0,\ \forall b\in\mathbb{Z}$. Ademas, si un entero $x$ cumple que $b\mid x,\ \forall b\in\mathbb{Z}$, entonces $x=0$.
		\item $\pm n\mid n\ \forall n\in\mathbb{Z}$.
		\item Si $b\mid a$ y $a\neq 0$ entonces $|b|\le |a|$.
		\item Si $a\mid b$ y $b\mid a$ entonces $a=\pm b$.
		\item Si $a\mid b$ y $b\mid c$ entonces $a\mid c$ (transitiva).
		\item Si $db\mid da$ y $d\neq 0$ entonces $b\mid a$ (cancelativa).
		\item Si $b\mid a$, entonces $db\mid da$ para todo $d\in\mathbb{Z}$
		\item En particular, si $d$ divide a $n$ y a $m$, entonces $d$ divide al resto de dividir $n$ entre $m$.
	\end{enumerate}}\newpage\section{Máximo Común Divisor}
\dfn{}{Si $a\in\mathbb{Z}$ escribiremos $Div(a)$ al conjunto de divisores de $a$ y $Div_+(a)$ al conjunto de divisores positivos de $a$. Es decir $Div(a)=\{x\in\mathbb{Z}:\ x\mid a\}$ y $Div_+(a)=\{x\in\mathbb{Z}^+:\ x\mid a\}$.}
\cor{}{Observar que si $a\neq 0$ y $x\mid a$ entonces como $|x|\le |a|,\ Div(a)\subset\{\pm 1,\pm 2,\dots,\pm a\}$, y por lo tanto $Div(a)$ es un conjunto finito (y en particular acotado).\\
	Dados $a,b\in\mathbb{Z}$ diremos que $x\in\mathbb{Z}$ es un divisor común de $a$ y $b$ si $x\mid a$ y $x\mid b$; es decir, el conjunto de divisores comunes de $a$ y $b$ es $Div(a)\cap Div(b)$.\\
	Observar que si $a\neq 0$ o $b\neq 0$ entonces el conjunto de divisores comunes de $a$ y $b$ es finito y por lo tanto tiene máximo.}
\dfn{}{Sean $a,b\in\mathbb{Z}$, definimos el máximo común divisor de $a$ y $b$, que escribiremos $mcd(a,b)$, de la siguiente manera:
	\begin{itemize}
		\item Si $a\neq 0$ o $b\neq 0$, definimos $$mcd(a,b)=max(Div(a)\cap Div(b))=max\{x\in\mathbb{Z}:\ x\mid a\text{ y }x\mid b\}$$
		\item En caso contrario definimos $mcd(0,0)=0$.
	\end{itemize}}
\mprop{}{\begin{enumerate}
		\item $mcd(1,a)=1\ \forall a\in\mathbb{Z}$.
		\item $mcd(0,b)=|b|\ \forall b\in\mathbb{Z}$.
		\item $mcd(a,b)=mcd(|a|,|b|)\ \forall a,b\in\mathbb{Z}$.
		\item Cuando $mcd(a,b)=1$ decimos que $a$ y $b$ son coprimos o primos entre sí.
	\end{enumerate}}
\cor{}{Dados $a,b\in\mathbb{Z}$ con $a,b\neq 0$ entonces:
	\begin{enumerate}
		\item $mcd(a,b)=mcd(b,a-bx)\ \forall x\in\mathbb{Z}$.
		\item En particular, si $r$ es el resto de dividir $a$ entre $b$, se tiene que $mcd(a,b)=mcd(b,r)$.
	\end{enumerate}}
\begin{myproof}
	Por la propiedad $3$ que mencione anteriormente, basta con probarlo para $a$ y $b$ positivos. Llamemos $d=mcd(a,b)$ y $d'=mcd(b,a-bx)$. Como $d|a$ y $d|b$, por lo visto en las propiedades del Corolario 1.2 tenemos que $d$ divide a cualquier combinación lineal entera de $a$ y $b$, en particular, $d|a-bx$.\\Por lo tanto $d\in Div(b)\cap Div(a-bx)$, y entonces $d\le max(Div(b)\cap Div(a-bx))=d'$.\\\\
	Por otro lado, $d'|b$ y $d'|a-bx$; utilizando el mismo razonamiento, tenemos que $d'$ divide a $(a-bx)+x(b)=a$. Así que $d'\in Div(a)\cap Div(b)$ y tenemos qué $d'\le max(Div(a)\cap Div(b))=d$.
\end{myproof}
\dfn{Algoritmo de Euclides}{Dados $a,b\in\mathbb{Z}$ con $a\ge b>0$. Y sea $r(a,b)$ el resto de dividir $a$ entre $b$:
	\begin{itemize}
		\item Fijamos $r_0=b$.
		\item Sea $r_1=r(a,b)$; por lo tanto tenemos que $mcd(a,b)=mcd(b,r_1)$ y que $0\le r_1<b$.
		\item Si $r_1=0$, entonces $mcd(a,b)=mcd(b,r_1)=mcd(b,0)=b$; y si no, sea $r_2=r(b,r_1)$. Por lo tanto $0\le r_2<r_1<b$ y $mcd(a,b)=mcd(b,r_1)=mcd(r_1,r_2)$.
		\item Se sigue de esta forma, definiendo en el paso $i+1$, $r_{i+1}=r(r_{i-1},r_i)$, en particular tenemos que $0\le r_{i+1}<r_i$ y que $mcd(r_{i-1},r_i)=mcd(r_i,r_{i+1})$. De esta forma, vamos construyendo enteros, hasta conseguir $r_n=0$, para obtener $$mcd(a,b)=mcd(b,r_1)=mcd(r_1,r_2)=\dots=mcd(r_{n-1},r_n)=mcd(r_{n-1},0)=r_{n-1}$$
	\end{itemize}}
\thm{Igualdad de Bezout}{Sean $a,\ b\in\mathbb{Z}$ con $(a,b)\neq(0,0)$, entonces:
	\begin{enumerate}
		\item $mcd(a,b)=min\{s\in\mathbb{Z}^+: s=ax+by\ \text{para algún }x,y\in\mathbb{Z}\}$
		\item (Identidad de Bezout) $\exists x,y\in\mathbb{Z}\ /\ mcd(a,b)=ax+by$.
	\end{enumerate}}
\nt{Alcanza probarlo para $a,b\in\mathbb{Z}^+$.}
\mprop{}{Los números $x,y\in\mathbb{Z}$ de la segunda parte se llaman "\textnormal{c}oeficientes de Bezout" (no son únicos).}
\begin{myproof}
	Llamemos $S=\{s\in\mathbb{Z}^+:\ s=ax+by \text{ con }x,y\in\mathbb{Z}\}$, por definición, tenemos que $S\subset\mathbb{Z}^+$ y además $S\neq\emptyset$ ya que tomando $x=a$ e $y=b$, tenemos que $s=ax+by=a^2+b^2>0$ así que $a^+b^\in S$.\\
	Entonces por el principio de buen orden, $S$ tiene mínimo, y lo llamamos $s_0=min\ S$.\\Queremos probar que $s_0=mcd(a,b)$ y lo haremos probando las dos desigualdades. Tenemos entonces que $s_0>0$ y que existen $x_0,y_0\in\mathbb{Z}$ tales qué $s_0=ax_0+by_0$.\\Llamemos $d=mcd(a,b)$. Como $d|a$ y $d|b$, tenemos $d|ax_0+by_0=s_0$. Por lo tanto $d\le s_0$.\\\\
	Probemos ahora que $s_0$ divide a $a$ y $b$. Por el teorema de división entera, tenemos que existen $q,r\in\mathbb{Z}$ con $a=qs_0+r$ y $0\le r<s_0$.\\Luego $r=a-qs_0=a-q(ax_0+by_0)=a(1-qx_0)+b(-qy_0)$. Por lo tanto, si $r$ fuera positivo tendríamos que $r\in S$; pero como $s_0$ es el menor entero positivo en $S$ y $r<s_0$, tenemos qué $r=0$. Resulta entonces que $a=qs_0$ y, por lo tanto $s_0|a$. De igual modo se muestra qué $s_0|b$.\\Hemos obtenido que $s_0$ es un divisor común de $a$ y $b$, luego $s_0\le d$.
\end{myproof}
\mprop{}{Sean $a,b\in\mathbb{Z}$, no nulos
	\begin{enumerate}
		\item Si $e\in\mathbb{Z}$ es tal que $e|a$ y $e|b$ entonces $e|mcd(a,b)$.
		\item $mcd(a,b)=1\Leftrightarrow\exists x,y\in\mathbb{Z}$ tal que $ax+by=1$.
		\item Si $n\in\mathbb{Z}$ entonces $mcd(na,nb)=|n|mcd(a,b)$.
		\item Sea $d\in\mathbb{Z}^+$ tal que $a=da^*$ y $b=db^*$ con $a^*,b^*\in\mathbb{Z}$. Entonces $d=mcd(a,b)\Leftrightarrow mcd(a^*,b^*)=1$.\\A los enteros $a^*$ y $b^*$ tales que $a=mcd(a,b)a^*$ y $b=mcd(a,b)b^*$ se les llama cofactores de $a$ y $b$.
	\end{enumerate}}
\cor{}{Sean $a,b,c\in\mathbb{Z}$ con $mcd(a,b)=1$. Si $a|bc$ entones $a|c$.}
\begin{myproof}
	Por la igualdad de Bezout, tenemos que existen $x,y\in\mathbb{Z}$ tales que $1=ax+by$. Multiplicando por $c$ tenemos que $c=cax+cby$. Ahora, $a|a$ y por hipótesis $a|cb$ y, por lo tanto $a|a(cx)+cb(y)=c$.
\end{myproof}
\cor{}{Sea $p$ un entero primo y $b,c\in\mathbb{Z}$. Si $p|bc$ entonces $p|b$ o $p|c$.}
\begin{myproof}
	Si $p\nmid b$, entonces (al ser $p$ primo) tenemos que $mcd(p,b)=1$, y por el Lema de Euclides concluimos que $p|c$.
\end{myproof}
\cor{}{Sea $p\in\mathbb{N}$ que cumple que si $p|bc$ entonces $p|b$ o $p|c$, luego $p$ es primo.}
\begin{myproof}
	Supongamos por absurdo que $p$ no es primo, entonces existen $b$ y $c$ tales que $1<b,\ c<p$ y $p=bc$. Por hipótesis, como $p|p=bc$, se tiene que $p|b$ o $p|c$. Además $b|p$ y $c|p$. Concluimos que $p=b$ o $p=c$, pero $b,c<p$. Por lo tanto, $p$ tiene que ser primo.
\end{myproof}
\cor{}{\label{cor:DivPrimos}Sea $p$ un entero primo, y $a_1,\dots,a_n$ enteros, tales que $p|a_1a_2\dots a_n$. Entonces $p|a_i$ para algún $i\in\{1,\dots,n\}$.}
\dfn{}{Dados $a,b\in\mathbb{Z}$ no nulos, definimos el mínimo común múltiplo de $a$ y $b$ como:$$mcm(a,b)=min\{x\in\mathbb{Z}^+:\ a|x\text{ y }b|x\}$$En el caso de que alguno sea nulo, definimos $mcm(0,b)=0,\ \forall b\in\mathbb{Z}$.}
\dfn{}{Dados $a,b\in\mathbb{Z}$ no nulos, se cumple que $$mcm(a,b)=\frac{|ab|}{mcd(a,b)}$$}
\begin{myproof}
	Llamemos $m=mcm(a,b)$ y sean $a^*$ y $b^*$ los cofactores de $a$ y $b$. Claramente $\frac{|ab|}{mcd(a,b)}>0$ y $\frac{|ab|}{mcd(a,b)}=|ab^*|=|a^*b|$ es múltiplo de $a$ y $b$; así que $m\le \frac{|ab|}{mcd(a,b)}$.\\
	Por otro lado, como $a|m$, existe $k\in\mathbb{Z}$ tal que $$m=ak=mcd(a,b)a^*k$$
	Como $b|m$ y $b=mcd(a,b)b^*$ tenemos que $mcd(a,b)b^*|mcd(a,b)a^*k$. Como $mcd(a,b)\neq 0$, por la cancelativa tenemos que entonces $b^*|a^*k$. Ahora como $mcd(a^*,b^*)=1$, por el Lema de Euclides, tenemos que $b*|k$. Por lo tanto, existe $k'\in\mathbb{Z}$ tal que $k=b^*k'$ y sustituyendo, obtenemos que $m=ab^*k'$ y, por lo tanto $\frac{|ab|}{mc(a,b)}=|ab^*|\le m$.
\end{myproof}
\section{Pruebas de Irracionalidad}
\cor{}{Si $p$ es primo entonces $\sqrt{p}$ no es racional.}
\newpage\section{Algoritmo de Euclides Extendido}
Veamos ahora un método para hallar coeficientes de Bezout; es decir, $x,y\in\mathbb{Z}$ tales que $mcd(a,b)=ax+by$.\\
Escribimos los datos de cada paso del Algoritmo de Euclides en forma de vector.\\
En general, si partimos del dato inicial $B_0=\begin{pmatrix} a \\ b \end{pmatrix}$:
\begin{enumerate}
	\item En el primer paso del algoritmo de Euclides realizamos $a=bq_1+r_1$, y obtenemos los nuevos datos $$B_1=\begin{pmatrix} b \\ r_1 \end{pmatrix}=\begin{pmatrix} 0 & 1 \\ 1 & -q_1 \end{pmatrix} B_0 $$
	      Llamemos $M_1=\begin{pmatrix} 0 & 1 \\ 1 & -q_1 \end{pmatrix}$.
	\item Luego realizamos lo mismo con estos nuevos datos: $b=q_2r_1+r_2$, y obtenemos los nuevos datos $$B_2=\begin{pmatrix} r_1 \\ r_2 \end{pmatrix} \text{ y } M_2=\begin{pmatrix} 0 & 1\\ 1 & -q_2 \end{pmatrix}$$
	      con la relación $$B_2=\begin{pmatrix} r_1 \\ r_2 \end{pmatrix}=\begin{pmatrix} 0 & 1 \\ 1 & -q_2 \end{pmatrix} B_1=M_2M_1B_0$$
	\item Y seguimos el algoritmo, donde cada paso con los datos $B_i=\begin{pmatrix} r_{i-1} \\ r_i \end{pmatrix} $ escribiendo $r_{i-1}=q_{i+1}r_i+r_{i+1}$ obtenemos los nuevos datos $B_{i+1}=\begin{pmatrix} r_i \\ r_{i+1} \end{pmatrix} $ y la matriz $M_{i+1}=\begin{pmatrix} 0 & 1 \\ 1 & -q_{i+1} \end{pmatrix} $, con la relación $$B_{i+1}=\begin{pmatrix} r_i \\ r_{i+1} \end{pmatrix} =\begin{pmatrix} 0 & 1 \\ 1 & -q_{i+1} \end{pmatrix} B_i=M_{i+1}B_i=M_{i+1}M_i\dots M_1B_0$$
	\item Al obtener el primer resto nulo, $r_n=0$ tendremos que en el paso anterior $$B_{n-1}=\begin{pmatrix} r_{n-2}\\r_{n-1} \end{pmatrix}=\begin{pmatrix} r_{n-2}\\mcd(a,b) \end{pmatrix}=M_{n-1}\dots M_1B_0$$
	      Llamando $M=M_{n-1}\dots M_1$ tenemos que $$B_{n-1}=\begin{pmatrix} r_{n-2}\\mcd(a,b) \end{pmatrix}=M\ B_0=M\begin{pmatrix} A \\ B \end{pmatrix}:$$
	      por lo tanto, si $M=\begin{pmatrix} z & w \\ x & y \end{pmatrix} $, la ultima fila nos dice que $mcd(a,b)=xa+yb$; es decir, la segunda fila de $M$ son coeficientes de Bezout para $a$ y $b$.
\end{enumerate}
\newpage\ex{}{El dato inicial del algoritmo es el vector $B_0=\begin{pmatrix} 132 \\ 28 \end{pmatrix} $.
	\begin{itemize}
		\item En el primer paso, a partir de $132=4\times 28+20$, cambiamos los datos del algoritmo a $B_1=\begin{pmatrix} 28\\20 \end{pmatrix} $, observar que:$$B_1=\begin{pmatrix} 28\\20 \end{pmatrix}=\begin{pmatrix} 0 & 1\\1 & -4 \end{pmatrix} \begin{pmatrix} 132\\28 \end{pmatrix}  $$
		\item En el segundo paso, a partir de $28=1\times 20+8$, cambiamos los datos del algoritmo a $B_2=\begin{pmatrix} 20\\8 \end{pmatrix} $, queda:$$B_2=\begin{pmatrix} 20\\8 \end{pmatrix}=\begin{pmatrix} 0&1\\1&-1 \end{pmatrix}\begin{pmatrix} 0 & 1\\1&-4 \end{pmatrix}\begin{pmatrix} 132\\28 \end{pmatrix}  $$
		\item En el segundo paso, a partir de $20=2\times 8+4$, cambiamos los datos del algoritmo a $B_3=\begin{pmatrix} 8\\4 \end{pmatrix} $, observamos que: $$B_3=\begin{pmatrix} 8\\4 \end{pmatrix}=\begin{pmatrix} 0&1\\1&-2 \end{pmatrix} \begin{pmatrix} 20\\8 \end{pmatrix}=\begin{pmatrix} 0&1\\1&-2 \end{pmatrix}\begin{pmatrix} 0&1\\1&-1 \end{pmatrix} \begin{pmatrix} 0&1\\1&-4 \end{pmatrix} \begin{pmatrix} 132\\28 \end{pmatrix}  $$
		      Ahora, como $8=2\times 4+0$, es decir el resto es 0, ya podemos hacer el producto $$\begin{pmatrix} 0&1\\1&-1 \end{pmatrix}\begin{pmatrix} 0&1\\1&-1 \end{pmatrix} \begin{pmatrix} 0&1\\1&-4 \end{pmatrix}=\begin{pmatrix} -1&5\\3&-14 \end{pmatrix}  $$
		      Obteniendo $$\begin{pmatrix} 8\\4 \end{pmatrix}=\begin{pmatrix} -1&5\\3&-14 \end{pmatrix} \begin{pmatrix} 132\\28 \end{pmatrix}$$
		      En particular, obtenemos que $4=3(132)-14(28)$. Obtuvimos entonces que $x=3$ e $y=-14$ verifican que $4=132x+28y$.
	\end{itemize}}
\newpage\section{Ecuaciones diofánticas lineales}
\dfn{}{Una ecuación diofántica lineal en las variables $x,y$ es una ecuación de la forma $ax+by=c$, con $a,b,c\in\mathbb{Z}$.\\Nos interesa buscar todas las soluciones enteras a la ecuación, por lo tanto, diremos que el conjunto solución es:$$S=\{(x,y)\in\mathbb{Z}\times\mathbb{Z}:\ ax+by=c\}$$A partir de ahora, cuando hablamos de una solución a la ecuación, nos referimos a un par $(x,y)\in S$.}
\thm{}{\label{thm:Diofanticas}Sean $a,b,c$ enteros con $(a,b)\neq (0,0)$. Entonces la ecuación diofántica $ax+by=c$
	\begin{itemize}
		\item Tiene solución si y solo si $mcd(a,b)|c$.
		\item Ademas, si tiene una solución, tiene infinitas. Es mas, si $(x_0,y_0)$ es una solución, entonces el conjunto de soluciones es $$S=\{\left( x_0+\frac{b}{mcd(a,b)}k,y_0-\frac{a}{mcd(a,b)}k \right):k\in\mathbb{Z}\}=\{(x_0+b^*k,y_0-a^*k):k\in\mathbb{Z}\}$$
	\end{itemize}}
\begin{myproof}
	Llamemos $d=mcd(a,b)$. Al ser $(a,b)\neq (0,0)$, tenemos que $d\neq 0$.
	\begin{enumerate}
		\item Si la ecuación tiene solución, entonces existen $x_0,y_0\in\mathbb{Z}$ tales que $ax_0+by_0=c$. Como $d|a$ y $d|b$, entonces $d|ax_0+by_0=c$.
		      \\Supongamos ahora que $d|c$ y veamos que la ecuación tiene solución: como $d|c$, existe $e\in\mathbb{Z}$ tal que $c=de$. Por la igualdad de Bezout, existen $x',y'\in\mathbb{Z}$ tales que $ax'+by'=d$. Multiplicando por $e$ tenemos que $a(x'e)+b(y'e)=de=c$, y por lo tanto el par $(x,y)=(x'e,y'e)$ es solución de la ecuación $ax+by=c$.
		\item Sea $(x_0,y_0)$ una solución. Veamos primero que para todo $k\in\mathbb{Z}$, el par $$\left( x_0+\frac{b}{mcd(a,b)}k,y_0-\frac{a}{mcd(a,b)}k \right) $$es solución de la ecuación. Para esto simplemente sustituimos:$$a \left( x_0+\frac{b}{mcd(a,b)}k \right)+b\left( y_0-\frac{a}{mcd(a,b)} \right)=ax_0+\frac{abk}{d}+by_0-\frac{abk}{d}=ax_0+by_0=c  $$donde la ultima igualdad vale porque $(x_0,y_0)$ es solución.\\
		      Veamos ahora que para cualquier solución $(x_1,y_1)$ de la ecuación, existe un $k\in\mathbb{Z}$ tal que\\ $(x_1,y_1)=\left( x_0+\frac{b}{d}k,y_0-\frac{a}{d}k\right)=(x_0+b^*k,y_0-a^*k)$. Al ser $(a,b)\neq(0,0)$ podemos suponer que $b\neq 0$ (y en consecuencia $b^*=\frac{b}{d}\neq0$).\\
		      Sea entonces $(x_1,y_1)$ una solución, tenemos pues que $$\begin{aligned} ax_{1} +by_{1} =c & \ \text{y}\\ ax_{0} +by_{0} =c &  \end{aligned}$$Por lo tanto $ax_1+by_1=ax_0+by_0$, entonces $a(x_1-x_0)=b(y_0-y_1)$. Al ser $d\neq0$, podemos dividir entre $d$ y obtenemos $a^*(x_1-x_0)=b^*(y_0-y_1)$.\\
		      Tenemos en particular que $b^*|a^*(x_1-x_0)$ y como $mcd(a^*,b^*)=1$, por el Lema de Euclides tenemos que $b^*|(x_1-x_0)$. Por lo tanto existe un $k\in\mathbb{Z}$ tal que $x_1-x_0=b^*k$ y por lo tanto $\textcolor{blue}{x_1=x_0+b^k}$. Si ahora sustituimos en la ecuación anterior obtenemos:$$a^*b^*k=b^*(y_0-y_1)$$y como supusimos $b^*\neq0$, cancelando obtenemos $a^*k=y_0-y_1$, y por lo tanto $\textcolor{blue}{y_1=y_0-a^*k}$.
	\end{enumerate}
\end{myproof}\newpage\ex{}{Una barraca vende ladrillos a 12 pesos la unidad y baldozas a 21 pesos cada una. Tenemos 333 pesos y queremos gastarlo todo en baldozas y ladrillos (y que no sobre nada). De cuantas formas podemos hacerlo?\\Si llamamos $x$ a la cantidad de ladrillos que compramos, e $y$ la cantidad de baldozas, tenemos que $x,y\in\mathbb{N}$ y la condición de gastar los 333 pesos se traduce a $$12x+21y=333$$Algunas observaciones:
	\begin{itemize}
		\item La primera es que como tanto 12 y 21 son múltiplos de 3, el dinero que gastamos tendrá que ser múltiplo de 3, es decir, si en vez de 333 pesos quisiéramos gastar 100 pesos, no podríamos hacerlo.
		\item La segunda observación es que como $3=mcd(12,21)$, por la igualdad de Bezout podemos hallar $x',y'\in\mathbb{Z}$ tales que $12x'+21y'=3$.
		\item Por ejemplo $x'=2$ e $y'=-1$ cumplen la ultima ecuación: $12(2)+21(-1)=3$.
		\item Si multiplicamos la ultima igualdad por 111, obtenemos que $12(222)+21(-111)=333$, es decir, que $x=222$ e $y=-111$ verifican la ecuación; pero estos valores de $x$ e $y$ no nos resuelven el problema original, ya que buscamos que $x,y\ge0$. No nos interesa entonces hallar TODOS los pares de enteros $(x,y)$ que son solución, hay que buscar los que no sean negativos.
		\item Como ya tenemos una solución, viendo el ultimo teorema, sabemos que el conjunto solución es $\{(x,y)=(222-7k,-111+4k): k\in\mathbb{Z}\}$, siendo $x_0=222,\ \frac{b}{mcd(a,b)}=\frac{21}{3}=7,\ y_0=-111,$ y $\frac{a}{mcd(a,b)}=\frac{12}{3}=4$.
	\end{itemize}
	Entonces, para terminar de resolver el problema original, necesitamos las soluciones tales que $x=222-7k\ge0$, e $y=-111+4k\ge0$, es decir las soluciones para valores de $k$ tales que $222\ge 7k$ y $4k\ge 111$. O sea, necesitamos $k\in\mathbb{Z}$ con $\frac{111}{4}\le k\le \frac{222}{7}$, así que los valores de $k$ son $k=28,29,30,31$, y por lo tanto, las soluciones al problema son $(x=26,y=1),(x=19,y=5),(x=12,y=9),(x=5,y=13)$.}
\newpage\section{El problema de los sellos}
\mprop{}{Sean $a>1,\ b>1$ enteros, primos entre si. Entonces no hay enteros $x,\ y$ no negativos con $ax+by=ab-a-b$.}
\mprop{}{Sean $a$ y $b$ enteros positivos tales que $mcd(a,b)=1$. Si $n\ge ab-a-b+1$, entonces existen enteros no negativos $x,y$ tales que $ax+by=n$.}
\begin{myproof}
	Por el teorema \ref{thm:Diofanticas} como $mcd(a,b)=1$, existe un par de enteros $(x_0,y_0)$ que cumplen $$ax_0+by_0=n\ge ab-a-b+1$$que nos permite expresar todas las soluciones en la forma $$x=x_0+bk,\ y=y_0-ak,\ k\in\mathbb{Z}$$Usando el algoritmo de división, podemos dividir $y_0$ por $a$ y escribir $y_0=at+y_1$, con $0\le y_1\le a-1$, para algún entero $t$. Probaremos que $x_1=x_0+bt$ es no negativo. Si $x_1\le -1$, entonces, como $y_1\le a-1$,$$\begin{aligned} n & =ax_{0} +by_{0}\\  & =a( x_{1} -bt) +b( y_{1} +at)\\  & =ax_{1} +by_{1}\\  & \le a( -1) +b( a-1)\\  & \le ab-a-b \end{aligned}$$que contradice la hipótesis $n\ge ab-a-b+1$. Concluimos que $(x_1,y_1)$ es una solución de enteros no negativos.
\end{myproof}
\newpage\section{Teorema Fundamental de la Aritmética}
\thm{Teorema Fundamental de la Aritmética}{Sea $n\in\mathbb{N},\ n>1$; entonces
	\begin{enumerate}
		\item Existen primos $p_1,\dots,p_k$ (no necesariamente distintos) con $k\ge 1$, tales que $n=p_1\dots p_k$.
		\item Hay unicidad en la factorización. Es decir, $k$ (la cantidad de factores primos) es único y la lista de primos (contando repeticiones), $p_1,\dots,p_k$ es única.
	\end{enumerate}}
\begin{myproof} \ \\
	\begin{enumerate}
		\item Demostraremos la existencia de la factorización en primos por inducción en $n$.
		      \begin{itemize}
			      \item Si $n=2$, al ser 2 primo, tomando $p_1=2$ tenemos que $2=p_1$.
			      \item Sea $n>2$. Supongamos que las factorizaciones en productos de primos existen para todo natural $m$ con $2\le m<n$ (hipótesis inductiva) y probemoslo para $n$ (tesis inductiva):\\
			            Si $n$ es primo, entonces tomando $p_1=n$ tenemos lo deseado. Si $n$ no es primo, entonces $n$ tiene un divisor positivo $a$, con $1<a<n$. Entonces existe $b\in\mathbb{Z}$ tal que $n=ab$ y luego $1<b<n$. Por lo tanto $a$ y $b$ se encuentran en nuestra hipótesis inductiva, y por lo tanto existen primos $p_1,\dots,p_k$ y $p_1',\dots,p_r'$ tales que $a=p_1\dots p_k$ y $b=p_1'\dots p_r'$. Al ser $n=ab$ tenemos que $n=p_1\dots p_kp_1'\dots p_r'$ y hemos probado la tesis inductiva.
		      \end{itemize}
		\item Para probar la unicidad supongamos que existe un natural $n>1$ que se escribe de dos formas distintas como producto de primos. Podemos considerar $n_0$, el menor natural que verifica lo anterior. Entonces existen primos $p_1,\dots,p_k,q_1,\dots,q_r$ tales que $n_0=p_1\dots p_k,\ n_0=q_1\dots q_r$ con $\{p_1,\dots,p_k\}\neq\{q_1,\dots,q_r\}$ (y como claramente $n_0$ no puede ser primo, tenemos que $k,r\ge2$.)\\
		      Tenemos entonces que $p_1\dots p_k=q_1\dots q_r$ y por lo tanto $p_1|q_1\dots q_r$. Al ser $p_1$ primo, por el corolario \ref{cor:DivPrimos} existe $j\in\{1,\dots,r\}$ tal que $p_1|q_j$; y al ser $p_1>1$ y $q_j$ primo, debe ser $p_1=q_j$. Podemos asumir que $j=1$. Así que ahora tenemos que $p_1\dots p_k=p_1q_2\dots q_r$ y cancelando $p_1$ obtenemos $p_2\dots p_k=q_2\dots q_r$ es un entero $>1$ que se escribe de dos formas distintas como producto de primos, y esto es absurdo ya que $m=\frac{n_0}{p_1}<n_0$ y $n_0$ era el menor entero mayor que uno que se podía escribir de dos formas distintas como producto de primos.
	\end{enumerate}
\end{myproof}
\cor{}{Existen infinitos primos.}
\begin{myproof}
	Supongamos por absurdo que existe una cantidad finita de primos y sea $\{p_1,\dots,p_k\}$ el conjunto de todos los primos. Consideremos el entero $n=p_1p_2\dots p_k+1$. Al ser $n>1$, por el Teorema Fundamental de la Aritmética, $n$ se escribe como producto de primos. En particular, existe algún primo $p$ que divide a $n$, y como supusimos que todos los primos son $\{p_1,\dots,p_k\}$ tenemos que $p_i|n$ para algún $i\in\{1,\dots,k\}$. Tenemos entonces que $p_i|p_1p_2\dots p_k+1$, pero como $p_i|p_1p_2\dots p_k$, tenemos que $p_i|1$ lo cual es absurdo al ser $p_i>1$.
\end{myproof}
\nt{Si en la descomposición de un entero positivo $a$, tomamos primos distintos, entonces estos pueden aparecer con exponentes. Por lo tanto, todo entero $a>1$ se escribe (de forma única, al menos del orden) como $a=p_1^{e_1}p_2^{e_2}\dots p_k^{e_k}$, con $p_i$ primos distintos y $e_i\in\mathbb{Z}^+$.}
\newpage\mprop{}{Sean $a,b$ enteros positivos con descomposición en factores primos $$a=2^{a_2}3^{a_3}5^{a_5}\text{ y }b=2^{b_2}3^{b_3}5^{a_5}$$ entonces:
\begin{enumerate}
	\item $a|b$ si y solo si $a_p\le b_p$ para todo $p$ (cabe aclarar que estamos notando como $a=p^{a_p}$ como lo hice arriba)
	\item $mcd(a,b)=2^{d_2}3^{d_3}5^{d_5}\dots$ siendo $d_p=min\{a_p,b_p\}$ para todo primo $p$.
	\item $mcm(a,b)=2^{m_2}3^{m_3}5^{m_5}\dots$ siendo $m_p=max\{a_p,b_p\}$ para todo primo $p$.
\end{enumerate}}
\begin{myproof}
	\ \\
	\begin{enumerate}
		\item Si $a|b$, existe $c\in\mathbb{Z}^+$ tal que $ac=b$. Escribimos $c=2^{c_2}3^{c_3}5^{c_5}\dots$ y tenemos $$2^{a_2+c_2}3^{a_3+c_3}5^{a_5+c_5}\dots=ac=b=2^{b_2}3^{b_3}5^{b_5}\dots$$Por la unicidad de la descomposición factorial debe ser $a_p+c_p=b_p$ para todo primo $p$ y en particular (al ser $c_p\ge0$) $a_p\le b_p$.
		\item Por lo visto en la parte anterior, tenemos que $$Div_+(a)=\{c=2^{c_2}3^{c_3}5^{c_5}\dots\text{ con }0\le c_p\le a_p,\ \forall p\}$$$$Div_+(b)=\{c=2^{c_2}3^{c_3}5^{c_5}\dots\text{ con }0\le c_p\le b_p,\ \forall p\}$$Por lo tanto, los divisores comunes de $a$ y $b$ son:$$\begin{aligned} Div_{+}( a) \cap Div_{+}( b) & =\left\{c=2^{c_{2}} 3^{c_{3}} 5^{c_{5}} \dotsc \text{ con } 0\leq c_{p} \leq a_{p} ,\text{ y } c_{p} \leq b_{p} ,\ \forall p\right\}\\  & =\left\{c=2^{c_{2}} 3^{c_{3}} 5^{c_{5}} \dotsc \text{ con } 0\leq c_{p} \leq min\{a_{p} ,b_{p}\} ,\ \forall p\right\} \end{aligned}$$El máximo de este conjunto es claramente $c=2^{d_2}3^{d_3}5^{d_5}\dots$ siendo $d_p=min\{a_p,b_p\}$ para cada primo $p$.
		\item Se deduce de la parte anterior y del hecho de que para enteros positivos $a$ y $b$ se tiene que $mcm(a,b)=\frac{ab}{mcd(a,b)}$.
	\end{enumerate}
\end{myproof}
\cor{}{\label{cantdivpos}Sea $n=p_1^{e_1}p_2^{e_2}\dots p_k^{e_k}$, con $p_i$ primos distintos y $e_i\in\mathbb{Z}^+$. Entonces:
\begin{enumerate}
	\item $Div_+(n)=\{p_1^{c_1}p_2^{c_2}\dots p_k^{c_k}:\ c_i\in\mathbb{N}\text{ y }c_i\le e_i,\ \forall i=1,\dots,k\}$.
	\item La cantidad de divisores positivos de $n$ es $\#Div(n)=(e_1+1)(e_2+1)\dots(e_k+1)$.
	\item El entero $n$ es un cuadrado perfecto (es decir, existe $m\in\mathbb{Z}$ tal que $n=m^2$) si y solo si $2|e_i\ \forall i=1\dots,k$.
	\item Existe $m\in\mathbb{Z}^+$ y $k\in\mathbb{Z}^+$ tales que $n=m^k$ si y solo si, todos los $e_i$ son múltiplos de $k$.
\end{enumerate}}\newpage
\chapter{Congruencias}
\section{Definiciones y primeras propiedades}
\dfn{}{Fijado $n\in\mathbb{Z}$, y dados $a,b\in\mathbb{Z}$, decimos que $a$ es congruente con $b$ módulo $n$ y escribimos $a\equiv b$ (mod $n$) si $n|a-b$. En caso contrario, escribiremos $a\nequiv b$ (mod $n$).}
\mprop{}{
\begin{enumerate}
    \item La congruencia módulo $n$ es una relación de equivalencia.
    \item $a\equiv b$ (mod $n$) si y solo si $a\equiv b$ (mod $(-n)$).
    \item $a\equiv b$ (mod $n$) si y sólo si $a$ y $b$ tienen el mismo resto al dividirlos entre $n$.
    \item Dado $n\in\mathbb{Z}^+$, y $a\in\mathbb{Z}$ existe un único $r\in\{0,1,\dots,n-1\}$ tal que $a\equiv r$ (mod $n$) ($r$ es el resto de dividir $a$ entre $n$).
\end{enumerate}}
\ex{Propiedad cancelativa}{Observemos por ejemplo que $6\equiv 16$ (mod $5$); es decir, $2\times 3\equiv 2\times 8$ (mod $5$). En este caso podemos cancelar el $2$ ya que claramente $3\equiv 8$ (mod $5$).\\Ahora, porque podemos cancelar el $2$?\\
La congruencia $6\equiv 16$ (mod $5$) es cierta pues $5|(16-6)$; factorizando el $2$, tenemos que $5|2(8-3)$, y como $5$ y $2$ son coprimos, por el Lema de Euclides, obtenemos entonces $5|(8-3)$ y por lo tanto $3\equiv 8$ (mod $5$). Aquí utilizamos que $mcd(5,2)=1$; esto es absolutamente necesario para poder cancelar y obtener una congruencia con el mismo modulo.}
\ex{}{Observar que $5\equiv 10$ (mod $5$); es decir $5\times 1\equiv 5\times 2$ (mod $5$) y sin embargo $1\nequiv 2$ (mod $5$). Aquí no podemos cancelar el $5$ pues el hecho de que $5|(10-5)=5(2-1)$ no implica que $5|(2-1)$.}
\newpage\mprop{Propiedades Cancelativas}{Sea $a,b,c,n\in\mathbb{Z}$ con $c\neq 0$.
\begin{enumerate}
    \item Si $ca\equiv cb$ (mod $n$) y $mcd(c,n)=1$ entonces $a\equiv b$ (mod $n$).
    \item Si $c|n$ y $ca\equiv cb$ (mod $n$) entonces $a\equiv b$ (mod $\frac{n}{c}$).
    \item Si $ca\equiv cb$ (mod $n$) entonces $a\equiv b$ (mod $\frac{n}{mcd(c,n)}$).
\end{enumerate}}
\begin{myproof}\ \\
    \begin{enumerate}
        \item Tenemos que $ca\equiv cb$ (mod $n$), es decir $n|(ca-cb)$. Entonces $n|c(a-b)$ y como $mcd(c,n)=1$ por el Lema de Euclides obtenemos que $n|(a-b)$ y por lo tanto $a\equiv b$ (mod $n$).
        \item Si $c|n$, existe un $k\in\mathbb{Z}$ tal que $n=ck$. Si ademas $ca\equiv cb$ (mod $n$) entonces $ck=n|c(a-b)$. Por lo tanto existe $e\in\mathbb{Z}$ tal que $c(a-b)=cke$, y como $c\neq 0$, por la cancelativa en $\mathbb{Z}$ tenemos que $a-b=ke$. Por lo tanto $k|(a-b)$ y entonces $a\equiv b$ (mod $k$), es decir $a\equiv b$ (mod $\frac{n}{c}$).
        \item Si llamamos $d=mcd(c,n)$ tenemos que $c=dc^*$ y $n=dn^*$, con $c^*,n^*$ enteros coprimos. Si $ca\equiv cb$ (mod $n$), entonces $dc^*a=dc^*b$ (mod $dn^*$), y por la parte anterior tenemos que $c^*a\equiv c^*b$ (mod $n^*$). Ahora como $mcd(c^*,n^*)=1$, utilizando la primer parte para estos enteros obtenemos que $a\equiv b$ (mod $n^*$); es decir $a\equiv b$ (mod $\frac{n}{mcd(c,n)}$).
    \end{enumerate}
\end{myproof}
\newpage\section{Algunas aplicaciones}
\mprop{}{Sean $a,b,c,n,m\in\mathbb{Z}$.
\begin{enumerate}
    \item $a\equiv b$ (mod $n$) y $c\equiv b$ (mod $n$)$\Rightarrow a+c\equiv b+d$ (mod $n$) y $ac\equiv bd$ (mod $n$).
    \item $b\equiv c$ (mod $n$)$\Rightarrow a+b\equiv a+c$ (mod $n$).
    \item $a\equiv b$ (mod $n$) y $m|n\Rightarrow a\equiv b$ (mod $m$).
    \item $a\equiv b$ (mod $m$)$\Rightarrow na\equiv nb$ (mod $m$).
    \item $a\equiv b$ (mod $m$) y $n\in\mathbb{N}\Rightarrow a^n\equiv b^n$ (mod $m$).
\end{enumerate}}
\subsection{Criterios de divisibilidad}
\mprop{}{Si los dígitos de $a$ son $a=a_k\dots a_1a_0$. Entonces $3|a$ si y sólo si $3|a_0+a_1+\dots+a_k$.}
\begin{myproof}
    Tenemos que $a=a_k10^k+\dots a_110+a_0$. Tenemos que $3|a$ si y sólo si $a\equiv 0$ (mod $3$); es decir, si y solo si $a_k10^k+\dots+a_110+a_0\equiv 0$ (mod $3$).\\
    Ahora $10\equiv 1$ (mod $3$), y entonces (por la ultima propiedad de la proposición anterior) $10^i\equiv 1^i$ (mod $3$) para todo $i\in\mathbb{N}$. Así que $10^i\equiv 1$ (mod $3$) y por lo tanto para todo $i=0,\dots k$ tenemos que $a_i 10^i\equiv a_i$ (mod $3$) (por la propiedad $(4)$); y sumando, utilizando la propiedad $(1)$ obtenemos que $a=a_k10^k+\dots+a_110+a_0\equiv a_k+\dots+a_1+a_0$ (mod $3$).\\
    Entonces (por la transitividad de la congruencia) $a\equiv 0$ (mod $3$)$\Leftrightarrow a_k+\dots+a_1+a_0\equiv 0$ (mod $3$); es decir $3$ divida a $a$, si y sólo si $3$ divide a la suma de sus dígitos.
\end{myproof}
\mprop{}{Si los dígitos de $a$ son $a=a_k\dots a_1a_0$. Entonces $9|a$ si y sólo si $9|a_0+a_1+\dots+a_k$.}
\newpage\subsection{Dígitos Verificadores}
\dfn{Código ISBN}{El ISBN (International Standard Book Number) es una cadena de diez símbolos que identifica a los libros. Los primeros nueve símbolos con dígitos, y el ultimo el símbolo verificador.\\Es entonces una cadena $x_1,x_2,\dots,x_9-x_{10}$, donde cada $x_1,x_2,\dots,x_9$ es un dígito de $0$ a $9$, mientras que $x_{10}\in\{0,1,2,3,4,5,6,7,8,9,X\}$. Al símbolo $x_{10}$ se llama el símbolo verificador y se calcula de la siguiente manera: $$c=\sum_{i=1}^{9} i\cdot x_i$$y sea $r\in\{0,1,\dots,10\}$ tal que $r\equiv c$ (mod $11$) (es decir, $r$ el resto de dividir $c$ entre 11). Entonces:$$x_{10} =\begin{cases} r & \text{si } 0\le  r\le  9\\ X & \text{si } r=10 \end{cases}$$}
\mprop{}{Sean $x_1x_2\dots x_9-x_{10}$ y $y_1y_2\dots y_9-y_{10}$ dos códigos ISBN. Sea $k$ un entero tal que:
\begin{itemize}
    \item $1\le k\le 9$.
    \item $x_k\neq y_k$.
    \item $x_i=y_i$ para todo $i\le 9, i\neq k$.
\end{itemize}
Entonces $x_{10}\neq y_{10}$.}
\begin{myproof}
    Supongamos que $x_{10}=y_{10}$; entonces tendríamos que $$\sum_{i=1}^{9} i\cdot x_i\equiv\sum_{i=1}^{9} i\cdot y_{10}\text{ (mod $11$)}$$
    Pero en estas sumas tenemos que para $i\neq k,\ i\cdot x_i=i\cdot y_i$, y por lo tanto cancelando tendríamos que $$k\cdot x_k\equiv k\cdot y_k\text{ (mod $11$)}$$y como $mcd(k,11)=1$, por la propiedad cancelativa tendríamos que $x_k\equiv y_k$ (mod $11$) lo cual es absurdo pues $x_k\neq y_k$ y son números entre $0$ y $9$. 
\end{myproof}
\newpage\section{Ecuaciones con congruencias}
\thm{}{Dados $a,b,n\in\mathbb{Z}$ y sea $d=mcd(a,n)$. Entonces la ecuación $ax\equiv b$ (mod $n$) tiene solución si y sólo si $d|b$. Ademas, si $d|b$ existen exactamente $d$ soluciones distintas al modulo $n$.}
\begin{myproof}
    Tenemos que $ax\equiv b$ (mod $n$) si y sólo si $n|(ax-b)$, si y sólo si $ax-b=ny$ para algún $y\in\mathbb{Z}$. Por lo tanto, la ecuación $ax\equiv b$ (mod $n$) tiene solución, si y solo si existen $x,y\in\mathbb{Z}$ tales que $ax-ny=b$. Por el Teorema de Ecuaciones Diofánticas, sabemos que esto sucede si y sólo si $d|b$.\\
    Ahora, en el caso que $d|b$, si $(x_0,y_0)$ es solución de la ecuación diofántica, tenemos (por el mismo teorema) que el conjunto de soluciones de la diofántica es $\{\left( x,y \right)=\left( x_0+\frac{n}{d}k,y_0+\frac{a}{d}k;\ k\in\mathbb{K} \right)\}$. Por lo tanto, las soluciones de la ecuación $ax\equiv b$ (mod $n$) son $x=x_0+\frac{n}{d}k$, con $k\in\mathbb{Z}$.\\
    Observar que $x_0,x_1=x_0+\frac{n}{d},x_2=x_0+2\frac{n}{d},\dots,x_{d-1}=x_0+(d-1)\frac{n}{d}$ son $d$ soluciones que no son congruentes entre ellas modulo $n$. Esto es porque si $i\neq j,\ 0\neq |x_i-x_j|=|x_0+i\frac{n}{d}-x_0-j\frac{n}{d}|=|(i-j)\frac{n}{d}|\le (d-1)\frac{n}{d}<n$; por lo tanto $n\nmid x_i-x_j$ y entones $x_i\nequiv x_j$ (mod $n$). Veamos ahora que cualquier otra solución es congruente (modulo $n$) a una de estas.\\
    Si $x=x_0+\frac{n}{d}k$, dividiendo $k$ entre $d$, tenemos que $k=dq+i$ con $0\le i<d$, y por lo tanto $x=x_0+\frac{n}{d}k=x_0+\frac{n}{d}(dq+i)=x_0+i\frac{n}{d}+qn=x_i+qn\equiv x_i$ (mod $n$).
\end{myproof}
\dfn{}{Decimos que un entero $a$ es invertible modulo $n$, si existe otro entero $x$ tal que $ax\equiv 1$ (mod $n$). Al entero $x$ se le llama inverso de $a$ modulo $n$.}
\cor{}{Un entero $a$ es invertible modulo $n$ si y sólo si $mcd(a,n)=1$. Ademas, si $a$ es invertible, el inverso de $a$ modulo $n$ es único modulo $n$.}
\newpage\section{Teorema Chino del Resto}
\thm{Teorema Chino del Resto}{Sean $m_1,m_2,\dots,m_n$ enteros coprimos dos a dos y $a_1,a_2,\dots,a_k\in\mathbb{Z}$. Entonces el sistema $$\begin{cases} x\equiv a_{1} & ( mod\ m_{1})\\ x\equiv a_{2} & ( mod\ m_{2})\\ \ \ \ \vdots  & \\ x\equiv a_{k} & ( mod\ m_{k}) \end{cases}$$tiene solución, y hay una única solución modulo $m_1m_2\dots m_k$. Es decir, si $x_0$ es solución, entonces todas las soluciones son $x\equiv x_0\ (mod\ m_1m_2\dots m_k)$.}
\begin{myproof}
    Haremos la demostración por inducción en $k$ (la cantidad de ecuaciones) partiendo del caso $k=2$. Consideremos dos enteros $m_1,m_2$ coprimos $a_1,a_2\in\mathbb{Z}$ y el sistema $$\begin{cases} x\equiv a_{1} & ( mod\ m_{1})\\ x\equiv a_{2} & ( mod\ m_{2}) \end{cases}$$La primer congruencia equivale a que exista $s\in\mathbb{Z}$ tal que $$x=a_1+m_1s$$ y la segunda equivale a que exista $t\in\mathbb{Z}$ tal que $$x=a_2+m_2t$$Por lo tanto, debemos encontrar $x\in\mathbb{Z}$ que verifique estas dos ultima condiciones, es decir, que existe $x\in\mathbb{Z}$ que verifica las congruencias si y solo si $$\exists s,t\in\mathbb{Z}:\ a_1+m_1s=a_2+m_2t$$Es decir, si y solo si $$\exists s,t\in\mathbb{Z}:\ m_1s-m_2t=a_2-a_1$$Como $mcd(m_1,m_2)=1$, por el teorema de Ecuaciones Diofánticas, esta ultima ecuación tiene solución.\\
    Ademas, dada una particular $(s_0,t_0)$, todas las soluciones de la diofántica son $(s,t)=(s_0+m_2k,t_0+m_1k)$ tal que $k\in\mathbb{Z}$. Ahora sustituyendo el $s$ de estas soluciones, obtenemos que $x=a_1+m_1s=a_1+m_1(s_0+m_2k)=a_1+m_1s_0+m_1m_2k,\ k\in\mathbb{Z}$.\\
    Si llamamos $x_0=a_1+m_1s_0$, tenemos que las soluciones de las dos congruencias son $$x=x_0+m_1m_2k,\ k\in\mathbb{Z}$$Es decir, que el sistema tiene solución $x_0$ y todas las soluciones son $x\equiv x_0\ (mod\ m_1m_2)$.\\
    Así que obtuvimos una única solución módulo $m_1m_2$.\\
    Ahora, el paso inductivo: sea $k>2$ y asumamos que el teorema es cierto para $k-1$, probemos que es cierto para $k$ ecuaciones. Por la hipótesis inductiva tenemos que el sistema $$\begin{cases} x\equiv a_{1} & ( mod\ m_{1})\\ x\equiv a_{2} & ( mod\ m_{2})\\ \ \ \ \vdots  & \\ x\equiv a_{k-1} & ( mod\ m_{k-1}) \end{cases}$$ tiene solución $x_1$ y que ademas cualquier solución cumple que $x\equiv x_1\ (mod\ m_1m_2\dots m_{k-1})$; por lo tanto, este sistema con $k-1$ ecuaciones es equivalente a la ecuación $x\equiv x_1\ (mod\ m_1m_2\dots m_{k-1})$.\\Entonces el sistema con $k$ ecuaciones $$\begin{cases} x\equiv a_{1} & ( mod\ m_{1})\\ x\equiv a_{2} & ( mod\ m_{2})\\ \ \ \ \vdots  & \\ x\equiv a_{k-1} & ( mod\ m_{k-1})\\ x\equiv a_{k} & ( mod\ m_{k}) \end{cases}$$es equivalente al sistema $$\begin{cases} x\equiv x_{1} & ( mod\ m_{1} m_{2} \dotsc m_{k-1})\\ x\equiv a_{k} & ( mod\ m_{k}) \end{cases}$$Como los enteros $m_1,m_2\dots,m_k$ son coprimos dos a dos, tenemos que $mcd(m_1m_2\dots m_{k-1},m_k)=1$. Por lo tanto tenemos un sistema con $2$ ecuaciones que involucran módulos coprimos. Por lo ya probado para $k=2$, tenemos entonces que este ultimo sistema tiene solución $x_0\in\mathbb{Z}$ y ademas que toda solución cumple $x\equiv x_0\ (mod(m_1m_2\dots m_{k-1})\cdot k)$.\\Por lo tanto el sistema tiene solución $x_0$, y las soluciones son $x\equiv x_0\ (mod\ m_1m_2\dots m_{k-1}m_k)$; es decir, la solución es única módulo $m_1m_2\dots m_k$.
\end{myproof}
\cor{}{Generalizando el teorema, tenemos que si $m_1,\dots,m_k$ no son coprimos dos a dos, entonces el sistema puede o no tener solución.\\En caso de que tenga una solución $x_0$, todas las soluciones son $$x\equiv x_0\ (mod\ mcm(m_1,m_2,\dots,m_k))$$}
\newpage\section{Exponenciación y Teoremas de Fermat y Euler}
\dfn{}{La función de Euler es $\varphi:\mathbb{Z}^+\to\mathbb{Z}^+$ dada por $$\varphi (n)=\# \{a\in\{1,\dots,n\}:mcd(a,n)=1\}$$Es decir, que la función de Euler cuenta la cantidad de naturales coprimos con $n$ y menores que $n$.}
\mprop{}{Si $p$ es primo, entonces $\varphi(p)=\#\{1,2,\dots,p-1\}=p-1$.}
\mprop{}{Si $p$ es primo, obtengamos una formula para obtener $\varphi(p^k)$, tenemos que $$\varphi(p^k)=\#\{a\in\{1,2,\dots,p^k\}:mcd(a,p^k)=1\}=\#\{a\in\{1,2,\dots,p^k\}:mcd(a,p)=1\}$$Entonces:$$\varphi(p^k)=\#\{1,2,\dots,p^k\}-\#\{a\in\{1,2,\dots,p^k\}:mcd(a,p)\neq 1\}$$Por lo tanto$$\varphi(p^k)=p^k-\#\{a\in\{1,2,\dots,p^k\}:mcd(a,p)\neq 1\}$$Ahora, como $p$ es primo, tenemos que $mcd(a,p)\neq 1\Leftrightarrow a=pk$ para algún $k\in\mathbb{Z}$. Por lo tanto $\{a\in\{1,\dots,p^k\}:mcd(a,p)\neq 1\}=\{a=pk$ con $k\in\{1,2,\dots,p^{k-1}\}\}$ y el cardinal de este conjunto es $p^{k-1}$.\label{fphiprimo} Sustituyendo obtenemos que $$\varphi(p^k)=p^k-p^{k-1}=p^k\left( 1-\frac{1}{p} \right) $$}
\thm{}{Si $mcd(m,n)=1,\ \varphi(mn)=\varphi(m)\varphi(n)$.}
\begin{myproof}
    Como la tesis es obvia si $m$ o $n$ es $1$, demostremoslo para $m,n>1$. La idea de la demostración es la siguiente: daremos dos conjuntos $C$ y $D$ tales que tales que $\#C=\varphi(mn)$ y $\#D=\varphi(m)\varphi(n)$, y lego construiremos una función biyectiva $f:C\to D$ lo cual terminaría probando que $\#C=\#D$; es decir que $\varphi(mn)=\varphi(m)\varphi(n)$.\\
    Sea $C=\{c\in\{0,\dots,mn-1\}:mcd(c,mn)=1\}$; claramente $\#C=\varphi(mn)$. Ademas, tenemos que $$mcd(c,mn)=1\Leftrightarrow mcd(c,m)=1\text{ y }mcd(c,n)=1$$Así que $C=\{c\in\{0,\dots,mn\}:mcd(c,m)=1$ y $mcd(c,n)=1\}$.\\
    Sea $A=\{a\in\{0,\dots,m-1\}:mcd(a,m)=1\}$ y $B=\{b\in\{0,\dots,n-1\}:mcd(b,n)=1\}$; tenemos que $\#A=\varphi(m)$ y $\#B=\varphi(n)$ y por lo tanto si $D=A\times B=\{(a,b):a\in A,\ b\in B\}$ tenemos que $\#D=\varphi(m)\varphi(n)$.\\Consideramos ahora la función $f:C\to D$ dada por $f(c)=(a,b)$ siendo $a$ el resto de dividir $c$ entre $m$ y $b$ el resto de dividir $c$ entre $n$. Es decir $f(c)=(a,b)$ con $a\in\{0,\dots,m-1\}, b\in\{0,\dots,n-1\}$ y $$\begin{cases} c\equiv a & ( mod\ m)\\ c\equiv b & ( mod\ n) \end{cases}$$Veamos primero que efectivamente, si $c\in C$ y $f(c)=(a,b)$, entonces $(a,b)\in D$. Como $c=mq+a$ y $c=nq'+b$ tenemos que$$mcd(c,m)=mcd(a,m)\text{ y } mcd(c,n)=mcd(b,n)$$Por lo tanto si $mcd(c,m)=1$ y $mcd(c,n)=1$ tenemos que $mcd(a,m)=1$ y $mcd(b,n)=1$. Como ademas claramente $a\in\{0,\dots,m-1\}$ y $b\in\{0,\dots,n-1\}$ concluimos que $(a,b)\in D$.\\Veamos ahora que la función $f$ es biyectiva. Para esto tenemos que ver que dado $(a,b)\in D$, existe un único $c\in C$ tal que $f(c)=(a,b)$ (la existencia de $c$ nos da la sobreyectividad de $f$ y la unicidad nos da la inyectividad de $f$). Tenemos que probar entonces que dado $(a,b)\in D$ existe un único $c\in C$ tal que$$\begin{cases} c\equiv a & ( mod\ m)\\ c\equiv b & ( mod\ n) \end{cases}$$Como $mcd(m,n)=1$, por el Teorema Chino del resto sabemos que el sistema tiene solución $c_0$, todas las soluciones son $c\equiv c_0\ (mod mn)$. Por lo tanto, existe un único $c\in\{0,\dots,mn-1\}$ que verifica el sistema. Resta ver que efectivamente este $c\in C$: como $mcd(a,m)=1$, $mcd(b,n)=1$ y $c\equiv a\ (mod m)$, $c\equiv b\ (mod n)$, tenemos que $$mcd(c,m)=1\text{ y }mcd(c,n)=1$$y por lo tanto $c\in C$.
\end{myproof}
\cor{}{Sea $n\in\mathbb{Z}^+$
\begin{enumerate}
    \item Si $n$ tiene descomposición factorial $n=p_1^{e_1}p_2^{e_2}\dots p_k^{e_k}$ (con los $p_i$ primos distintos y $e_i>0$), entonces:$$\varphi(n)=\left( p_1^{e_1}-p_1^{e_1-1} \right)\left( p_2^{e_2}-p_2^{e_2-1} \right)\dots \left( p_k^{e_k}-p_k^{e_k-1} \right)$$
    \item $\varphi(n)=n \prod _{p\ \text{primo,} \ p|n}\left( 1-\frac{1}{p}\right)$
\end{enumerate}
}
\begin{myproof}
    \begin{enumerate}
        \item Como los $p_i$ son primos distintos, tenemos que los $p_i^{e_i}$ son coprimos 2 a 2, y por lo visto en el teorema anterior, reiteradas veces obtenemos que$$\varphi(n)=\varphi(p_1^{e_1}p_2^{e_2}\dots p_k^{e_k})=\varphi(p_1^{e_1})\varphi(p_2^{e_2})\dots\varphi(p_k^{e_k})$$y utilizando la formula \ref{fphiprimo} obtenemos lo deseado.
        \item Como cada $(p_i^{e_i}-p_i^{e_i-1})=p_i^{e_i}\left( 1-\frac{1}{p_i} \right) $ sustituyendo en la formula recién obtenida nos queda que $$\begin{aligned} \varphi ( n) & =p_{1}^{e_{1}}\left( 1-\frac{1}{p_{1}}\right) p_{2}^{e_{2}}\left( 1-\frac{1}{p_{2}}\right) \dotsc p_{k}^{e_{k}}\left( 1-\frac{1}{p_{k}}\right)\\  & =p_{1}^{e_{1}} p_{2}^{e_{2}} \dotsc p_{k}^{e_{k}}\left( 1-\frac{1}{p_{1}}\right)\left( 1-\frac{1}{p_{2}}\right) \dotsc \left( 1-\frac{1}{p_{k}}\right)\\  & =n\prod _{p\text{ primo,} \ p|n}\left( 1-\frac{1}{p}\right) \end{aligned}$$
    \end{enumerate}
\end{myproof}
\thm{Teorema de Euler}{Sean $n,a\in\mathbb{Z}$ tales que $mcd(a,n)=1$, entonces $$a^{\varphi(n)}\equiv 1\ (mod\ n)$$}
\begin{myproof}
    Sea $B=\{b\in\{1,\dots,n\}:mcd(b,n)=1\}$; claramente $\#B=\varphi(n)$. Observar que si $b\in B$ en particular $mcd(b,n)=1$, y como $mcd(a,n)=1$ tenemos que tambien $mcd(ab,n)=1$. Por lo tanto (tomando el resto de dividir $ab$ entre $n$), existe un unico $b'\in B$ tal que $ab\equiv b'\ (mod\ n)$. Ademas, dados dos elementos distintos de $B$, $b_1$ y $b_2$, al multiplicarlos por $a$ obtenemos enteros no congruentes modulo $n$, ya que si $ab_1\equiv ab_2\ (mod\ n)$, al ser $mcd(a,n)=1$ podemos cancelar $a$ y obtendriamos $b_1\equiv b_2\ (mod\ n)$, lo cual es absurdo ya que en $B$ no hay dos elementos congruentes modulo $n$. Por lo tanto, si multiplicamos por $a$ a todos los elementos de $B$, y luego tomamos los restos de dividir entre $n$, volvemos a obtener todos los elementos de $B$ (permutados).\\Entonces $$\prod _{b\in B} ab\equiv \prod _{b'\in B} b'\ ( mod\ n) \Rightarrow \prod _{b\in B} ab\equiv \prod _{b\in B} b\ ( mod\ n)$$En la izquierda, el factor $a$ aparece $\#B=\varphi(n)$ veces, por lo que obtenemos$$a^{\varphi ( n)}\prod _{b\in B} b\equiv \prod _{b\in B} b\ ( mod\ n)$$y como cada $b\in B$ es coprimo con $n$, no lo podemos cancelar de la congruencia y obtenemos$$a^{\varphi(n)}\equiv 1\ (mod\ n)$$
\end{myproof}
\cor{Teorema de Fermat}{Si $p$ es primo y $a\in\mathbb{Z}$ tal que $p\nmid a$, entonces$$a^{p-1}\equiv 1\ (mod\ p)$$}
\cor{}{Sean $a,n$ dos enteros coprimos
\begin{itemize}
    \item Si $m\in\mathbb{Z}$ y $m=\varphi(n)q+r$ entonces $a^m\equiv a^r\ (mod\ n)$.
    \item Si $m\equiv k\ (mod\ \varphi(n))$ entonces $a^m\equiv a^k\ (mod\ n)$.
\end{itemize}
}
\begin{myproof}
    \begin{enumerate}
    \item Si $m=\varphi(n)q+r$ entonces $$a^m=a^{\varphi(n)q+r}=\left( a^{\varphi(n)} \right)^q a^r\equiv 1^qa^r\ (mod\ n)\equiv a^r\ (mod\ n) $$
            \item Es claro a partir de lo anterior.
    \end{enumerate}
\end{myproof}\newpage\chapter{Teoría de Grupos}\section{Definición y propiedades}
\dfn{}{Un grupo es un conjunto $G$ con una operación binaria $*:G\times G\to G$ tal que
\begin{itemize}
    \item (asociativa) $x*(y*z)=(x*y)*z\ \forall x,y,z\in G$.
    \item (neutro) existe un elemento $e\in G$ tal que $e*x=x$ y $x*e=x\ \forall x\in G$.
    \item (inverso) para todo elemento $g\in G$, existe $g'\in G$ tal que $g*g'=e$ y $g'*g=e$.
\end{itemize}
En general escribimos al grupo como $(G,*)$ o $(G,*,e)$. Si la operación y neutro son claros simplemente notamos $G$.
}
\mprop{}{Sea $(G,*)$ un grupo y $g,h\in G$. Entonces:
\begin{enumerate}
    \item El neutro de $G$ es único.
    \item $\forall g\in G$, el inverso de $g$ es único (y lo escribimos $g^{-1}$; si la operación es una suma, generalmente lo llamamos opuesto y lo escribimos $-g$).
    \item Si $e$ es el neutro de $G$, entonces $e^{-1}=e$.
    \item El inverso de $g^{-1}$ es $g$.
    \item $(gh)^{-1}=h^{-1}g^{-1}$.
    \item Propiedad cancelativa a derecha: si $g,x,h\in G$ y $gx=hx$, entonces $g=h$.
    \item Propiedad cancelativa a izquierda: si $g,x,h\in G$, y $xg=xh$, entonces $g=h$.
    \item Soluciones de ecuaciones a derecha: si $g,h\in G$, entonces existe un único $x\in G$ tal que $gx=h$.
    \item Soluciones de ecuaciones a izquierda: si $g,h\in G$, entonces existe un único $x\in G$ tal que $xg=h$.
    \item (un inverso a izquierda es el inverso) Si $g'*g=e$ entonces $g'=g^{-1}$.
    \item (un inverso a derecha es el inverso) Si $g*g'=e$ entonces $g'=g^{-1}$.
\end{enumerate}
}\section{Grupos de permutacion}
\dfn{}{Un grupo de permutaciones es un conjunto de funciones que reordenan los elementos de un conjunto finito y que, al componerlas, siguen siendo permutaciones del mismo conjunto. El grupo de permutaciones de un conjunto finito de $n$ elementos se denota como $S_n$, es decir, para cada $n\in\mathbb{Z}^+$ llamamos$$S_n=\{f:\{1,2,\dots,n\}\to\{1,2,\dots,n\}:f\text{ es una función biyectiva}\}$$Y ademas, $|S_n|=n!$}
\ex{}{Si $n=2$, en $S_2$ tenemos dos funciones, $Id$ (la función identidad) y la función $f$ tal que $f(1)=2$ y $f(2)=1$.}
\mprop{}{$\left( S_n,\circ,Id \right)$ es un grupo.}
Utilizaremos la siguiente notación: a una función en $S_n$ la escribiremos como una matriz, cuya primera fila consta de los números del $1$ al $n$ en orden, y en su segunda fila escribiremos $f(1),f(2),\dots,f(n)$.
\ex{}{$S_{2} =\left\{Id=\begin{pmatrix} 1 & 2\\ 1 & 2 \end{pmatrix} ,\tau =\begin{pmatrix} 1 & 2\\ 2 & 1 \end{pmatrix}\right\}$.\\Observar en este caso que $\tau\circ\tau=Id$.}
\ex{}{$\begin{aligned} S_{3} & =\{Id=\begin{pmatrix} 1 & 2 & 3\\ 1 & 2 & 3 \end{pmatrix} ,\tau _{1} =\begin{pmatrix} 1 & 2 & 3\\ 1 & 3 & 2 \end{pmatrix} ,\tau _{2} =\begin{pmatrix} 1 & 2 & 3\\ 3 & 2 & 1 \end{pmatrix} ,\tau _{3} =\begin{pmatrix} 1 & 2 & 3\\ 2 & 1 & 3 \end{pmatrix}\\  & \sigma _{1} =\begin{pmatrix} 1 & 2 & 3\\ 2 & 3 & 1 \end{pmatrix} ,\sigma _{2} =\begin{pmatrix} 1 & 2 & 3\\ 3 & 1 & 2 \end{pmatrix}\} \end{aligned}$\\En este caso por ejemplo, $\tau_1\circ\tau_2=\sigma_1$ y $\tau_2\circ\tau_1=\sigma_2$ y por lo tanto $S_3$ no es abeliano. En general, si $n\ge3\ S_n$ no es abeliano.}
\newpage\section{Tablas de Cayley}
Para grupos de orden finito puede resultar conveniente escribir la tabla de multiplicación. A esta tabla se la conoce como Tabla de Cayley del grupo, y se construye de la siguiente manera: se colocan los elementos de $G$ arriba de la tabla, y en el mismo orden también a la izquierda de la tabla; luego en la entrada correspondiente a la fila del elemento $g$ y a la columna del elemento $h$ colocamos $g*h$.
\ex{}{La tabla de Cayley de $S_2$ es \[ \begin{tabular}{c|c c} $\circ$ & Id & $\tau$ \\ \hline Id & Id & $\tau$ \\ $\tau$ & $\tau$ & Id \\ \end{tabular} \] }
\ex{}{Algunas de las entradas de la Tabla de Cayley de $S_3$ son \[\begin{tabular}{c|c c c c c c} $\circ$ & Id & $\tau_1$ & $\tau_2$ & $\tau_3$ & $\sigma_1$ & $\sigma_2$ \\ \hline Id & Id & $\tau_1$ & $\tau_2$ & $\tau_3$ & $\sigma_1$ & $\sigma_2$ \\ $\tau_1$ & $\tau_1$ & Id & $\sigma_1$ & & &\\ $\tau_2$ & $\tau_2$ & & Id & & \\ $\tau_3$ & $\tau_3$ & & & Id  & \\ $\sigma_1$ & $\sigma_1$ & & & & $\sigma_1$ & Id\\ $\sigma_2$ & $\sigma_2$ & & & & Id & $\sigma_1$\\ \end{tabular}\]}
\mprop{}{En la tabla de Cayley de un grupo, cada elemento de $G$ aparece exactamente una vez en cada fila y columna. Es decir, que cada columna y cada fila de la tabla es una premutacion de los elementos de $G$.}
\begin{myproof}
    El elemento $h$ aparece en la fila correspondiente a $g$ y en la columna correspondiente a $x$, si y solo si $gx=h$. Ya vimos que dados $g$ y $h$ en $G$ existe un único $x\in G$ tal que $gx=h$. Por lo tanto, en la fila de $g$ el elemento $h$ aparece una sola vez (en la columna $x$). De forma análoga probamos que cada elemento aparece una sola vez en cada columna.
\end{myproof}
\newpage\section{El grupo de enteros módulo $n$}
\dfn{Clase de congruencia}{Una clase $\overline{z}$ es un conjunto de números enteros que comparten el mismo residuo cuando se dividen por un número entero (módulo).$$\overline{z}=\{x\in\mathbb{Z}|\ x\equiv z\ (mod\ n)\}$$}
\dfn{}{Llamaremos $\mathbb{Z}_n$ al conjunto de clases de modulo $n$. Por ejemplo $\mathbb{Z}_2=\{\overline{0},\overline{1}\}$ y $\mathbb{Z}_3=\{\overline{0},\overline{1},\overline{2}\}$. Es claro entonces que $\mathbb{Z}_n$ tiene $n$ elementos; es decir $\mathbb{Z}_n=\{\overline{0},\overline{1},\dots,\overline{n-1}\}$.}
\cor{}{Queremos definir en $\mathbb{Z}_n$ una operación que le de estructura de grupo. Quisiéramos definir una operación que llamaremos suma y la escribiremos como $+$, de forma mas natural $\overline{a}+\overline{b}=\overline{a+b}$.}
\ex{}{En $\mathbb{Z}_5$ tendríamos que $\overline{3}+\overline{4}=\overline{3+4}=\overline{7}=\overline{2}$.}
\mprop{}{Sea $n\in\mathbb{Z}$, entonces $\left( \mathbb{Z}_N, + \right) $ es un grupo abeliano.}
\begin{myproof}
    Veamos que la operación antes definida es asociativa: sean $a,b,c\in\mathbb{Z}$, entonces\\$(\overline{a}+\overline{b})+\overline{c}\overset{\text{def}}{=} \overline{(a+b)}+\overline{c}\overset{\text{def}}{=} \overline{(a+b)+c}$. Ahora, como la suma de enteros es asociativa, tenemos que $\overline{(a+b)+c}=\overline{a+(b+c)}\overset{\text{def}}{=} \overline{a}+\overline{b+c}\overset{\text{def}}{=} \overline{a}+(\overline{b}+\overline{c})$. Claramente $\overline{0}$ es neutro de esta operación
\end{myproof}
\mprop{}{Dados dos grupos $\left( G,*,e_G \right),\ \left( K,*,e_K \right) $ si consideramos el conjunto $G\times K=\{(g,k):g\in G,k\in K \}$ con la operación coordenada a coordenada: $(g,k)(g',k')=(g*g',k*k')$, entonces obtenemos un nuevo grupo (llamado el producto directo de $G$ y $K$).}
\newpage\section{El grupo de los invertibles módulo $n$}
\cor{}{De manera análoga a la suma de clases en $\mathbb{Z}_n$, podemos definir el producto de clases: $$\overline{a}\times\overline{b}=\overline{ab}$$}
\dfn{}{Llamamos $U(n)$ al conjunto de todas las clases de $z$ modulo $n$ que sean coprimos con $n$. Formalmente lo definimos como$$U(n)=\{\overline{a}:mcd(a,n)=1\}$$}
\ex{}{Por ejemplo $U(4)=\{\overline{1},\overline{3}\},\ U(5)=\{\overline{1},\overline{2},\overline{3},\overline{4}\}$ y $U(8)=\{\overline{1},\overline{3},\overline{5},\overline{7}\}$.}
\cor{}{Observar que $|U(n)|=\varphi(n)$.}
\mprop{}{$(U(n),\times,\overline{1})$ es un grupo abeliano con $\varphi(n)$ elementos.}
\section{Grupos Dihedrales}
\dfn{Grupos dihedrales}{Estos grupos describen las simetrías de figuras geométricas regulares, como polígonos y poliedros. El grupo dihedrico de orden $n$, denotado como $D_n$, consiste en todas las transformaciones rígidas (geométricas) que preservan las propiedades del objeto original. Estas transformaciones pueden ser rotaciones y reflexiones.\\La cantidad de elementos en el grupo dihedrico $D_n$ es $2n$, donde $n$ es el número de lados del polígono o caras del poliedro.}
\ex{}{Tomando $n=3$, consideremos en el plano, un triangulo equilátero $T$. Sea $D_3=\{f:\mathbb{R}^2\to\mathbb{R}^2:f$ es un movimiento del plano y $f(T)=T\}$. Entonces en $D_3$ tenemos al movimiento identidad, $id$; también las simetrías axiales $s_1,s_2,s_3$ con ejes en las mediatrices de los lados de $T$, y ademas tenemos las rotaciones antihorarias $r_1$ y $r_2$ con centro el centro del triangulo y ángulos $120$ y $240$ grados respectivamente.\\Entonces:$$D_3=\{id,r_1,r_2,s_1,s_2,s_3\}$$Es claro que si dos movimientos del plano preservan el triangulo, entonces su composición también.}
\mprop{}{$(D_3,\circ,id)$ es un grupo de orden $6$. Este grupo se llama grupo dihedral.}
\newpage\begin{myproof}
    Ya vimos que la composición de dos elementos de $D_3$ es nuevamente un elemento de $D_3$. La función $id$ es el neutro de la composición así que resta ver que todo elemento de $D_3$ tiene inverso:
    \begin{itemize}
        \item Claramente $(id)^{-1}=id$.
        \item Para todo $i=1,2,3$, tenemos que $s_i\circ s_i=id$ y por lo tanto cada simetría es inversa de si misma.
        \item Tenemos que $r_1\circ r_2=id$ y por lo tanto $(r_1)^{-1}=r_2$ y $(r_2)^{-1}=r_1$.
    \end{itemize}
\end{myproof}
\cor{}{Observar que $s_1\circ r_1=s_2$ y $r_1\circ s_1=s_3$, por lo tanto $D_3$ no es abeliano.}
\mprop{}{Por simplicidad notaremos $s=s_1$ y $r=r_1$. Tenemos las siguientes propiedades:
\begin{enumerate}
    \item $D_3=\{id,s,sr,sr^2,r,r^2\}$.
    \item $r^3=id$.
    \item $s^2=id$.
    \item $rs=sr^2$.
    \item Las relaciones anteriores (y la asociatividad) son suficientes para obtener todas las multiplicaciones en $D_3$.
\end{enumerate}
}
\ex{}{Para $n=4$, se considera un cuadrado $C$ en el plano y $D_4=\{f:\mathbb{R}^2\to\mathbb{R}^2:f$ es un movimiento del plano y $f(C)=C\}$.\\En $D_4$ tenemos el movimiento identidad $id$, cuatro simetrías axiales $s_1,s_2,s_3,s_4$ y tres rotaciones antihorarias $r_1,r_2,r_3$ con centro en el centro del cuadrado y ángulos $90,180$ y $270$ grados. Así que tenemos$$D_4=\{id,r_1,r_2,r_3,s_1,s_2,s_3,s_4\}$$Y en este caso tenemos que $s_2=s_1\circ r_1^3,\ s_3=s_1\circ r_1^2,\ s_4=s_1\circ r_1,\ r_2=r_1^2$ y $r_3=r_1^3$.}
\mprop{}{De forma analoga a lo hecho con $D_3$, se prueba que $(D_4,\circ,id)$ es un grupo no abeliano (con 8 elementos).\\En este caso, si llamamos $s=s_1$ y $r=r_1$ tenemos que
\begin{enumerate}
    \item $D_4=\{id,r,r^2,r^3,s,sr,sr^2,sr^3\}$.
    \item $r^4=id$.
    \item $s^2=id$.
    \item $rs=sr^3$.
    \item Las relaciones anteriores (y la asociatividad) son suficientes para obtener todas las multiplicaciones en $D_4$.
\end{enumerate}
}
\newpage\mprop{}{$(D_n,\circ,id)$ es un grupo no abeliano y $|D_n|=2n$. Estos grupos se llaman grupos dihedrales.\\
En este caso general, si llamamos $s=s_1$ y $r=r_1$ es la rotación antihoraria con centro en el centro del polígono y angulo $\frac{360}{n}$ grados, tenemos que
\begin{enumerate}
    \item $D_n=\{id,r,r^2,\dots,r^{n-1},s,sr,sr^2,\dots,sr^{n-1}\}$.
    \item $r^n=id$.
    \item $s^2=id$.
    \item $rs=sr^{n-1}$.
    \item Las relaciones anteriores (y la asociatividad) son suficientes para obtener todas las multiplicaciones en $D_n$.
\end{enumerate}
}
\newpage\section{Subgrupos y grupos cíclicos}
\dfn{}{Dado un grupo $(G,*,e)$, un subconjunto $H\subset G$ es un subgrupo de $G$ si cumple:
\begin{enumerate}
    \item (Cerrado con la operación) para todo $h,h'\in H,\ h*h'\in H$.
    \item (Neutro) $e\in H$.
    \item (Cerrado por inversos) si $h\in H$, entonces $h^{-1}\in H$.
\end{enumerate}
Escribiremos $H<G$ cuando $H$ es un subgrupo de $G$.\\Claramente un subgrupo es en particular un grupo con la misma operación de $G$.
}
\dfn{}{Si $(G,*,e)$ es un grupo definimos las potencias de $g$ como $g^0=e$ y si $n\in\mathbb{Z}^+$ entonces $$g^n=\underbrace{g*g*\dots *g}_\text{n veces}$$ $$g^{-n}=\underbrace{g^{-1}*g^{-1}*\dots g^{-1}}_\text{n veces}$$}
\mprop{}{Para todo $g\in G$ y $m,n\in\mathbb{Z}$, se cumple:
\begin{enumerate}
    \item $g^n*g^m=g^{n+m}$.
    \item $g^{-n}=(g^n)^{-1}$.
    \item $(g^n)^m=g^{mn}$.
\end{enumerate}}
\dfn{}{Si $(G,*,e)$ es un grupo y $g\in G$, al conjunto de todas las potencias de $g$ lo escribiremos $\langle g\rangle$; es decir $$\langle g\rangle = \{g^n:n\in\mathbb{Z}\}$$Como $g^0=e$, tenemos que $e\in\langle g\rangle$; ademas, por las dos primeras propiedades de la proposición anterior, tenemos que $\langle g\rangle$ es cerrado con la operación y cerrado por inversos y por lo tanto $\langle g\rangle$ es un subgrupo de $G$, al que llamamos subgrupo generado por $g$. En el caso en que para $G$, exista un elemento $g\in G$ tal que $\langle g\rangle = G$ decimos que $G$ es un grupo cíclico generado por $g$ (o decimos que $g$ es generador de $G$).}
\dfn{}{Sea $(G,*,e)$ un grupo y $g\in G$. Definimos el orden del elemento $g$ y lo escribiremos $o(g)$, de la siguiente manera:
\begin{itemize}
    \item Si $g^n\neq e\ \forall n\in\mathbb{Z}^+$, decimos que $o(g)=\infty$.
    \item En caso contrario, definimos $o(g)=min\{n\in\mathbb{Z}^+ :g^n=e\}$.
\end{itemize}}
\label{propord}\mprop{}{Si $(G,*,e)$ es un grupo y $g\in G$ entonces:
\begin{enumerate}
    \item Si $n\in\mathbb{Z}^+$, tenemos que $$o( g) =n\Leftrightarrow \begin{cases} g^{n} =e\\ \text{si } g^{m} =e\Rightarrow n|m \end{cases}$$
    \item Si $n\in\mathbb{Z}^+$ entonces $o(g)=n$ si y solo si $\begin{cases} g^{n} =e\\ g^{d} \neq e\ \forall \ d|n,\ d\neq n,\ d >0 \end{cases}$.
    \item Si $n\in\mathbb{Z}^+$ entonces $o(g)=n$ si y solo si $\begin{cases} g^{n} =e\\ g^{\frac{n}{p}} \neq e\ \forall \ p|n,\ p\neq n,\ p\ \text{primo} \end{cases}$.
    \item Se tiene que $g^m=e\Leftrightarrow o(g)|m$.
    \item Si $o(g)$ es finito, entonces $g^m=g^k$ si y solo si $m\equiv k$ (mod $o(g)$).
    \item Si $o(g)=\infty$ y $m\neq k$ entonces $g^m\neq g^k$.
    \item Si $o(g)$ es finito y $k\in\mathbb{Z}$ entonces $o(g^k)=\frac{o(g)}{mcd(k,o(g))}$.
    \item Si $o(g)$ es finito y $k\in\mathbb{Z}$, entonces $o(g)=o(g^k)$ si y solo si $mcd(k,o(g))=1$.
\end{enumerate}
}
\begin{myproof}
\begin{enumerate}
    \item Veamos primero el directo: si $n=o(g)$, por definición tenemos que $g^n=e$. Ademas, si $g^m=e$, dividiendo $m$ entre $n$ tenemos que $m=qn+r$ con $0\le r<n$. Tenemos que $e=g^m=g^{nq+r}=(g^n)^qg^r=e^qg^r=g^r$. Por lo tanto $g^r=e$ y como $n$ es la menor potencia positiva de $g$ con la que se obtiene $e$, y $0\le r<n$ concluimos que debe ser $r=0$ y por lo tanto $n|m$.\\Para el reciproco es evidente que si $n\in\mathbb{Z}^+$, $g^n=e$ y si cada vez que $g^m=e$ se tiene que $n|m$, entonces $n$ es la menor potencia positiva de $g$ con la cual se llega a $e$ y por lo tanto $n=o(g)$.
    \item Para el directo, si $n=o(g)$, por la definición sabemos que $g^n=e$. Ahora, si $g^d=e$ con $d|n,\ d\neq n$, por la primera parte sabemos que $n|d$, lo que implica que $n=d$ lo cual contradice la hipótesis sobre $d$. Concluimos que no existe tal $d$.\\Para el reciproco, supongamos que $m=o(g)\neq n$, que por definición de orden cumple $m<n$. Sabemos que $g^m=e$ y por la parte 1, $m|n$, que contradice la hipótesis. Por lo tanto $n=o(g)$.
    \item El directo es similar a la demostración anterior ya que $\frac{n}{p}|n$.\\El reciproco también es similar al anterior, supongamos que $m=o(g)\neq n$, de vuelta $m<n$. Por la parte 1, vemos que $m|n$ y como $m<n$ existe un primo $p$ tal que $p|n$ y $m|\frac{n}{p}$. Como $g^m=e$, entonces $g^{\frac{n}{p}}=e$ contradiciendo la hipótesis. Concluimos que $o(g)=n$.
    \item Se puede deducir de la primer parte de la proposición.
    \item Tenemos que $g^m=g^k$ si y solo si $g^m(g^k)^{-1}=e$; si y solo si, $g^{m-k}=e$. Y por la primer parte, esto sucede si y solo si $o(g)|(m-k)$; es decir, si y solo si $m\equiv k$ (mod $o(g)$).
    \item Supongamos que $m>k$; si tuviéramos que $g^m=g^k$, tendríamos que $g^{m-k}=e$ con $m-k>0$ y por lo tanto tendríamos que $o(g)$ es finito.
    \item Llamemos $n=o(g)$, y $d=mcd(n,k)$. Entonces tenemos que $n=dn'$, $k=dk'$ siendo $n'$ y $k'$ enteros coprimos. Entonces queremos probar que $o(g^k)=n'$. Usando la primer parte, debemos probar dos cosas: que $(g^k)^{n'}=e$ y que si $(g^k)^m=e$ entonces $n'|m$. Veamos lo primero: $(g^k)^{n'}=(g^{dk'})^{n'}=g^{dn'k'}=g^{nk'}=(g^n)^{k'}=e^{k'}=e$.\\Para lo segundo: si $(g^k)^m=e$ entonces $g^{km}=e$ y como $n=o(g)$, por la primer parte tenemos que $n|(km)$. Cancelando $d$ obtenemos que $n'|(k'm)$, y como $mcd(n',k')=1$, por el Lema de Euclides concluimos que $n'|m$.
    \item Es claro por la parte anterior.
\end{enumerate}
\end{myproof}
\mprop{}{\label{grupoorden}Si $(G,*,e)$ es un grupo y $g\in G$ entonces $$|\langle g\rangle |=o(g)$$}
\begin{myproof}
    Si $o(g)=\infty$, por la parte $4$ de la proposición, si $m\neq k$ tenemos que $g^m\neq g^k$ y por lo tanto en $\langle g\rangle=\{g^k:k\in\mathbb{Z}\}$ no hay elementos repetidos, y entonces $|\langle g\rangle |=\#\{g^k:k\in\mathbb{Z}\}=\infty=o(g)$.\\Ahora si $o(g)=n$ es finito, por la parte 3 de la proposición anterior tenemos que $g^m=g^k$ si y solo si $k\equiv m$ (mod $n$) y por lo tanto $\langle g\rangle =\{g^k:k\in\mathbb{Z}\}=\{g^0=e,g,g^2,\dots,g^{n-1}\}$ y entonces $|\langle g\rangle |=\#\{g^0=e,g,g^2,\dots,g^{n-1}\}=n=o(g)$.
\end{myproof}
\cor{}{sea $G$ un grupo de orden finito, entonces:\label{ordfinito}
\begin{enumerate}
    \item $G$ es cíclico si y solo si existe $g\in G$ tal que $o(g)=|G|$.
    \item Si $G=\langle g\rangle$, entonces $G=\langle g^k\rangle$ si y solo si $mcd(k,|G|)=1$.
    \item Si $G=\langle g\rangle$ entonces $G$ tiene $\varphi(|G|)$ generadores distintos.
\end{enumerate}
}
\begin{myproof}
    \begin{enumerate}
        \item $G$ es es cíclico si y solo si existe $g\in G$ tal que $\langle g\rangle = G$. Como $|G|$ es finito, esto sucede si y solo si existe $g\in G$ tal que $|\langle g\rangle |=|G|$. Y como $|\langle g\rangle |=o(g)$ queda demostrada la primera parte.
        \item Tenemos que $G=\langle g^k\rangle$ si y solo si $|G|=o(g^k)$. Como $|G|=o(g)$, tenemos que $G=\langle g^k\rangle$ si y solo si $o(g^k)=o(g)$ y por la parte 8 de la proposición anterior, tenemos que $o(g^k)=o(g)$ si y solo si $mcd(k,o(g))=1$ y como $o(g)=|G|$ se concluye lo deseado.
        \item Al ser $G=\langle g\rangle$ y $G$ finito, tenemos que $G=\{e=g^0,g,g^2,\dots,g^{|G|-1}\}=\{g^k:k\in\{0,\dots,|G|-1\}\}$.\\Junto con lo visto en la parte anterior concluimos que $\{h\in G:\langle h\rangle=G\}=\{g^k:k\in\{0,\dots,|G|-1\}$ y $mcd(k,|G|)=1\}$ y este conjunto tiene cardinal $\varphi (|G|)$.
    \end{enumerate}
\end{myproof}
\mprop{}{Sea $G$ un grupo cíclico, entonces todo subgrupo de $G$ también es cíclico.}
\newpage\section{Teorema de Lagrange}
\thm{Teorema de Lagrange}{Si $G$ es un grupo finito y $H<G$, entonces $|H|$ divide a $|G|$.}
\begin{myproof}
    La idea de la demostración es la siguiente: definiremos en $G$ una relación de equivalencia de forma tal que si $C$ es una clase de equivalencia, entonces $\#C=|H|$. Entonces, como $G$ es finito, la cantidad de clases de equivalencia también lo es; sean $C_1,C_2,\dots,C_k$ las clases de equivalencia distintas. Sabemos que el conjunto de clases de equivalencia (de cualquier relación de equivalencia) es una partición de $G$; es decir que $G=C_1\cup C_2\cup\dots\cup C_k$ y esta unión es disjunta. Por lo tanto tendremos que $|G|=\#C_1+\#C_2+\dots+\#C_k=\underbrace{|H|+|H|+\dots+|H|}_\text{$k$ veces}=k|H|$ y por lo tanto obtendremos que $|H|$ divide a $|G|$.\\Resta entonces definir la relación de equivalencia en $G$ que cumpla con lo deseado: para $g,g'\in G$ definimos $g\sim g'$ si existe $h\in H$ tal que $g=hg'$; o equivalentemente, $g\sim g'$ si $g(g')^{-1}\in H$. Veamos primero que esto define una relación de equivalencia:
    \begin{itemize}
        \item (reflexiva) Para todo $g\in G$, tenemos que $g\sim g$ pues $g=eg$ y $e\in H$ (pues $H$ es subgrupo de $G$).
        \item (simétrica) Sean $g,g'\in G$ tales que $g\sim g'$. Entonces $g(g')^{-1}\in H$. Al ser $H$ un subgrupo, es cerrado por inversos y por lo tanto $(g(g')^{-1})^{-1}\in H$. Por lo tanto $g'g^{-1}\in H$ y entonces $g'\sim g$.
        \item (transitiva) Si $g\sim g'$ y $g'\sim g''$ entonces existen $h,h'\in H$ tales que $g=hg'$ y $g'=h'g''$. Por lo tanto tenemos que $g=hg'=h(h'g'')=(hh')g''$. Al ser $H$ un subgrupo (en particular cerrado con la operación) tenemos que $hh'\in H$ y entonces $g\sim g''$.
    \end{itemize}
    Resta ver entonces que una clase de equivalencia tiene tantos elementos como $H$. Observar que si $g'\in G$ entonces la clase de equivalencia de $g'$ es $C=\{g\in G:g\sim g'\}=\{g\in G:\exists h\in H: g=hg'\}$. Por lo tanto $C=\{hg':h\in H\}$. Ademas, al multiplicar a todos los elementos de $H$ por $g'$, no hay repeticiones; es decir que si $h_1\neq h_2$ entonces $h_1g'\neq h_2g'$ (por la propiedad cancelativa). Por lo tanto $\#C=|H|$.
\end{myproof}
\cor{}{\label{proplagrange}Si $(G,*,e)$ es un grupo de orden finito y $g\in g$ tenemos que
\begin{enumerate}
    \item $o(g)\ |\ |G|$.
    \item $g^{|G|}=e$.
    \item Si $|G|$ es primo, entonces $G$ es cíclico.
    \item $G=\langle g\rangle$ si y solo si $g^d\neq e$ para todo $d|\ |G|$, $d\neq |G|$.
    \item $G=\langle g\rangle$ si y solo si $g^{\frac{|G|}{p}}\neq e$ para todo $p|\ |G|$, $p$ primo, $p\neq |G|$.
\end{enumerate}}
\begin{myproof}
    Consideramos $H=\langle g\rangle$; ya vimos que $H$ es un subgrupo de $G$ y que $|H|=o(g)$. Entonces, por el Teorema de Lagrange tenemos que $o(g)=|H|$ divide a $|G|$ y hemos probado la primer parte.\\Ademas, como $|G|$ es un múltiplo de $o(g)$, se deduce que $g^|G|=e$.\\Para la tercer parte, como $|G|>2$ entonces existe un $g\in G$ tal que $g\neq e$. Por el Teorema de Lagrange debemos tener que $|\langle g\rangle|$ divide a $|G|$. Como $|\langle g\rangle|>1$ y $|G|$ es primo tenemos que $|\langle g\rangle|=|G|$ y entonces $\langle g\rangle =G$.\\Por ultimo, las partes $4$ y $5$ son consecuencias de las partes $2$ y $3$ de la proposición \ref{propord}.
\end{myproof}
\newpage\section{Homomorfismos}
\dfn{}{Sean $(G,*)$ y $(K,*)$ dos grupos. Una función $f:G\to K$ es un homomorfismo o morfismo de grupos si para todo $g,g'\in G,\ f(g*g')=f(g)*f(g')$.}
\mprop{}{Sean $(G,*,e_G)$ y $(K,\star,e_K)$ dos grupos, $f:G\to K$ un homomorfismo y $g\in G$. Entonces:
    \begin{enumerate}\label{prophomo}
    \item $f(e_G)=e_K$.
    \item $f(g^{-1})=f(g)^{-1}$.
    \item $f(g^n)=f(g)^n$ para todo $n\in\mathbb{Z}$.
    \item Si $g\in G$ es un elemento de orden finito, entonces $o(f(g))$ también es finito y ademas divide a $o(g)$.
\end{enumerate}}
\dfn{}{Sean $(G,*,e_G)$ y $(K,\star,e_K)$ grupos y $f:G\to K$ un homomorfismo. Definimos:
\begin{itemize}
    \item  El núcleo de $f,\ Ker(f)=\{g\in G: f(g)=e_K\}$.
    \item La imagen de $f,\ Im(f)=\{k\in K: \exists g\in G: f(g)=k\}=\{f(g):g\in G\}$.
\end{itemize}}
\mprop{}{Sean $(G,*,e_G)$ y $(K,\star,e_K)$ dos grupos y $f:G\to K$ un homomorfismo, entonces:
\begin{enumerate}
    \item $Ker(f)<G$.
    \item $Im(f)<K$.
    \item $f$ es inyectiva si y solo si $Ker(f)=\{e_G\}$.
    \item $f$ es sobreyectiva si y solo si $Im(f)=K$.
\end{enumerate}}
\thm{Teorema de órdenes}{Sean $G$ y $K$ dos grupos y $f:G\to K$ un homomorfismo, entonces$$|G|=|Ker(f)|\times |Im(f)|$$}
\begin{myproof}
    Para cada $y\in Im(f)$, sea$$f^{-1}(y)=\{g\in G:f(g)=y\}\subset G$$es decir, $f^{-1}(y)$ es el conjunto de preimagenes de $y$. Observar que$$G=\bigcup _{y\in Im( f)} f^{-1}( y)$$y la unión es disjunta; esto es porque:
    \begin{itemize}
        \item Claramente la unión de las preimagenes es un subconjunto de $G$. A su vez, cada $g\in G$, esta en $f^{-1}(f(g))$, así que $G$ esta incluido en la unión de todas las preimagenes.
        \item Los conjuntos son disjuntos: si $g\in f^{-1}(y)\cap f^{-1}(y')\Rightarrow f(g)=y$ y $f(g)=y'$, al ser $f$ función, esto puede pasar solo si $y=y'$.
    \end{itemize}
    Si probamos que para todo $y\in Im(f),\ \#(f^{-1}(y))=|Ker(f)|$ entonces tendremos que:$$|G|=\#\left(\bigcup _{y\in Im( f)} f^{-1}( y)\right) =\sum _{y\in Im( f)} \#\left( f^{-1}( y)\right) =\sum _{y\in Im( f)} |Ker( f) |=|Ker( f) |\times |Im( f) |$$Probaremos esto ultimo verificando que si $y\in Im(f)$ y fijamos que $g\in f^{-1}(y)$, entonces$$f^{-1}(y)=\{gx:x\in Ker(f)\}$$Observemos que $\#\{gx:x\in Ker(f)\}=|Ker(f)|$ puesto que para cada $x\in Ker(f)$ tenemos un elemento $gx$ en este conjunto, y no hay repeticiones pues si $x\neq x'$, por la cancelativa se tiene que $gx\neq gx'$.\\Probaremos entonces que si $y\in Im(f)$ y fijamos que $g\in f^{-1}(y)=\{gx:x\in Ker(f)\}$.
    \begin{itemize}
        \item Veamos primero que $\{gx:x\in Ker(f)\}\subset f^{-1}(y):$ si $x\in Ker(f)$ entonces$$f(gx)=f(g)f(x)=f(g)e_K=f(g)=y\Rightarrow gx\in f^{-1}(y)$$(en la primer igualdad usamos que $f$ es homomorfismo y en la segunda que $x\in Ker(f)$).
        \item Veamos que ahora que $f^{-1}(y)\subset\{gx:x\in Ker(f)\}:$ sea $g'\in f^{-1}(y)$, queremos ver que existe $x\in Ker(f)$ tal que $g'=gx$. Ahora $g'=gx\Leftrightarrow x=g^{-1}g'$, así que basta con ver $g^{-1}g'\in Ker(f)$. Veamos:$$f(g^{-1}g')=f(g^{-1})f(g')=f(g)^{-1}f(g')=y^{-1}y=e_K\Rightarrow g^{-1}g'\in Ker(f)$$(en la primer igualdad usamos que $f$ es homomorfismo y en la segunda, la propiedad de homomorfismo para el inverso).
    \end{itemize}
\end{myproof}
\mprop{}{Sean $G$ un grupo cíclico finito con generador $g$ y $K$ un grupo finito. Sea $k\in K$, la función $f:G\to K$ dada por$$f(g^n)=k^n,\ n\in\mathbb{Z}$$esta bien definida y es un homomorfismo si y solo si $o(k)|o(g)$.}
\begin{myproof}
    El directo de la proposición es consecuencia de la parte 4 de la proposición \ref{prophomo}.\\Para el reciproco tenemos que verificar dos cosas, primero que $f$ esta bien definida y luego que es un homomorfismo. Para ver que esta bien definida tenemos que ver que si $g^n=g^m$ entonces $k^n=k^m$. Para eso recordamos que como $g^n=g^m$ entonces $n\equiv m\ (mod\ o(g))$, o sea que $o(g)|n-m$, pero $o(k)|o(g)$ entonces $o(k)|n-m$ y por lo tanto $k^n=k^m$. Solo queda verificar que $f$ es un homomorfismo.
\end{myproof}
\mlenma{}{Sea $G$ un grupo cíclico finito con generador $g$. Si $K$ es otro grupo finito, entonces todos los morfismos$$f:G\to K$$quedan determinados por $f(g)\in K$ tal que $o(f(g))|o(g)$.}
\cor{}{Sean $G$ y $K$ grupos finitos:
\begin{enumerate}
    \item Si $f:G\to K$ es un homomorfismo, entonces $|Im(f)|$ divide a $mcd(|G|,|K|)$.
    \item Si $|G|$ y $|K|$ son coprimos, entonces el único homomorfismo $f:G\to K$ es el trivial.
\end{enumerate}
}
\dfn{}{Dados dos grupos $(G,*,e_G)$ y $(K,\star,e_K)$, una función $f:G\to K$ es un isomorfismo si es un homomorifsmo biyectivo. Decimos que $G$ y $K$ son isomorfos si existe un isomorfismo $f:G\to K$.}
\cor{}{Tenemos que
\begin{enumerate}
    \item Un homomorifsmo $f:G\to K$ es un isomorfismo si y solo si $Ker(f)=\{e_G\}$ e $Im(f)=K$.
    \item Si $f:G\to K$ es un isomorfismo, entonces la función $f^{-1}:K\to G$ también es un isomorfismo.
    \item Si $G$ y $K$ son grupos isomorfos, entonces $|G|=|K|$.
    \item Si $G$ y $K$ son grupos isomorfos, entonces $G$ es abeliano si y solo si $K$ es abeliano.
    \item Si $f:G\to K$ es un isomorfismo y $g\in G$ entonces $o(g)=o(f(g))$.
\end{enumerate}
}
\newpage\chapter{Raíces Primitivas}\section{Raíces Primitivas}
\dfn{}{Dado un $n\in\mathbb{Z}^+$, un entero $g\in\{1,\dots,n\}$ es raíz primitiva modulo $n$, si $\langle\overline{g}\rangle =U(n)$.}
\mprop{}{Sean $n\in\mathbb{Z}^+$. Si existe una raíz primitiva modulo $n$, entonces hay $\varphi(\varphi(n))$ raíces primitivas módulo $n$.}
\begin{myproof}
    Si existe $g$ raíz primitiva módulo $n$, entonces $\langle\overline{g}\rangle =U(n)$. Como $U(n)$ es finito, por la ultima parte del corolario \ref{ordfinito} tenemos que $U(n)$ tiene $\varphi(|U(n)|)$ generadores; y al ser $|U(n)|=\varphi(n)$ concluimos que $U(n)$ tiene $\varphi(\varphi(n))$ generadores distintos. Es decir, tiene $\varphi(\varphi(n))$ raíces primitivas.
\end{myproof}
\nt{Observar que en particular $g$ es raíz primitiva modulo $n$, entonces el conjunto de raíces primitivas modulo $n$ son los restos de dividir entre $n$ los elementos del conjunto$$\{g^k:k\in\{1,\dots,\varphi(n)-1\}\text{ y }mcd(k,\varphi(n))=1\}$$}
\mprop{}{Sea $n\in\mathbb{Z}^+$ y $g\in\{1,\dots,n\}$, entonces las siguientes afirmaciones son equivalentes
\begin{enumerate}
    \item $g$ es raíz primitiva modulo $n$.
    \item $mcd(g,n)=1$ y el orden de $\overline{g}$ en $U(n)$ es $\varphi(n)$.
    \item $mcd(g,n)=1$ y $g^d\nequiv 1\ ($mod $n)$ para todo $d$ divisor de $\varphi(n)$ y $d\neq\varphi(n)$.
    \item $mcd(g,n)=1$ y $g^{\frac{\varphi(n)}{p}}=\nequiv 1\ ($mod $n)$ para todo $p$ primo divisor de $\varphi(n)$.
\end{enumerate}
}
\begin{myproof}
    \begin{enumerate}
        \item $\Leftrightarrow 2.$ Tenemos, por definición que $g$ es raíz primitiva modulo $n$ si y solo si $U(n)=\rangle\overline{g}\langle$. Por la proposición \ref{grupoorden} esto pasa si y solo si $\overline{g}\in U(n)$ y $o(\overline{g})=|U(n)|=\varphi(n)$; si y solo si $mcd(g,n)=1$ y el orden de $\overline{g}$ en $U(n)$ es $\varphi(n)$.
        \item $\Leftrightarrow 3.$ Ahora, si consideramos $G=U(n)$, por la primer parte del Corolario \ref{proplagrange} tenemos que si $\overline{g}\in U(n)$, entonces $o(\overline{g})$ divide a $|U(n)|=\varphi(n)$. Es decir $o(\overline{g})=d$ con $d$ divisor de $\varphi(n)$. Por lo tanto $o(\overline{g})=\varphi(n)$ si y solo si $\overline{g}^d\neq 1$ para todo $d$ divisor de $\varphi(n)$ y $d\neq\varphi(n)$; es decir, si y solo si $g^d\equiv 1\ ($mod $n)$ para todo $d$ divisor de $\varphi(n)$ y $d\neq\varphi(n)$.
        \item $\Leftrightarrow 4.$ Observar primero que si $m\in\mathbb{Z}$ y $d$ un divisor de $m$ y $d\neq m$, entonces existe un primo $p$ divisor de $m$ tal que $\frac{m}{p}$ es múltiplo de $d$. Esto es porque si consideramos la descomposición factorial de $m=p_1^{a_1}\dots p_k^{a_k}$ si $d$ es un divisor positivo de $m$, por el Corolario \ref{cantdivpos} tenemos que $d=p_1^{b_1}\dots p_k^{b_k}$ con $b_i\le a_i$ para todo $i=1,\dots k$. Y si $d\neq m$ entonces para algún $i,b_i<a_i$ y entonces $\frac{m}{p_i}$ es múltiplo de $d$.\\
            Ahora bien, volviendo a lo que queremos demostrar. Tenemos entonces que si $d$ es un divisor de $\varphi(n)$, y $d\neq\varphi(n)$, por lo visto recién existe un primo $p$ divisor de $\varphi(n)$ tal que $\frac{\varphi(n)}{p}$ es múltiplo de $d$; es decir, existe $c\in\mathbb{Z}$ tal que $cd=\frac{\varphi(n)}{p}$. Entonces si $g^{\frac{\varphi(n)}{p}}\nequiv 1\ ($mod $n)$ tenemos que $g^d\nequiv 1\ ($mod $n)$ (pues si tuviéramos que $g^d\equiv 1\ ($mod $n)$ elevando ambos lados por $c$ obtendríamos que $g^{\frac{\varphi(n)}{p}}\equiv 1\ ($mod $n)$.) Esto prueba que $4\Rightarrow 3$. Es claro que $3\Rightarrow 4$ pues $d=\frac{\varphi(n)}{p}$ es un divisor de $\varphi(n)$ y $d\neq\varphi(n)$.
    \end{enumerate}
\end{myproof}
\thm{}{Sea $n\in\mathbb{Z}^+$. Si existe una raíz primitiva modulo $n$, entonces
\begin{itemize}
    \item $n=1,2,4$ o
    \item $n=p$ con $p$ primo impar, o
    \item $n=p^k$ con $p$ primo impar y $k\in\mathbb{Z}^+$ o
    \item $n=2p^k$ con $p$ primo impar y $k\in\mathbb{Z}^+$.
\end{itemize}
}
\mlenma{}{\label{lemaxy}En un grupo $G$, si $x,y\in G$ son elementos de orden $a,b$ respectivamente tales que $xy=yx$ y $mcd(a,b)=1$ entonces el orden de $xy$ es $ab$.}
\begin{myproof}
    Veamos que $n=ab$ cumple las propiedades de la primer parte de la Proposición \ref{propord}. Como $xy=yx$ tenemos que para todo $m,\ (xy)^m=x^my^m$. Entonces:
    \begin{itemize}
        \item $(xy)^{ab}=x^{ab}y^{ab}=(x^a)^b(y^b)^a=e^be^a=e$.
        \item Sea $m\in\mathbb{Z}^+$ tal que $(xy)^m=e$; entonces elevando ambos a la $b$, obtenemos que $(xy^{mb}=e$ y por lo tanto $x^{mb}y^{mb}=e$ y entonces $x^{mb}=e$. Como $a=o(x)$ concluimos entonces que $a|mb$; y como $mcd(a,b)=1$, por el Lema de Euclides concluimos que $a|m$. De forma análoga se prueba $b|m$. Tenemos entonces que $a$ y $b$ dividen a $m$ y como $a$ y $b$ son coprimos concluimos que $ab|m$.
    \end{itemize}
\end{myproof}
\mlenma{}{\label{lema412}Sea $d\in\mathbb{Z}^+,\ f(x)=x^d+a_{d-1}x^{d-1}+\dots+a_1x+a_0$ con los coeficientes $a_i\in\mathbb{Z}$ para todo $i$. Entonces si $p$ es primo, la ecuación$$f(x)\equiv 0\ (\text{mod }p)$$tiene a lo sumo $d$ soluciones en $\mathbb{Z}_p$.}
\begin{myproof}
    Lo demostramos por inducción en $d$. El resultado es claro si $d=1$, ya que $x+a_0\equiv 0\ ($mod $p)$ si y solo si $x\equiv -a_0\ ($mod $p)$, y por lo tanto hay una única solución modulo $p$.\\Asumamos ahora que $d>1$ y el resultado es cierto para $d-1$. Si $f(x)\equiv 0\ ($mod $p)$ no tiene soluciones enteras, ya esta. Si tiene una solución entera $x=a$ (es decir que $f(a)\equiv 0\ ($mod $p)$), entonces dividiendo entre $x-a$ obtenemos que $f(x)=(x-a)q(x)+f(a)\equiv 0\ ($mod $p)$ si y solo si $(x-a)q(x)\equiv 0\ ($mod $p)$; si y solo si $p$ divide a $(x-a)q(x)$. Ahora como $p$ es primo, esto ultimo sucede si y solo si $p|(x-a)$ o $p|q(x)$. Es decir, si y solo si $x\equiv a\ ($mod $p)$ o $q(x)\equiv 0\ ($mod $p)$. Ahora, $q(x)$ es un polinomio en las mismas hipótesis que $f$, pero con grado $d-1$, y por lo tanto (hipótesis inductiva) $q(x)\equiv 0\ ($mod $p)$ tiene a lo sumo $d-1$ soluciones en $\mathbb{Z}_p$.
\end{myproof}
\mlenma{}{Sea $p$ primo y $d$ un divisor de $p-1$. Entonces la ecuación $x^d\equiv 1\ ($mod $p)$ tiene exactamente $d$ raíces distintas en $U(p)$.}
\begin{myproof}
    Si $p-1=de$ con $d,e\in\mathbb{Z}^+$, entonces$$x^{p-1}-1=(x^d)^e-1=(x^d-1)(x^{d(e-1)}+x^{d(e-2)}+\dots+x^d+1)$$Llamemos $n$ a la cantidad de soluciones modulo $p$ de $x^d-1\equiv 0\ ($mod $p)$ y $m$ a la cantidad de soluciones modulo $p$ de $g(x)=x^{d(e-1)}+x^{d(e-2)}+\dots+x^d+1\equiv 0\ ($mod $p)$. Por Fermat tenemos que la cantidad de soluciones modulo $p$ de $x^{p-1}-1\equiv 0\ ($mod $p)$ es exactamente $p-1$. Entonces$$\begin{aligned} p-1 & =\text{cantidad de soluciones modulo } p\text{ de }\left( x^{p-1} -1\equiv 0\ \left(\text{mod} \ p\right)\right)\\  & \le  n+m\\  & \le d+m\\  & \le d+d( e-1)\\  & =de=p-1 \end{aligned}$$donde la primer desigualdad es porque puede haber repeticiones, y la segunda y tercera es por el Lema \ref{lema412}. Por lo tanto, todas las desigualdades son igualdades y en particular $n=d$.
\end{myproof}
\thm{Teorema de la raíz primitiva}{Si $p$ es primo, entonces existen raíces primitivas modulo $p$.}
\begin{myproof}
    Si $p-1=p_1^{d_1}p_2^{d-2}\dots p_k^{d_k}$ es la factorización en primos de $p-1$, la idea es encontrar elementos $x_1,x_2,\dots,x_k$ en $U(p)$ con ordenes $p_1^{d_1},p_2^{d_2},\dots,p_k^{d_k}$ respectivamente. Luego si$$g=x_1x_2\dots x_k$$por el Lema \ref{lemaxy} tendremos que $o(g)=o(x_1)\dots o(x_k)=p-1$ y por lo tanto $g$ sera una raíz primitiva modulo $p$.\\Veamos entonces, que para $i=1,\dots,k$ existe $x_i$ con orden $p_i^{d_i}$. Por el Lema \ref{lema412} sabemos que $x^{p_i^{d_i}}\equiv 1\ ($mod $p)$ tiene exactamente $p_i^{d_i}$ soluciones en $U(p)$ y que $x^{p_i^{d_i}}\equiv 1\ ($mod $p)$ tiene exactamente $p_i^{d_i-1}$ soluciones.\\Por lo tanto, existe un $x_i$ solución de la primer ecuación y que no es solución de la segunda.\\Es decir $x_i^{p_i^{d_i}}\equiv 1\ ($mod $p)$ y $x_i^{p_i^{d_i-1}}\nequiv 1\ ($mod $p)$; por lo tanto $x_i$ tiene orden $p_i^{d_i}$.
\end{myproof}
\mlenma{}{Sea $p$ un primo impar. Si $g$ es raíz primitiva modulo $p$ entonces $g$ o $g+p$ es raíz primitiva modulo $p^2$.}
\mlenma{}{Sea $p$ un primo impar. Si $g$ es raíz primitiva modulo $p^2$, entonces $g$ es raíz primitiva modulo $p^k$ para todo $k\in\mathbb{Z}^+$.}
\mlenma{}{Si $p$ es un primo impar, $k\in\mathbb{Z}^+$ y $g$ es raíz primitiva modulo $p^k$ entonces:
\begin{itemize}
    \item Si $g$ es impar, $g$ es raíz primitiva módulo $2p^k$.
    \item Si $g$ es par, $g+p^k$ es raíz primitiva módulo $2p^k$.
\end{itemize}}
\thm{}{Sea $n\in\mathbb{Z}^+$. Entonces existe una raíz primitiva módulo $n$ si y solo si cumple alguna de las siguientes:
\begin{itemize}
    \item $n=1,2,4$.
    \item $n=p$ con $p$ primo impar.
    \item $n=p^k$ con $p$ primo impar y $k\in\mathbb{Z}^k$.
    \item $n=2p^k$ con $p$ primo impar y $k\in\mathbb{Z}^+$.
\end{itemize}}
\chapter{Criptografía}\section{Método de cifrado Vigenere}
Primero, a cada letra del alfabeto le daremos un numero, $A=0,\ B=1,\dots,\ Z=26$ y el espacio sera el $27$.
\\Ahora, el método consiste en repetir debajo del texto cifrado la palabra clave, luego sumar cada letra del texto plano con la letra de la palabra clave que esta debajo de ella (codificando cada letra con un natural) y reduciendo modulo la cantidad de símbolos (por ejemplo 28 si utilizamos las letras de A a la Z y el espacio).
\\\\A modo de ejemplo ciframos el texto plano ``ATACAREMOS POR LA NOCHE'', utilizando el método Vigenere con la palabra clave "PRUEBA":
\begin{table}[h]
\begin{tabular}{|l|l|l|l|l|l|l|l|l|l|l|l|l|l|l|l|l|l|l|l|l|l|l|}
\hline
A & T & A & C & A & R & E & M & O & S &   & P & O & R &   & L & A &   & N & O & C & H & E \\ \hline
P & R & U & E & B & A & P & R & U & E & B & A & P & R & U & E & B & A & P & R & U & E & B \\ \hline
P & K & U & G & B & R & V & C & I & W & A & P & D & I & T & P & B &   & B & F & W & L & F \\ \hline
\end{tabular}
\end{table}

Donde la primer fila consiste en el texto plano, en la segunda hemos repetido la palabra clave varias veces, y en la ultima el texto cifrado que fue calculado usando las dos letras que aparecen en la misma columna, de esta forma obtenemos el texto cifrado.\\Para sumar las letras lo que hemos hecho es sumar sus valores numéricos correspondientes modulo 28 y luego sustituimos este valor por su carácter correspondiente, por ejemplo $A+P=0+16\equiv 16\ ($mod $28)$, letra que corresponda a $16$ es P, luego $A+P=P;\ T+R=20+18=39\equiv 10\ ($mod $28)$, la letra que corresponde a $10$ es K, y así sucesivamente.\\Para descifrar el texto simplemente repetimos la palabra clave debajo del texto plano, pero esta vez en vez de sumar, restamos.
\newpage\section{Cripotosistemas de clave privada, métodos de intercambio de clave}
Se llaman criptosistemas de clave privada a aquellos criptosistemas que se pueden obtener fácilmente la clave de descifrado a partir de la de cifrado.
\\En estos sistemas, la clave de cifrado ha de ser confidencial entre las personas que llevan la comunicación, dado que a partir de ellas un espía puede calcular la clave de descifrado con facilidad. Pero, como hacer para intercambiar claves a distancia sin que alguien pueda interceptar la conversación no sea capaz de encontrar cual es la clave?
\subsection{Método Diffie-Helmann de intercambio de clave}
Supongamos que Ana y Bernardo quieren ponerse de acuerdo en una clave común que sea secreto. Pero ellos se encuentran lejos uno del otro y la única forma de comunicarse entre ellos es a través de un canal. El problema es que el canal esta interceptado por espiás que pueden acceder a la conversación de Ana y Bernardo. Diffie-Helmann nos da un posible método para resolver el problema:
\begin{enumerate}
    \item Ana y Bernardo se ponen de acuerdo en un primo $p$ y raíz primitiva $g$ con $1<g<p$.
    \item Ana elige un numero al azar $n$ y Bernardo elige un numero al azar $m$.
    \item Ana calcula $g^n\ ($mod $p)$ y se lo manda por el canal.
    \item Bernardo calcula $g^m ($mod $p)$ y se lo manda por el canal.
    \item La clave común es $c\equiv g^{nm}\ ($mod $p)\equiv (g^m)^n\ ($mod $p)$, que tanto Ana como Bernardo pueden calcular.
\end{enumerate}
El espía que accede a la conversación puede conocer $p,g,g^n$ y $g^m$. Si el espía con esos datos fuese capaz de calcular $g^{nm}$ entonces hemos fallado en el intento de acordar la clave común. Pero la única manera de calcular $g^{mn}$ es calculando previamente $n$ o $m$. Esto en general es un problema computacional mente difícil y es conocido como el problema del logaritmo discreto en $U(p)$.\\\\
\textbf{Problema del logaritmo discreto en $U(p)$:} Dados un primo $p,\ g$ una raíz primitiva módulo $p$ y $a\in U(p)$ hallar un $m$ tal que $g^m\equiv a\ ($mod $p)$. A un tal $m$ se le llama logaritmo discreto de $a$ en base $g$ y se lo denota por $m=dlog_ga$.\\Se puede probar fácilmente que el logaritmo discreto de un numero, si existe, no es único sino que esta determinado modulo $p-1$.\\En la parte 1 se podría pedir que $g$ solo sea coprimo con $p$. La ventaja de tomar $g$ raíz primitiva es que tiene orden $p-1$ que es lo mas grande que puede ser y hay por lo tanto mas posibilidades para potencias de $g$. Esto hace que sea mas difícil de resolver el logaritmo discreto.\\Otra cosa a observar es que tanto Ana en el paso 4 como Bernardo en el 5 necesitan calcular $g^n\ ($mod $p)$ (y $g^m\ ($mod $p)$ respectivamente), esto puede hacerse usando el método de exponenciación rápida.
\newpage\section{Criptosistemas de clave publica}
Los criptosistemas de clave publica basan su seguridad en que no haya un método eficiente de calcular la clave de descifrado. Estos sistemas tienen la ventaja de que como la clave de cifrado no nos ayuda a calcular la clave de cifrado, puede almacenarse todas las claves de cifrado de muchos usuarios en una guía publica a la cual todos tengan acceso, evitando así que cada vez que dos usuarios quieran comunicarse tengan que ponerse de acuerdo en una clave común.
\subsection{Criptosistema RSA}
Este es uno de los sistemas de clave publica mas famosos. La idea detrás de este criptosistema es la de construir una función que sea fácil de calcular (en este caso multiplicar dos primos), pero que su inversa sea difícil de calcular (en este caso dado un numero que es producto de 2 primos, hallar esos primos). Veamos en que consiste:
\begin{enumerate}
    \item Ana elige dos primos (distintos) grandes $p$ y $q$ y calcula $n=pq$.
    \item Luego calcula: $\varphi(n)=\varphi(n)\varphi(m)=(p-1)(q-1)$.
    \item Luego elige un número aleatorio $e$ con $1<e<\varphi(n)$ y $mcd(e,\varphi(n))=1$.
    \item Finalmente Ana tiene definida una función (función de cifrado) definida por:$$E:\mathbb{Z}_n\to\mathbb{Z}_n:E(x)=x^e\ (\text{mod }n)$$
\end{enumerate}
\end{document}
