\documentclass[10pt]{article}
\usepackage[spanish]{babel}
\usepackage{amsmath, amssymb, amsthm, tikz}
\usepackage{nicematrix}
\usepackage{enumitem}
\usepackage[margin=1in]{geometry}
\usetikzlibrary{babel}

\theoremstyle{definition}
\newtheorem{definition}{Definición}[section]
\newtheorem{theorem}{Teorema}[section]
\newtheorem{corollary}{Corolario}[theorem]
\newtheorem{example}{Ejemplo}[section]

\DeclareMathOperator{\coord}{coord}

\title{Geometría y Álgebra Lineal 2}
\author{Santiago Sierra}

\begin{document}
\maketitle \tableofcontents \newpage
\section{Matriz asociada, cambio de base y semejanza}
\subsection{Representación matricial de una transformación lineal}
Se le llama matriz asociada a $T$ en bases $A$ y $B$ a la matriz que representaremos por $_B(T)_A$ y cuya i-esima columna son las coordenadas del vector $T(v_1)$ en la base $B$.\\
Sea $T:V\rightarrow W$ una TL, $A=\{v_{1} ,v_{2} ,\dotsc ,v_{n}\}$ una base de $V$ y $B=\{w_{1} ,w_{2} ,\dotsc ,w_{n}\}$ una base de $W$.
$$\begin{array}{l}
		T(v_1)=a_{11}w_1+…+a_{m1}w_m \\
		T(v_2)=a_{12}w_1+…+a_{m2}w_m \\
		T(v_n)=a_{1m}w_1+…+a_{mn}w_m
	\end{array}$$
Es decir:
$$coord_B(T(v_1))=\left(\begin{matrix}a_{11}\\a_{21}\\ \vdots \\a_{m1}\end{matrix}\right), coord_B(T(v_2))=\left(\begin{matrix}a_{12}\\a_{22}\\ \vdots \\ a_{m2}\end{matrix}\right), \dots, coord_B(T(v_n))=\left(\begin{matrix}a_{1n}\\a_{2n}\\\vdots \\ a_{mn}\end{matrix}\right)$$
\begin{definition}
	Se le llama representación matricial de $T$ en las bases $A$ y $B$ o matriz asociada a $T$ en las bases $A$ y $B$, a la matriz que representaremos por $_B(T)_A$ y cuya i-esima columna son las coordenadas del vector $T(v_i)$ en la base $B$.
\end{definition}
\begin{equation*}
	\begin{NiceMatrix}
		{}_{B}(T)_{A}\hspace*{-0.4em} & =                     &
		[\coord_{B} T(v_{1})]         & [\coord_{b} T(v_{2})] & \cdots & [\coord_{B} T(v_{n})] \\[2ex]
		                              & \Block{4-1}{=}        &
		a_{11}                        & a_{12}                & \cdots & a_{1n}                \\
		                              &                       &
		a_{21}                        & a_{22}                & \cdots & a_{2n}                \\
		                              &                       &
		\vdots                        & \vdots                &        & \vdots                \\
		                              &                       &
		a_{m1}                        & a_{m2}                & \cdots & a_{mn}
		\CodeAfter
		\SubMatrix({1-3}{1-6})
		\SubMatrix({2-3}{5-6})
	\end{NiceMatrix}
\end{equation*}
\begin{corollary}
	La transformación lineal coordenadas es un isomorfismo entre $V$ y $\mathbb{K}^n$
\end{corollary}
\begin{example}{ \ }
	\\Consideremos la transformación lineal $T:\mathbb{R}^{2} \rightarrow \mathbb{R}^{3}$ tal que: $T(x,y)=(4x-2y,2x+y,x+y)$  $\forall (x,y)\in \mathbb{R}$.\\
	Y las bases $A={(1,0),(0,1)}$ de $\mathbb{R}^2$ y $B={(1,0,0),(0,1,0),(0,0,1)}$ de $\mathbb{R}^3$.\\
	Hallar la matriz asociada a $T$ en dichas bases:
	\begin{enumerate}
		\item Hallamos las imágenes de los vectores de la base $A$:\\
		      $T(1,0)=(4,2,1)$\\
		      $T(0,1)=(-2,1,1)$
		\item Calculamos las coordenadas en la base $B$:\\
		      $T(1,0)=(4,2,1)=\alpha (1,0,0)+\beta (0,1,0)+\gamma (0,0,1)\rightarrow \begin{matrix}\alpha =4\\\beta =2\\\gamma =1\end{matrix}$\\
		      $coord_B(T(1,0))=\begin{pmatrix}4\\2\\1\end{pmatrix}$\\
		      $T(0,1)=(-2,1,1)=\alpha(1,0,0)+\beta(0,1,0)+\gamma(0,0,1)\rightarrow \begin{matrix}\alpha =-2\\\beta =1\\\gamma =1\end{matrix}$\\
		\item Tenemos que $_B(T)_A=\begin{pmatrix}4 & -2\\2 & 1\\1 & 1\end{pmatrix}$
	\end{enumerate}
\end{example}
\begin{corollary}
	Sea $T:V\to W$, $dim(V)=n$ y $dim(W)=m$, la matriz asociada tiene dimensión $m\times n$.
\end{corollary}
La matriz $_B(T)_A$ queda completamente determinada conocidas la transformación lineal $T$ y las bases $A$ y $B$ del dominio y codominio respectivamente.\\
Recíprocamente, dada una matriz $M$ de tamaño $m\times n$ y dos bases $A$ y $B$ de los espacios $V$ y $W$ respectivamente, queda completamente determinada una transformación lineal $T$ tal que $_B(T)_A=M$.\\
En efecto, conociendo la matriz $M$, sus columnas son las coordenadas en la base $B$ de las imágenes de dos vectores de la base $A$, esto nos permite conoces las imágenes de los vectores de la base $A$ y esto es suficiente para determinar $T$.
\begin{example}{ \ }
	\\Hallar la transformación lineal $T:\mathbb{R}^3\rightarrow\mathbb{R}^2$ sabiendo que $_B(T)_A=\begin{pmatrix}2&3&-1\\1&0&2\end{pmatrix}$, donde $A=\{(1,0,1),(2,0,0),(0,1,0)\}$ es base de $\mathbb{R}^3$ y $B=\{(2,-1),(0,2)\}$ es base de $\mathbb{R}^2$.\\
	$coord_B(T(1,0,1))=\begin{pmatrix}2\\1\end{pmatrix}$\\
	$coord_B(T(2,0,0))=\begin{pmatrix}3\\0\end{pmatrix}$\\
	$coord_B(T(0,1,0))=\begin{pmatrix}-1\\2\end{pmatrix}$\\
	$T(1,0,1)=2(2,-1)+1(0,2)=(4,0)$\\
	$T(2,0,0)=3(2,-1)+0(0,2)=(6,-3)$\\
	$T(0,1,0)=-1(2,-1)+2(0,2)=(-2,5)$\\
	$(x,y,z)=\alpha(1,0,1)+\beta(2,0,0)+\gamma(0,1,0)=(\alpha+2\beta,\gamma,\alpha)\rightarrow\begin{matrix}\alpha=z\\\gamma=y\\\beta=\frac{x-z}{2}\end{matrix}$\\
	$(x,y,z)=z(1,0,1)+\frac{x-z}{2}(2,0,0)+y(0,1,0)$\\
	Por linealidad de $T$:\\
	$$\begin{array}{ c l }
			T(x,y,z) & =z T(1,0,1)+\frac{x-z}{2}  T(2,0,0)+y T(0,1,0) \\
			         & =z(4,0)+\frac{x-z}{2} (6,-3)+y(-2,5)           \\
			         & =(3x-2y+z,-\frac{3}{2} x+5y+\frac{3}{2} z)
		\end{array}$$
\end{example}
\begin{theorem}
	Sean $V$, $W$ espacios vectoriales sobre un mismo cuerpo $\mathbb{K}$, $A=\{v_1,v_2,\dots,v_n\}$ y $B=\{w_1,w_2,\dots,w_n\}$ bases ordenadas de $V$ y $W$ respectivamente y $T:V\rightarrow W$ una transformación lineal, entonces se cumple que:
	$$coord_B(Tv)=\ _B(T)_A\ coord_A(v)$$
\end{theorem}
\begin{example}
	Dado $T:\mathbb{R}^3\rightarrow\mathbb{R}^3$ y las bases $A=B=\{(1,0,0),(1,1,0),(1,1,1)\}$ tal que
	$$ _B(T)_A=\begin{pmatrix} 1 & -1 & 0 \\ 2 & 0 & 0 \\ 3 & 4 & 1 \end{pmatrix}$$
	Hallar $T(2,0,1)$
	$$(2,0,-1)=\alpha(1,0,0)+\beta(1,1,0)+\gamma(1,1,1)=(\alpha+\beta+\gamma,\beta+\gamma,\gamma)\rightarrow\begin{matrix}\alpha=2 \\ \beta=1 \\ \gamma = -1\end{matrix}$$
	$$
		coord_B(T(2,0,1))=\ _B(T)_A \ coord_A(2,0,1)=\begin{pmatrix} 1&-1&0 \\2 & 0 & 0 \\ 3 & 4 & 1\end{pmatrix} \begin{pmatrix}2\\1\\-1\end{pmatrix}=\begin{pmatrix}1\\4\\9\end{pmatrix}
	$$
	$$
		T(2,0,1)=1(1,0,0)+4(1,1,0)+9(1,1,1)=(14,13,9)
	$$
\end{example}\newpage
\subsection{Matriz asociada y operaciones con transformaciones}
\begin{theorem}
	Sean dos transformaciones lineales $T:V\to W$ y $S:V\to W$.\\
	Sea $B=\{v_1,\dots,v_n\}$ base de $V$ y $E=\{w_1,\dots,w_m\}$ base de $W$. Entonces:
	$$_E(T+S)_B= _E(T)_B+_E(S)_B$$
\end{theorem}
\begin{theorem}
	Sea $T:V\to W$ una transformación lineal y $\alpha$ un escalar de $\mathbb{K}$.\\
	Sea $B=\{v_1,\dots,v_n\}$ base de $V$ y $E=\{w_1,\dots,w_m\}$ base de $W$. Entonces:
	$$_E(\lambda T)_B=\lambda \ _E(T)_B$$
\end{theorem}
\begin{theorem}
	Sean $U, V$ y $W$ espacios vectoriales con $dim(U)=s$, $dim(V)=n$ y $dim(W)=t$, y las transformaciones lineales $S:U\rightarrow V$ y $T:V\rightarrow W$.
	\\Sea $A=\{u_1,u_2,\dots,u_n\}$, $B=\{v_1,v_2,\dots,v_n\}$ y $C=\{w_1,w_2,\dots,w_n\}$ bases de $U, V$ y $W$ respectivamente. Entonces la matriz asociada a la composición $T\circ S$ es el producto de las matrices asociadas.
	$$
		_C(T\circ S)_A=\ _C(T)_B\ _B(S)_A
	$$
\end{theorem}
\underline{Observación:} Sea $T:V\rightarrow W$ un isomorfismo, $T^{-1}:W\rightarrow V$ su inversa, $B$ y $B'$ bases de $V$ y $W$ respectivamente. Como $T\circ T^{-1}=Id_W$ se cumple:
$$
	_{B'}(T)_B \ _B(T^{-1})_{B'}=\ _{B'}(Id_W)_{B'}=I
$$
También $T\circ T^{-1}=Id_V$ por lo que
$$
	_B(T^{-1})_{B'} \ _{B'}(T)_B=\ _B(T)_B=I
$$
Por lo que deducimos que la matriz asociada a la transformación inversa es la inversa de la matriz asociada a la transformación:
$$
	_{B'}(T)_B=A\rightarrow \ _B(T^{-1})_{B'}=A^{-1}
$$
Observar también $dim(V)=dim(W)$
\subsection{Cambio de bases}
Sean $A=\{v_1,v_2,\dots,v_n\}$ y $A'=\{v_1',v_2',\dots,v_n'\}$ del espacio $V$ e $I:V\to V$ la transformación identidad.
\begin{definition}
	Llamaremos matriz de cambio de base de la base ("vieja") $A$ a la base ("nueva") $A'$ a la matriz:$$_{A'}(I)_A$$
\end{definition}
\begin{theorem}
	Sean $A$ y $A'$ bases del espacio vectorial $V$. Entonces $$coord_{A'}(v)=\ _{A'}(I)_A coord_A(v)$$
\end{theorem}
\begin{theorem}
	Sean $V$ y $W$ espacios vectoriales sobre un mismo cuerpo $\mathbb{K}$ y $A$, $A'$ bases de $V$ y $B$, $B'$ bases de $W$ y $T:V\to W$ una transformación lineal. Entonces:
	$$_{B'}(T)_{A'}=\ _{B'}(I_W)_B \ _B(T)_A \ _A(I_V)_{A'}$$
	Donde $I_V:V\to V$ y $I_W:W\to W$ son las transformaciones lineales identidad en $V$ y $W$ respectivamente.
\end{theorem}
\begin{theorem}
	Sea $V$ un espacio vectorial sobre un mismo cuerpo $\mathbb{K}$, $A$ y $A'$ bases de $V$, $I:V\to V$ la transformación lineal identidad. Entonces:
	\begin{enumerate}
		\item $_{A'}(I)_A=\begin{pmatrix}1&0&\cdots&0\\0&1&\dots&0\\\vdots&\vdots&\ddots&\vdots\\0&0&\dots&1\end{pmatrix}\text{(Matriz identidad)}$
		\item $_{A'}(I)_A$ es invertible y $[_{A'}(I)_A]^{-1}=\ _A(I)_{A'}$
	\end{enumerate}
\end{theorem}
\subsection{Transformaciones y matrices semejantes}
\begin{definition}{Operador lineal.}
	\\Sea $V$ un espacio vectorial de dimensión finita sobre un cuerpo $\mathbb{K}$. Llamaremos operador en $V$ a toda transformación lineal $T:V\rightarrow V$; o sea, operador es una transformación lineal de un espacio vectorial en si mismo.
\end{definition}
\begin{definition}
	Sean $A$ y $B\in M(\mathbb{K})_{n\times n}$. Diremos que $A$ y $B$ son semejantes cuando existe $P \in M(\mathbb{K})_{n\times n}$ invertible tal que $B=P^{-1}AP$.`
\end{definition}
\begin{theorem}
	Dadas $A$ y $B\in M(\mathbb{K})_{n\times n}$. Las matrices $A$ y $B$ son semejantes si y solo si son matrices asociadas a un mismo operador $T$ en $V$.
\end{theorem}
\begin{theorem}
	Sean $A$ y $B$ matrices semejantes en $M(\mathbb{K})_{n\times n}$. Entonces:
	\begin{enumerate}
		\item $rango(A)=rango(B)$
		\item $traza(A)=traza(B)$
		\item $det(A)=det(B)$
	\end{enumerate}
\end{theorem}
\begin{corollary}
	No vale el reciproco de la proposición anterior, pues para $A=\begin{pmatrix} 1 & 0 \\ 0 & 1 \end{pmatrix}$ y $B=\begin{pmatrix} 1 & 0 \\ 1 & 1 \end{pmatrix}$.\\
	Se cumple que $rango(A)=rango(B)=2$, $traza(A)=traza(B)=2$ y $det(A)=det(B)=1$.\\
	Sin embargo, no existe $P$ invertible tal que $B=P^{-1}AP$, es decir, $A$ y $B$ no son semejantes. Esto lo podemos observar ya que $A$ es la matriz asociada al operador identidad y $B$ es la matriz asociada a operadores que no son la identidad.
\end{corollary}
\section{Diagonalización}
\subsection{Valores, Vectores y Subespacios propios}
\begin{definition}{Valor y vector propio.}
	\\Sea $V$ un espacio vectorial sobre el conjunto de escalares $\mathbb{K}$ y $T:V\to V$ un operador lineal.\\
	Llamamos vector propio de $T$ asociado al valor propio $\lambda\in\mathbb{K}$ a todo vector $v\neq \vec{0}$ tal que $T(v)=\lambda v$.
\end{definition}
\begin{corollary}
	\hfill
	\begin{itemize}
		\item El vector nulo se excluye de la definición anterior, pues $T(\vec{0})=\lambda\vec{0}=\vec{0}$, $\forall\lambda\in\mathbb{K}$ y consecuentemente todo escalar resultaría valor propio.
		\item Dada una transformación lineal, los valores propios son números del cuerpo $\mathbb{K}$, sobre el que esta definido el espacio vectorial.
		\item Si $v$ es vector propio de $T$ asociado al valor propio $\lambda$, entonces para cualquier escalar $\alpha$ no nulo, $\alpha v$ también lo es, pues: $$T\left( \alpha v \right) = \alpha T(v)=\alpha\lambda v=\lambda(\alpha v)$$
	\end{itemize}
\end{corollary}
\begin{definition}{Subespacio propio.}
	\\El subespacio $S_{\lambda}$ es llamado subespacio propio asociado al valor propio $\lambda$.
\end{definition}
Sea $T$ y $\lambda$ un valor propio de $T$, definimos el subespacio propio como $$S_{\lambda}=\{v\in V : T(v)=\lambda v\}$$
En estas condiciones, $$S_{\lambda}=N(T-\lambda Id)$$
\newpage
Propiedades de los valores propios:
\begin{enumerate}
	\item Si $T$ es invertible y $\lambda$ un valor propio de $T$, entonces $\lambda ^{-n}$ es valor propio de $T^{-n}$ $\forall n\in\mathbb{N}$.
	\item Si $\lambda$ es valor propio de $T$, entonces $\lambda ^n$ es valor propio de $T^n$ $\forall n\in\mathbb{N}$.
	\item $0$ es valor propio de $T$ si y solo si $T$ no es invertible.
	\item $A$ y $A^T$ tienen igual polinomio característico, valores propios y vectores propios.
	\item La suma de los valores propios es igual a la $traza(A)$.
	\item El producto de los valores propios es igual al $det(A)$.
\end{enumerate}
\subsection{Cálculo de valores y vectores propios.}
\begin{definition}{Polinomio, ecuación y raíces característica.}
	\\Se llama polinomio característico de una matriz cuadrada $A$ al polinomio $X_A\left( \lambda \right) $. Se le llama ecuación característica a $X_A(\lambda)=0$, y raíces características de $A$ a todas las soluciones del polinomio.
	\\Siendo $X_A(\lambda)=det(A-\lambda Id)$
\end{definition}
\begin{corollary}
	Sea $A$ una matriz $n\times n$, su polinomio característico es de grado $n$ en $\lambda$, y su termino independiente coincide con del $det(A)$.
\end{corollary}
Calcular los valores propios de una matriz $A={}_B(T)_B$ es hallar las raíces del polinomio característico.
\\Una vez se obtenidos los valores propios, podemos hallar los vectores propios con el sistema de ecuaciones $\left( A-\lambda Id \right)\vec{v}=0 $.
\begin{corollary}
	Sean $A$, $B\in M\left( \mathbb{R} \right)_{n\times n}$ dos matrices semejantes.
	\\Entonces $X_A(\lambda)=X_B(\lambda)$, por lo tanto, tienen iguales valores propios.\\
	Observar que esto es una condición necesaria, pero no suficiente para asegurar la semejanza entre matrices.
\end{corollary}
\begin{definition}{Polinomio característico de un operador lineal.}
	\\Sea $V$ un espacio vectorial de dimensión finita, $T:V\to V$ un operador lineal. Llamamos polinomio característico de $T$ al polinomio característico de cualquier matriz asociada a $T$.\\
	De misma manera se define la ecuación características raíces características de un operador.\\
	Lo denotamos por $X_T$.
\end{definition}
\begin{theorem}
	Sea $A={}_B(T)_B$, la matriz asociada en la base $B$ a la transformación lineal $T:V\to V$. Entonces $v$ es vector propio de $T$ con valor propio $\lambda$ si y solo si las coordenadas de $v$ en la base $B$ son una solución no trivial del sistema: $$\left( A-\lambda Id \right) coord_B\left( v \right) =\begin{pmatrix} 0 \\ \vdots \\ 0 \end{pmatrix}  $$
\end{theorem}
\begin{theorem}
	Sea $V$ un espacio vectorial sobre el cuerpo $\mathbb{K}$, $T:V\to V$ un operador lineal; $B$ una base de $V$ y $A={}_B(T)_B$. Entonces $\lambda$ es valor propio de $T$ si y solo si $\lambda\in\mathbb{K}$ y $det\left(A-\lambda Id  \right)=0$.
	\\Es decir, los valores propios de la matriz asociada y del operador lineal se comparten.
\end{theorem}\newpage
\begin{example}{Practico 2 Ejercicio 3}
	\\Se considera la matriz $A=\begin{pmatrix} 0 & 2 & 0 \\ 2 & 0 & 0 \\ 0 & 0 & 2 \end{pmatrix} $ y la transformación lineal $T:\mathbb{R} ^3 \to\mathbb{R} ^3$ tal que $_B(T)_B=A$, donde $B=\{(1,0,0),(1,1,0),(1,1,1)\}$.
	\begin{enumerate}
		\item Hallar los valores propios y los subespacios propios de $A$.
		\item Hallar los valores propios y los subespacios propios de $T$.
	\end{enumerate}
	1. $$det\left( A-\lambda Id \right)=\begin{vmatrix}
			-\lambda & 2        & 0         \\
			2        & -\lambda & 0         \\
			0        & 0        & 2-\lambda
		\end{vmatrix} =( 2-\lambda )\begin{vmatrix}
			-\lambda & 2        \\
			2        & -\lambda
		\end{vmatrix} =( 2-\lambda )\left( \lambda ^{2} -4\right)\rightarrow \lambda =\pm 2 \text{ Valores propios.}$$
	$$S_{2} =\begin{pmatrix}
			-2 & 2  & 0 \\
			2  & -2 & 0 \\
			0  & 0  & 0
		\end{pmatrix}\begin{pmatrix}
			x \\
			y \\
			z
		\end{pmatrix} =0\rightarrow -2x+2y=0\rightarrow x=y$$
	$$ \begin{array}{l}
			S_{2} =\left\{( x,y,z) \in \mathbb{R}^{3}  : x=y\right\}=\{(1,1,0),(0,0,1)\} \text{ Subespacio propio.} \\
			S_{2}\xrightarrow{b}( x,x,z) =x\underbrace{( 1,1,0)}_{\text{Vector propio}} +z\underbrace{( 0,0,1)}_{\text{Vector propio}}
		\end{array}$$
	$$S_{-2} =\begin{pmatrix}
			2 & 2 & 0 \\
			2 & 2 & 0 \\
			0 & 0 & 4
		\end{pmatrix}\begin{pmatrix}
			x \\
			y \\
			z
		\end{pmatrix} =0\rightarrow \begin{matrix}
			z=0 \\
			x=-y
		\end{matrix}$$
	$$ \begin{array}{l}
			S_{-2} =\left\{( x,y,z) \in \mathbb{R}^{3}  : x=-y\land z=0\right\}=\{\left( 1,-1,0 \right) \} \text{ Subespacio propio.} \\
			S_{-2}\xrightarrow{b}( x,-x,0) =x\underbrace{( 1,-1,0)}_{\text{Vector propio}}
		\end{array}$$
	2. Sabemos que se comparten los valores propios de la matriz asociada y de la transformación lineal, por lo tanto también son $\lambda =\pm 2$.\\
	Y tenemos por el Teorema 2.1 que:\\
	$coord_B(v_1)=\left( 1,1,0 \right) \rightarrow v_1=1\left( 1,0,0 \right) + 1(1,1,0)+0\left( 1,1,1 \right)=(2,1,0)  $\\
	$coord_B(v_2)=(0,0,1)\rightarrow v_2=0(1,0,0)+0(1,1,0)+1(1,1,1)=(1,1,1) $\\
	Por lo tanto, $S_2=\{(2,1,0),(1,1,1)\}$\\
	$coord_B(v_3)=\left( 1,-1,0 \right)\rightarrow v_3=1(1,0,0)-1(1,1,0)+0(1,1,1)=(0,-1,0) $\\
	Por lo tanto, $S_{-2}=\{(0,-1,0)\}$
\end{example}\newpage
\subsection{Transformaciones y Matrices diagonalizables.}
\begin{definition}{Transformaciones lineales diagonalizables}
	\\Sea $T:V\to V$ una transformación lineal. Se le llama diagonalizable si existe alguna base $B$ tal que la matriz $_B(T)_B$ es una matriz diagonal, es decir, una matriz en la que todos los términos fuera de su diagonal principal son nulos.
\end{definition}
\begin{definition}{Matrices diagonalizables}
	Una matriz cuadrada se llama diagonalizable si es semejante a una matriz diagonal.
\end{definition}
\begin{example}
	Sea $T:\mathbb{R} ^3 \to\mathbb{R} ^3$ definida por: $T(x,y,z)=\begin{pmatrix} 3 & 0 & 0 \\ 0 & 4 & 0 \\ -1 & -2 & 2 \end{pmatrix} \begin{pmatrix} x \\ y \\ z \end{pmatrix} $.
	Si se elige como base de $\mathbb{R} ^3$ a $B'=\{(0,1,-1),(0,0,1),(1,0,-1)\}$ resulta:
	$$T(0,1,-1)=\begin{pmatrix} 3 & 0 & 0 \\ 0 & 4 & 0 \\ -1 & -2 & 2 \end{pmatrix} \begin{pmatrix} 0 \\ 1 \\ -1 \end{pmatrix}=\begin{pmatrix} 0 \\ 4 \\ -4 \end{pmatrix}$$
	O sea, $T(0,1,-1)=4(0,1,-1)+0(0,0,1)+0(1,0,-1)$.\\
	Análogamente tenemos:\\
	$T(0,0,1)=(0,0,2)=0(0,1,-1)+2(0,0,1)+0(1,0,-1)$\\
	$T(1,0,-1)=(3,0,-3)=0(0,1,-1)+0(0,0,1)+3(1,0,-1)$\\
	Entonces $_B(T)_B=\begin{pmatrix} 4&0&0 \\ 0&2&0 \\ 0&0&3 \end{pmatrix} $, por lo tanto $T$ es diagonalizable.
\end{example}
\begin{theorem}
	$T$ es diagonalizable si y solo si existe alguna base de $V$ constituida por vectores propios de $T$. En este caso la matriz asociada en una base de vectores propios (tomada como base de partida y llegada) es diagonal.
\end{theorem}
\begin{corollary}
	Si $T$ es diagonalizable, su forma diagonal
	$$\begin{pmatrix}
			\lambda _{1} & 0            & \cdots & 0            \\
			0            & \lambda _{2} & \cdots & 0            \\
			\vdots       & \vdots       & \ddots & \vdots       \\
			0            & 0            & \cdots & \lambda _{n}
		\end{pmatrix}$$
	es única a menos del orden de $\lambda _1, \lambda _2, \dots, \lambda _n$ que son los valores propios de $T$.
\end{corollary}
\begin{theorem}
	Una matriz $A\in M_{n\times n}$ es diagonalizable si tiene $n$ vectores propios linealmente independientes.
\end{theorem}
\begin{theorem}
	Sea $T:V\to V$ una transformación lineal; $\lambda _1, \dots \lambda _n$ valores propios dos a dos distintos; y $v_1, v_2, \dots, v_n$ vectores propios correspondientes a cada uno de los valores propios anteriores. Entonces, $\{v_1,v_2,\dots,v_n\}$ es un conjunto linealmente independiente.
\end{theorem}
\begin{corollary}
	Si $dim(V)=n$, $T:V\to V$ tiene $n$ valores propios todos distintos entonces $T$ es diagonalizable.\\
	De todas maneras, el reciproco de lo anterior es falso, puede ser diagonalizable teniendo valores propios iguales, un contra ejemplo es el siguiente.
\end{corollary}\newpage
\begin{example}
	¿Es diagonalizable la matriz $A=\begin{pmatrix} 2&-5&0\\0&7&0\\0&-5&2 \end{pmatrix} $?
	$$det(A-\lambda Id)=\begin{vmatrix}
			2-\lambda & -5        & 0         \\
			0         & 7-\lambda & 0         \\
			0         & -5        & 2-\lambda
		\end{vmatrix} =( 2-\lambda )^{2}( 7-\lambda )$$
	Entonces $\begin{matrix}
			\lambda =2 & \text{raíz doble} \\
			\lambda =7 &
		\end{matrix}$
	\\Para $\lambda =2$, los vectores propios asociados $(x,y,z)$ cumplen la ecuación $y=0$.\\
	Por lo tanto $S_2=\{(x,0,z)\in\mathbb{R}^3\}$\\
	Para $\lambda=7$ los vectores propios verifican el sistema $$\begin{cases}
			-5x-5y=0 \\
			-5y-5z=0
		\end{cases}\rightarrow x=-y=z$$
	Por lo tanto $S_7=\{\left( x,-x,x \right)\in\mathbb{R}^3 \}$.
	\\Resulta entonces que $\{(1,0,0),(0,0,1),(1,-1,1)\}$ es una base de vectores propios de $A$, y por lo tanto $A$ resulta diagonalizable. Una forma diagonal es $D=\begin{pmatrix} 2&0&0\\0&2&0\\0&0&7 \end{pmatrix} $.
\end{example}
\begin{definition}{Multiplicidad algébrica.}
	\\Sea $\lambda$ un valor propio de $A={}_B(T)_B$, la multiplicidad algebraica es el numero de factores $(\lambda-t)$ en el polinomio característico tras la factorización.
	\\Lo denotamos como $ma(\lambda)$
\end{definition}
\begin{example}
	Sea $X_A(\lambda)=(\lambda -1)(\lambda -4)(\lambda -3)^2$, $ma(1)=1$, $ma(4)=1$, $ma(3)=2$
\end{example}
\begin{definition}{Multiplicidad geométrica.}
	\\La multiplicidad geométrica de un valor propio es la dimensión del espacio propio asociado.\\
	Lo denotamos como $mg(\lambda)$, y lo podemos calcular como, $mg(\lambda)=n-rango(A-\lambda Id)$.
	\\Siendo $A\in M_{n\times n}$
\end{definition}
\begin{corollary}
	Para cualquier valor propio $\lambda$ se cumple que $1\le mg(\lambda) \le ma(\lambda) \le n$
\end{corollary}
\begin{theorem}
	$T$ es diagonalizable si y solo si los valores propios pertenecen al cuerpo y ademas se cumple que $ma(\lambda)=mg(\lambda)$.
\end{theorem}
\begin{theorem}
	Si una matriz $A$ es diagonalizable, existe $P$ invertible (cuyas columnas son los vectores propios de $A$) tal que $P^{-1}AP=D \rightarrow A=PDP^{-1}$, donde $D$ es una matriz diagonal.
\end{theorem}
Volviendo al ejemplo anterior, podemos probar si $A=PDP^{-1}$, siendo $P=\begin{pmatrix} 1&0&1\\0&0&-1\\0&1&1 \end{pmatrix} $.
\begin{theorem}
	Para hallar $A^n$ basta calcular $A^n=PD^nP^{-1}$.\\
	Siendo $D^n$ de la forma $\begin{pmatrix}
			\lambda _{1}^{n} & 0                & \cdots & 0                \\
			0                & \lambda _{2}^{n} & \cdots & 0                \\
			\vdots           & \vdots           & \ddots & \vdots           \\
			0                & 0                & \cdots & \lambda _{m}^{n}
		\end{pmatrix}$
\end{theorem}\newpage
\subsection{Teorema de Gershgorin}
\begin{definition}
	Dada una matriz $A=(a_{ij})\in M(\mathbb{C})_{n\times n}$ llamaremos $r_i$ a la suma de los módulos de las entradas de la fila i-esima de $A$, exceptuando la entrada ubicada en la diagonal.
	$$r_i=\sum_{j\neq i} |a_{ij}|$$
	Sea $C_i$ el disco de centro $a_{ii}$ y radio $r_i$ $$C_i=\{z\in\mathbb{C} : |z-a_{ii}|\le r_i\}$$
\end{definition}
\begin{theorem}{Teorema de Gershgorin}
	\\Sea $A\in M(\mathbb{C})_{n\times n}$
	\begin{enumerate}
		\item Si $\lambda$ es valor propio de $A$ entonces $\lambda\in\cup _{i}C_i$.\\Es decir, cada valor propio se encuentra en algún circulo $C_i$
		\item Si $M=C_{i_1} \cup\dots\cup C_{i_m}$ es disjunta con la unión de los restantes discos, entonces en $M$ hay exactamente $m$ valores propios de $A$.
	\end{enumerate}
\end{theorem}
\begin{corollary}
	Como $A$ y $A^T$ comparten los valores propios, podemos analizar tanto las filas como las columnas para aproximar mejor los valores propios.
\end{corollary}
\begin{example}
	Consideremos $A=\begin{pmatrix} 10&1&-1\\3&25&0\\2&1&32 \end{pmatrix} $.
	\\Para esta matriz como $a_{11}=10+0i$ el circulo va a ser de centro (10,0), en un caso que sea de la forma $a_{nn}=x+yi$ el circulo va a ser de centro $(x,y)$, $r_1=|1|+|-1|=2$, $a_{22}=25$, $r_2=3$, $a_{33}=32$ y $r_3=3$, por lo tanto $$ \begin{array}{l}
			C_{1} =\{z\in \mathbb{C} :|z-10|\leq 2\} \\
			C_{2} =\{z\in \mathbb{C} :|z-25|\leq 3\} \\
			C_{3} =\{z\in \mathbb{C} :|z-32|\leq 3\}
		\end{array}$$
	En este caso como los centros de todos los círculos están sobre el eje real, todos los valores propios van a ser reales.\\
	Tenemos que los valores propios van a estar en los intervalos $[8,12], [22,28], [29,35]$, y como los círculos son disjuntos, tenemos que son 3 valores propios todos distintos, por lo que tenemos que $A$ es diagonalizable.
\end{example}\newpage
\section{Forma Canónica de Jordan}
\subsection{Subespacios invariantes}
\begin{definition}
	Dado un operador $T:V\to V$ decimos que un subespacio $W$ es invariante si $T(W)\subset W$.\\
	Es decir si hay un vector $v\in W$, entonces $T(v)\in W$
\end{definition}
\begin{corollary}
	Sean $V$ un espacio de dimensión finita, $T:V\to V$ una transformación lineal y $W$ un subespacio invariante. Entonces existe una base $B$ de $V$ tal que
	$$_B(T)_B=\left( \begin{array}{ll}
			\multicolumn{1}{l|}{A} & \multicolumn{1}{|l}{B} \\ \cline{1-2}
			\multicolumn{1}{l|}{0} & \multicolumn{1}{|l}{C}
		\end{array} \right) $$
\end{corollary}
\begin{corollary}
	Si $V=U\oplus W$ y ambos subespacios son invariantes, entonces existe una base donde
	$$
		_B(T)_B=\left( \begin{array}{ll}
			\multicolumn{1}{l|}{A} & \multicolumn{1}{|l}{0} \\ \cline{1-2}
			\multicolumn{1}{l|}{0} & \multicolumn{1}{|l}{C}
		\end{array} \right)
	$$
	El caso mas sencillo de subespacio invariante es cuando $v\neq 0$ es vector propio, en ese caso el subespacio generado por $v$ es invariante.
\end{corollary}
\subsection{Forma Canónica de Jordan}
\begin{definition}{Sub-bloque de Jordan.}
	\\Se le llama sub-bloque de Jordan de un valor propio $\lambda$ y tamaño $k\times k$ de la forma
	$$SJ_k\left( \lambda \right) =\begin{pmatrix}
			\lambda & 0       & \cdots & \cdots & 0       \\
			1       & \lambda & \cdots & \cdots & 0       \\
			0       & 1       & \ddots & \cdots & \vdots  \\
			\vdots  & \vdots  & \ddots & \ddots & \vdots  \\
			0       & 0       & \cdots & 1      & \lambda
		\end{pmatrix} $$
\end{definition}
\begin{example}
	El sub-bloque de Jordan de tamaño $3$ y valor propio $2$ es:
	$$SJ_3(2)=\begin{pmatrix}
			2 & 0 & 0 \\
			1 & 2 & 0 \\
			0 & 1 & 2
		\end{pmatrix}$$
\end{example}
\begin{definition}{Bloque de Jordan.}
	\\Se llama bloque de Jordan de tamaño $m\times m$ y valor propio $\lambda$ a una matriz cuadrada $m\times m$ por
	\begin{itemize}
		\item Uno o mas bloques de Jordan de distintos o igual tamaño con el mismo valor propio.
		\item Ceros en los restantes términos del ejemplo del bloque.
	\end{itemize}
\end{definition}
\begin{example}
	Un bloque de Jordan de tamaño $5$ y valor propio $2$ puede ser:
	$$
		J(2)=\left(\begin{array}{lllll}
			\cline{1-1}
			\multicolumn{1}{|l|}{2} & 0 & 0                      & 0 & 0                      \\ \cline{1-3}
			\multicolumn{1}{l|}{0}  & 2 & \multicolumn{1}{l|}{0} & 0 & 0                      \\
			\multicolumn{1}{l|}{0}  & 1 & \multicolumn{1}{l|}{2} & 0 & 0                      \\ \cline{2-5}
			0                       & 0 & \multicolumn{1}{l|}{0} & 2 & \multicolumn{1}{l|}{0} \\
			0                       & 0 & \multicolumn{1}{l|}{0} & 1 & \multicolumn{1}{l|}{2} \\ \cline{4-5}
		\end{array}\right)
	$$

\end{example}\newpage
\begin{theorem}{Forma Canónica de Jordan.}
	Sean $V$ un espacio vectorial de dimensión $n$ sobre el cuerpo $\mathbb{K}$ y $T$ un operador lineal tal que su polinomio característico $$X_T(\lambda)=(\lambda - \lambda _1)^{m_1}(\lambda -\lambda _2)^{m_2}\dots (\lambda - \lambda _q)^{m_q}$$
	Entonces existe una base $B$ de $V$ tal que
	$$_B(T)_B=\begin{pmatrix}
			J(\lambda _1) &               &        &               \\
			              & J(\lambda _2) &        &               \\
			              &               & \ddots &               \\
			              &               &        & J(\lambda _q)
		\end{pmatrix}$$
	Donde cada bloque $J(\lambda _i)$ es un bloque de Jordan de valor propio $\lambda _i$, cuyo tamaño es su magnitud algebraica.
\end{theorem}
\begin{corollary}
	Supongamos que $\lambda$ es un valor propio de una transformación lineal $T$. Entonces la cantidad de sub-bloques de Jordan de valor propio $\lambda$ es igual a  la multiplicidad geométrica de $\lambda$.\\
	Un sub-bloque debe también terminar siempre con una columna correspondiente a un vector propio, y por lo tanto la cantidad de sub-bloques coincide con la cantidad de vectores propios que existan en la base $B$.
\end{corollary}
\begin{example}{Hallar bases de Jordan}
	\\Sea un operador lineal de $\mathbb{R}^3$ cuya matriz asociada en la base canónica es
	$$A=\begin{pmatrix}
			2 & -4 & 1 \\
			0 & 7  & 0 \\
			0 & -5 & 2
		\end{pmatrix}$$
	no es diagonalizable.\\
	Sus valores propios son $2$ y $7$, $S_2=\{(1,0,0)\}$ y $S_7=\{(1,-1,1)\}$ y $ma(2)=2$. En consecuencia, la forma de Jordan es $$J=\begin{pmatrix}
			2 & 0 & 0 \\
			1 & 2 & 0 \\
			0 & 0 & 7
		\end{pmatrix} $$
	Para calcular una base de Jordan $\{v_1,v_2,v_3\}$ de $\mathbb{R}^3$ asociada al operador, observemos que $Av_1=2v_1+v_2$, $Av_2=2v_2$ y $Av_3=7v_3$. Por lo tanto $v_2$ y $v_3$ son vectores propios asociados a $2$ y $7$ respectivamente, por lo cual podemos elegir $v_2=(1,0,0)$ y $v_3=(1,-1,1)$. Ahora podemos decir que $v_1=(x,y,z)$. Como sabemos que $Av_1=2v_1+v_2$ por lo tanto $(A-2I)v_1=v_2$, entonces:
	$$(A-2I)v_1=\begin{pmatrix} 0&-4&1\\0&5&0\\0&-5&0 \end{pmatrix} \begin{pmatrix} x\\y\\z \end{pmatrix} =\begin{pmatrix} 1\\0\\0 \end{pmatrix} $$
	Resolviendo el sistema tenemos que $v_1=(x,0,1)$ con $x\in\mathbb{R}$, entonces podemos elegir $v_1=(0,0,1)$, por lo cual tenemos que la base de Jordan $B$ resulta $B=\{(0,0,1),(1,0,0),(1,-1,1)\}$.
\end{example}\newpage
\subsection{Teorema de Cayley-Hamilton}
Un operador $T$ o una matriz $A$ verifican su polinomio característico, esto significa que, ya sea el operador $X_T(T)$ es el operador nulo o que $X_A(A)$ es la matriz nula.
\begin{theorem}
	Sea $T:V\to V$ un operador, $\lambda$ un valor propio, $v$ un vector propio asociado a $\lambda$ y $P(x)$ un polinomio. Entonces se cumple que el vector $v$ es vector propio del operador $P(T)$ con valor propio $P(\lambda)$.
\end{theorem}
\begin{corollary}
	Sea $P(x)=X_T(x)$ el polinomio característico de $T$, y $\mu$ un valor propio y $v$ un vector asociado a $\mu$, entonces:
	$$X_T(T)(v)=X_T(\mu)v=0v=0$$
\end{corollary}
Para transformaciones no diagonalizables hay que tener mas cuidado. Podemos observar que si $\mu$ es valor propio, el polinomio característico se factoriza como $$X_T(x)=(x-\mu)P(x)$$
Por lo tanto, $X_T(T)=(T-\mu Id)P(T)$. Como $v$ es vector propio asociado a $\mu$, tenemos $$X_T(T)(v)=P(T)(Tv-\mu v)=0$$
Cuando $v$ no es vector propio, debemos proceder de otra manera.
\begin{corollary}
	Dados un operador $T$ en $V$ y $v$ un vector no nulo en $V$. Entonces existe $k\in\mathbb{N}$ que cumple que
	$$\begin{array}{ l l }
			\left\{v,T( v) ,\dotsc ,T^{k}( v)\right\}              & \text{Es linealmente independiente} \\
			\left\{v,T( v) ,\dotsc ,T^{k}( v) ,T^{k+1}( v)\right\} & \text{Es linealmente dependiente}
		\end{array}$$
\end{corollary}
\begin{definition}
	Notamos con $[v]_T$ al subespacio generador por $\{v,T(v),\dots,T^k(v)\}$.
\end{definition}
\begin{corollary}
	Sea $T^{k+1}(v)=\sum_{i=0}^{k} a_iT^i(v)$. El subespacio $[v]_T$ es invariante por $T$ y la matriz asociada a $T$ en la base $\{v,Tv,\dots,T^kv\}$ del subespacio es:
	$$\begin{pmatrix}
			0      & 0      & \cdots & 0      & a_{0}  \\
			1      & 0      & \cdots & 0      & a_{1}  \\
			\vdots & \ddots & \vdots & \vdots & \vdots \\
			0      & 0      & \cdots & 1      & a_{k}
		\end{pmatrix}$$
\end{corollary}
\begin{theorem}{Cayley-Hamilton.}
	\\Sean $T:V\to V$ un operador en un espacio de dimensión finita $V$ y $X_T(x)$ el polinomio característico de $T$. Entonces el operador $X_T(T)$ es el operador nulo.
\end{theorem}\newpage
\section{Producto interno y norma}
\subsection{Producto interno}
\begin{definition}
	Sea $V$ un $\mathbb{K}$-espacio vectorial, una función de dos variables $$\langle \ ,\ \rangle :V\times V\rightarrow \mathbb{K}$$
	es un producto interno en $V$ si verifica:
	\begin{enumerate}
		\item $\langle u+v, w \rangle = \langle u,w \rangle + \langle v,w \rangle$ $\forall u,v,w\in V$.
		\item $\langle\alpha u,v \rangle = \alpha\langle u,v \rangle$ $\forall u,v\in V$, $\forall\alpha\in\mathbb{K}$
		\item $\langle u,v\rangle = \overline{\langle v,u \rangle}$ $\forall u,v\in V$, la barra indica el complejo conjugado.
		\item $\langle u,u \rangle$ real y $\langle u,u \rangle\ge 0$ $\forall u\in V$, y $\langle u,u \rangle =0\Leftrightarrow u=0$
	\end{enumerate}
\end{definition}
\begin{corollary}
	La propiedad 1, dice que la función $\langle \ ,\ \rangle$ es aditiva en la primera componente (se sobrentiende que la segunda componente permanece fija).
\end{corollary}
\begin{corollary}
	La propiedad 2, dice que la función $\langle \ ,\ \rangle$ es homogénea en la primera componente, mientras que la segunda permanece fija.
	\\Cuando se cumplen las propiedades 1 y 2 se dice que la función $\langle \ ,\ \rangle$ es lineal en la primera componente.
\end{corollary}
\begin{corollary}
	Si $\langle \ ,\ \rangle$ es un producto interno en el $\mathbb{K}$-espacio vectorial $V$ se tiene que
	\begin{enumerate}[label=\alph*)]
		\item $\langle u,v+w \rangle = \langle u,v \rangle + \langle u,w \rangle$ $\forall u,v,w\in V$
		\item $\langle u,\alpha v \rangle = \overline{\alpha}\langle u,v\rangle$ $\forall u,v\in V$, $\forall\alpha\in\mathbb{K}$
		\item $\langle u,\vec{0}\rangle =\langle\vec{0} ,u\rangle =0$ $\forall u\in V$
	\end{enumerate}
\end{corollary}
\begin{definition}
	Definimos el producto interno usual de un cuerpo $\mathbb{K}^n$ como $$\langle u,v \rangle = u_1 \overline{v_1} +u_2\overline{v_2}+\dots+u_n\overline{v_n}$$
\end{definition}
\subsection{Norma}
\begin{definition}
	Sea $V$ un espacio vectorial. Una norma en $V$ es una función tal que a cada vector $v$ le hace corresponder un real indicado como $||v||$, y cumple:
	\begin{enumerate}
		\item $||\lambda v||=|\lambda |||v||$ $\forall\lambda\in\mathbb{K}$ $\forall v\in V$.
		\item $||v|| \ge 0$ $\forall v\in V$ $||v||=0 \Leftrightarrow v=\vec{0}$
		\item $||v+w||\le ||v||+||w||$ Desigualdad triangular.
	\end{enumerate}
\end{definition}
\begin{theorem} 
	Todo espacio vectorial con producto interno es un espacio vectorial normado definiendo la norma de la siguiente manera: $$||v||=\sqrt{\langle v,v \rangle}$$ A esta norma la llamamos norma inducida por el producto interno.
\end{theorem} 
\begin{corollary}
	El reciproco no es cierto, es decir, hay normas que no son normas inducidas por ningún producto interno en $V$.
\end{corollary}
\begin{corollary}
	En el caso que se considere el espacio ordinario, la norma de un vector referida al producto escalar coincide con su módulo.
\end{corollary}
\begin{theorem}{Desigualdad de Cauchy-Schwarz}
	Sea $V$ un espacio con producto interno, para todo par de vectores $v,w\in V$ se tiene $$|\langle v,u \rangle| \le ||v|| \ ||u||$$ donde la norma es la inducida por el producto interno. La igualdad vale si y solo si $\{v,u\}$ es linealmente independiente.
\end{theorem}
\begin{corollary}
	Si $V$ es un espacio vectorial real con producto interno, y $v, w$ son dos vectores no nulos, la desigualdad de Cauchy-Schwarz permite asegurar que $$-1\le \frac{\langle v,w \rangle}{||v|| \ ||w||} \le 1$$ por ello se define el angulo $\alpha$ entre los vectores $v, w$ por $cos(\alpha)=\frac{\langle v,w \rangle}{||v|| \ ||w||}$
\end{corollary}
\begin{corollary}{Desigualdad triangular}
	\\Si $V$ es un espacio vectorial con producto interno $v,w\in V$, se tiene $||v+w||\le ||v||+||w||$.
\end{corollary}
\subsection{Ortogonalidad y ortonormalidad}
\begin{definition}
	Sea $V$ un espacio vectorial con producto interno. Dados $v,\ w\in V$, se dice que $v$ y $w$ son ortogonales, y se escribe como $v \bot w$, cuando $\langle v,w \rangle =0$.
\end{definition}
Esta definición coincide con la ortogonalidad en el espacio ordinario, trabajando con el producto interno usual.
\begin{definition}
	Sea $A \subset V$. Se dice que $A$ es un conjunto ortogonal si los elementos de $A$ son ortogonales dos a dos, o sea, $\forall v,\ w\in A,\ v\neq w$ se cumple $v \bot w$.\\
	Si ademas $||v||=1\ \forall v\in A$ se dice que $A$ es un conjunto ortonormal.
\end{definition}
\begin{corollary}{ \ }
	\begin{itemize}
		\item $0\bot v\ \forall v\in V$.
		\item $v\bot v \Leftrightarrow v=0$.
	\item Si $A$ es un conjunto ortogonal y $\vec{0}\notin A$ el conjunto: $$\left\{\frac{1}{||v||} v\ /\ v\in A\right\}$$ es ortonormal. A este proceso se le llama normalizar.
	\end{itemize}
\end{corollary}
\begin{theorem}
	Todo conjunto ortogonal que no tiene el vector nulo es linealmente independiente. Sea $V$ un $\mathbb{K}$-espacio vectorial con producto interno y $\{v_1 ,\dots , v_r\}$ un conjunto ortogonal tal que $v_i\neq\vec{0}$ con $i=1, \dots , r$. Entonces $\{v_1, \dots ,v_r\}$ es linealmente independiente.
\end{theorem}
\begin{theorem}{Pitágoras.}
	\\Sea $V$ un producto interno y $\{v_1,\dots,v_r\}$ un conjunto ortogonal. Entonces $||\sum_{i=1}^{r} v_i||^2=\sum_{i=1}^{r} ||v_i||^2$
\end{theorem}\newpage
\begin{theorem}{Método de ortonormalización de Gram-Schmidt.}
	\\Sean $V$ un espacio vectorial con producto interno y $\{v_1,\dots,v_k\}$ una base de $V$. Entonces existe \newline $B=\{y_1,\dots,y_n\}$ tal que $B$ es una base ortonormal de $V$ y $[v_1,\dots,v_k]=[y_1,\dots,y_k]\ \forall k=1,\dots,n$.\\
	\underline{Método:}
	\\Tomamos $u_1=v_1$, entonces $[v_1]=[u_1]$.
	\\Sea $u_2=v_2-cu_1$ donde $$c=\frac{\langle v_2,u_1 \rangle}{||u_1||^2}$$
	\\Generalizando decimos que $$u_k=v_k-c_{k-1}u_{k-1}-\dots-c_1u_1$$ donde $c_j=\frac{\langle v_k,u_j \rangle}{||u_j||^2}\ \forall k=2,\dots,n$, se obtiene un sistema $\{u_1,\dots,u_n\}$ ortogonal tal que $[u_1,\dots,u_n]=[v_1,\dots,v_k]\ \forall k=1,\dots,n$.
	\\Finalmente tomando $y_j=\frac{1}{||u_j||}u_j$ se tiene que $B=\{y_1,\dots,y_n\}$ en condiciones adecuadas.
\end{theorem}
\begin{corollary}
	Todo espacio vectorial de dimensión finita con producto interno tiene una base ortonormal.
\end{corollary}
\begin{corollary}
	Este método es también aplicable a subespacios vectoriales.
\end{corollary}
\begin{theorem}{Propiedades de las bases ortonormales.}
	\\Sea $V$ un espacio vectorial con producto interno y $\{v_1,\dots,v_n\}$ una base ortonormal. Entonces:
	\begin{enumerate}
		\item Si $v=\sum_{i=1}^{r} \alpha_iv_i$ y $w=\sum_{i=1}^{n} \beta_iv_i$ entonces $\langle v,w \rangle=\sum_{i=1}^{n} \alpha \overline{\beta_i}$.
		\item $\forall v\in V$ se tiene que: $v=\langle v,v_1 \rangle v_1+\dots+\langle v,v_n \rangle v_n$.
		\item $\forall v\in V$ se tiene que: $||v||^2=\sum_{i=1}^{n} |\langle v,v_i \rangle|^2$.
	\end{enumerate}
\end{theorem}
\begin{corollary}
    Para hallar un producto interno para el cual $\{v_1,v_2,\cdots,v_n\}$ es una base ortonormal primero recordemos que todos sus vectores son ortogonales entre ellos y ademas $\langle v_i,v_i\rangle=1$, con esto, llamamos
    $$\begin{matrix}
v=( x_{1} ,x_{2} ,\cdots ,x_{n}) =\alpha _{1} v_{1} +\alpha _{2} v_{2} +\cdots +\alpha _{n} v_{n}\\
w=( y_{1} ,y_{2} ,\cdots ,y_{n}) =\beta _{1} v_{1} +\beta _{2} v_{2} +\cdots +\beta _{n} v_{n}
\end{matrix}$$
Luego, se despeja $\alpha_i,\ \beta_i$ respectivamente sobre $x_i$ e $y_i$, para después hacer $$\langle v,w\rangle=\langle (x_1,\cdots,x_n),(y_1,\cdots,y_n\rangle=\langle(\alpha_1v_1+\cdots+\alpha_nv_n),(\beta_1v_1+\cdots+\beta_nv_n)\rangle=\langle\alpha_1v_1,\beta_1v_1\rangle+\cdots+\langle\alpha_nv_n,\beta_nv_n\rangle$$
Por las propiedades que comente al principio, nos queda $\alpha_1\overline{\beta_1}+\cdots+\alpha_n\overline{\beta_n}$, aquí deshacemos la sustitución cambiando $\alpha$ y $\beta$ por sus valores respecto a $x_i$ e $y_i$, para así hallar el producto interno.
\end{corollary}
\subsection{Complemento ortogonal}
\begin{definition}
    Sea $V$ un espacio vectorial con producto interno, si $S\subset V$. Llamamos complemento ortogonal de $S$ al conjunto $$\begin{array}{ c l }
S^{\bot } & =\{v\in V\ :\ v\bot s\ \ \forall s\in S\}\\
 & =\{v\in V\ :\ \langle v,s\rangle =0\ \ \forall s\in S\}
\end{array}$$
Note que no se pide que $S$ sea un subespacio vectorial de $V$.
\end{definition}
\begin{corollary}
    Sea $V$ un espacio vectorial con producto interno y $S$ un sub-conjunto de $V$. Entonces $S^\bot$ es un subespacio vectorial de $V$.
\end{corollary}
\begin{corollary}
    Si $V$ es un espacio vectorial de dimensión finita con producto interno y $B=\{s_1,s_2,\cdots,s_r\}$ es una base de un subespacio $S$ entonces $v\in S^\bot\ \Leftrightarrow\ v\bot s_i\ \forall i=1,2,\cdots,r$.
\end{corollary}
\begin{theorem}
    Sea $V$ un espacio vectorial de dimensión finita, y $W$ un subespacio de $V$:$$dim(V)=dim(W)+dim(W^\bot)$$
\end{theorem}
\begin{corollary}
    Sea $V$ un espacio vectorial con producto interno, y $S$ un sub-espacio vectorial de dimensión finita. Entonces $V=S\oplus S^\bot$.
\end{corollary}
Propiedades:
\begin{itemize}
    \item Si $A\subset B \Leftarrow B^\bot \subset A^\bot$.
    \item $A^\bot = [A]^\bot$.
    \item $A\subset (A^\bot)^\bot$.
    \item $S=(S^\bot)^\bot$.
    \item $(S+W)^\bot =S^\bot \cap W^\bot$.
    \item $(S\cap W)^\bot = S^\bot + W^\bot$.
\end{itemize}
\subsection{Proyección ortogonal}
Sea $V$ un espacio vectorial con producto interno, $S$ un subespacio tal que $V=S\oplus S^\bot$. Eso implica que dado $v\in V$ existen y son únicos $v_S\in S,\ v_{S^\bot}$ tales que $$v=v_S+v_{S^\bot}$$
\begin{definition}
    Dado $v\in V$ llamamos proyección ortogonal de $v$ sobre el subespacio $S$ al vector $P_S(v)=v_S$.
    \\Si $V$ tiene dimensión finita y $B_S=\{s_1,\cdots,s_k\}$ es una base ortonormal de $S$, entonces $$P_S(v)=\sum_{i=1}^{k} \langle v,s_i\rangle s_i\ ;\ s_i\in S$$
\end{definition}
\begin{corollary}
    La definición de proyección ortogonal no depende de la base elegida.\\
    Como vimos en el corolario 4.6.3; $P_S(v)$ es el único vector de $S$ tal que sumado con un vector de $S^\bot$ da $v$.
\end{corollary}
\begin{corollary}
    Como $V=S\oplus S^\bot$ podemos hallar la proyección, usando que la diferencia $v-s$ esté en $S^\bot$. Es más si se tiene una proyección, se tiene la proyección sobre el complemento ortogonal como lo indica el siguiente corolario.
\end{corollary}
\begin{corollary}
    Del mismo corolario 4.6.3, observando que $(S^\bot)^\bot=S$ se desprende que: $$v=P_S(v)+P_{S^\bot}(v)$$
\end{corollary}
\begin{theorem}
    Sea $V$ un espacio vectorial con producto interno y $S$ un sub-espacio de dimensión finita.\\
    Entonces: $||v-P_S(v)||\le ||v-s||\ \forall s\in S$.
\end{theorem}
\begin{corollary}
    El vector $P_S(v)$ es el vector de $S$ que mejor se aproxima a $v$, en el sentido del teorema anterior. En el sentido de que $||v-P_S(v)||$ hace mínima a $||v-s||$. Es decir, si queremos calcular la mínima, basta calcular $||v-P_S(v)||$.
\end{corollary}
\begin{corollary}
    Sea $V$ un espacio vectorial de dimensión finita con producto interno, $S\subset V$ un subespacio vectorial y $P_S(v)$ la proyección ortogonal de $v$ sobre $S$; es decir $P_S(v)$ es el único vector que verifica que $P_S(v)\in S$ y $v-P_S(v)\in S^\bot$. Se cumple:
    \begin{enumerate}
        \item $P_S(s)=s\ \forall s\in S$.
        \item $P_S(v)=\vec{0}\ \forall v\in S^\bot$.
        \item $P_S^2=P_S\circ P_S=P_S$.
        \item La función $P_S:V\to V$ dada por $v\xrightarrow{P_{S}} P_{S}( v)$ es una transformación lineal.
        \item $Im(P_S)=S$ y $Ker(P_S)=S^\bot$
        \item $||v||^2=||P_S(v)||^2+||P_{V^\bot}(v)||^2\ \forall v\in V$.
        \item $P_S(v)||\le ||v||$.
        \item $\langle v,P_S(v)\rangle = ||P_S(v)||^2\ \forall v\in V$.
    \end{enumerate}
\end{corollary}\newpage
\section{Transformaciones en espacios con producto interno}
\begin{corollary}
    Sea $V$ un espacio con producto interno. Si $\langle v,w_1\rangle=\langle v,w_2\rangle$ para todo $v\in V$ entonces $w_1=w_2$.
\end{corollary}
\begin{theorem}{Teorema de Riesz}
    \\Dada $T:V\to\mathbb{K}$ lineal y $V$ tal que si $M$ es un subespacio de $V$, se cumple que $V=M\oplus M^\bot$, entonces existe un único $w\in V$ tal que $$T(v)=\langle v,w\rangle$$
\end{theorem}
\subsection{Adjunta de una transformación}
\begin{theorem}
    Sean $V$ y $W$ espacios con producto interno y $T:V\to W$ una transformación lineal. Entonces existe una única transformación lineal $T^*:W\to V$ tal que: $$\langle T(v),w\rangle_W=\langle v,T^*(w)\rangle_V$$
\end{theorem}
\begin{definition}
    La transformación $T^*$ es llamada la adjunta de $T$.
\end{definition}
\begin{corollary}
    Propiedades del adjunto:
    \begin{enumerate}
        \item $(T+S)^*=T^*+S^*$.
        \item $(\alpha T)^*=\overline{\alpha}T^*$.
        \item $(TS)^*=S^*T^*$.
        \item $T$ es invertible si y sólo si $T^*$ lo es y $(T^*)^{-1}=(T^{-1})^*$.
        \item $(T^*)^*=T$.
        \item $\lambda$ es valor propio de $T\Leftrightarrow\overline{\lambda}$ es valor propio de $T^*$.
        \item $Ker(T^*)=(Im(T))^\bot$.
        \item $Im(T^*)=(Ker(T))^\bot$.
        \item $Ker(T^*T)=Ker(T)$.
        \item $dim(Im(T^*T))=dim(Im(T))$.
    \end{enumerate}
\end{corollary}
\begin{corollary}
    Sean $V$ y $W$ espacios con producto interno, $B$ y $C$ bases ortonormales de $V$ y $W$ respectivamente. Entonces:$$_B(T^*)_C=\overline{[_C(T)_B]}^T$$
\end{corollary}
\subsection{Transformaciones lineales auto-adjuntas}
\begin{definition}{Transformación lineal auto-adjunta}
   \\Sea $T:V\to V$ una transformación lineal. Diremos que $T$ es auto-adjunta si y solo si $$\langle T(v),w\rangle=\langle v,T(w)\rangle\ \forall v,w\in V$$
   Es lo mismo que decir que $T$ coincide con $T^*$.
\end{definition}
\begin{theorem}
    Sea $V$ un espacio vectorial \underline{real} con producto interno, $B$ una base ortonormal de $V$ y\\ $T:V\to V$ lineal, $T$ es auto-adjunta si y solo si $_B(T)_B$ es simétrica.
\end{theorem}
\begin{corollary}
    Recordar que una matriz $A$ es simétrica cuando $A=A^t$.
\end{corollary}
\begin{definition}
    Se dice que una matriz $A$ es hermética si y solo si $A=(\overline{A})^t$ (la matriz conjugada se obtiene conjugando cada una de sus entradas).
\end{definition}
\begin{theorem}
    Sea $V$ un espacio vectorial \underline{complejo} con producto interno, $B$ una base \underline{ortonormal} de $V$ y $T:V\to V$ lineal, $T$ es auto-adjunta si y solo si $_B(T)_B$ es hermética.
\end{theorem}
\begin{theorem}
    Sea $V$ un espacio vectorial \underline{complejo} con producto interno y $T:V\to V$ lineal, $T$ es auto-adjunta si y solo si $\langle T(v),v\rangle\in\mathbb{R}\ \forall v\in V$.
\end{theorem}
\begin{theorem}
    Sea $V$ un espacio vectorial \underline{complejo} con producto interno. Si $T:V\to V$ es auto-adjunta entonces todas las raíces del polinomio característico de $T$ son reales.
\end{theorem}
\begin{theorem}
    Sea $A\in M(\mathbb{C})_{n\times n}$. Si $A$ es hermética entonces todas las raíces del polinomio característico de $A$ son reales.
\end{theorem}
\begin{theorem}
    Sea $A\in M(\mathbb{R})_{n\times n}$. Si $A$ es simétrica entonces todas las raíces del polinomio característico de $A$ son reales.
\end{theorem}
\begin{theorem}
    Sea $V$ un espacio vectorial real con producto interno. Si $T:V\to V$ auto-adjunta entonces todas las raíces del polinomio característico $T$ son reales.
\end{theorem}
\begin{theorem}{Teorema espectral para transformaciones auto-adjuntas}
    \\Sea $V$ un espacio vectorial, real o complejo, de dimensión finita. Si $T:V\to V$ es una transformación lineal auto-adjunta, entonces existe una base ortonormal de $V$ formada por vectores propios de $T$.
\end{theorem}
\begin{corollary}
    Sea $V$ un espacio vectorial real o complejo, de dimensión finita. Si $T:V\to V$ es una transformación lineal auto-adjunta y $\lambda_1,\ \lambda_2$ son valores propios distintos de $T$, entonces los subespacios propios asociados a dichos valores propios son perpendiculares.
\end{corollary}
\begin{theorem}
    Si existe una base ortonormal $B$ del espacio $V$ formada por vectores propios y las raíces de $X_T$ son reales entonces $T$ es auto-adjunta.
\end{theorem}
\begin{theorem}{Teorema espectral para matrices simétricas}
    \\Si $A\in M(\mathbb{R})_{n\times n}$ es una matriz simétrica, entonces $\exists P\in M(\mathbb{R})_{n\times n}$ invertible con $P^{-1}=P^t$ tal que $A=PDP^t$ siendo $D\in M(\mathbb{R})_{n\times n}$ una matriz diagonal. 
\end{theorem}
\end{document}
