\documentclass[10pt]{article}
\usepackage[spanish]{babel}
\usepackage{amsmath, amssymb, amsthm, tikz}
\usepackage[margin=1in]{geometry}
\usepackage[shortlabels]{enumitem}
\usetikzlibrary{babel}

\theoremstyle{definition}
\newtheorem{definition}{Definición}[section]
\newtheorem{theorem}{Teorema}[section]
\newtheorem{corollary}{Corolario}[theorem]
\newtheorem{example}{Ejemplo}[section]

\title{Cálculo Diferencial e Integral en varias variables}
\author{Santiago Sierra}

\begin{document}
\maketitle \tableofcontents \newpage
\section{Números Complejos}
\begin{definition}
	Un numero complejo es un numero de forma $z=a+bi$ y $a,b\in \mathbb{R}$, donde $i^2=-1$, conocemos los números reales $a$ y $b$ como parte real e imaginaria respectivamente del numero $z$.
	$$
		Re(z)=a \ \ \ \ Im(z)=b
	$$
\end{definition}
Se le llama $i$ a la unidad imaginaria.
Esta expresión que describimos se le llama forma binómica del numero.
\begin{definition}
	Dos números complejos $z, w$ son iguales si y solo si
	$$
		Re(z)=Re(w) \ \text{y} \ Im(z)=Im(w)
	$$
\end{definition}
\subsection{Suma y Producto de números complejos}
\begin{definition}
Dados dos números complejos $z=a+bi$ y $w=c+di$ definimos la suma de $z+w$ y el producto $zw$ mediante:
\begin{gather*}
	\begin{array}{l}
		z+w=(a+bi)+(c+di)=(a+c)+(b+d)i \\
		zw=(a+bi)(c+di)=ac+adi+bci+bdi^2=(ac-bd)+(ad+bci)i
	\end{array}
\end{gather*}
\end{definition}
Ejemplo:
$$
	\begin{array}{l}
		(1-i)+(4+7i)=(1+4)+(-1+7)i=5+6i \\
		(-1+3i)(2-5i)=(-1)(2-5i)+(3i)(2-5i)=(-2+5i)+(6i-15i^2)=(-2+5i)+(15+6i)=13+11i
	\end{array}
$$
\textbf{Propiedades.} Sean $z$,$w$,$v\in\mathbb{C}$
\begin{enumerate}
	\item Conmutativas: $z+w=w+z$ y $zw=wz$
	\item Asociativas: $(z+w)+v=z+(w+v)$ y $(zw)v=z(wv)$
	\item Cada numero complejo $z=a+bi$ tiene un elemento opuesto, $-z=-a-bi$, tal que $z+(-z)=0$
	\item Distributiva (del producto respecto a la suma) $z(w+v)=zw+zv$.
\end{enumerate}
\subsection{Conjugado de un complejo}
\begin{definition}
	Sea $z=a+bi$ un numero complejo. Se define el conjugado de $z$ y se representa por $\overline{z}$, como el numero complejo $\overline{z}=a-bi$.
\end{definition}
Geométricamente, un complejo $z=a+bi$ se representa por el punto $P=(a,b)$, y su conjugado $\overline{z}=a-bi$ por el punto $P'=(a,-b)$\\\\
\begin{minipage}{0.3\textwidth}
	\begin{tikzpicture}[x=50pt,y=50,yscale=-1,xscale=1]
		\draw[<-] (0,0) -- (0,3);
		\draw[->] (-0.5,1.5) -- (2,1.5);
		\draw[-] (0,1.5) -- (1,2.5);
		\draw[-] (0,1.5) -- (1,0.5);
		\draw[dotted] (1,2.5) -- (1,0.5);
		\draw (1.3,0.5) node {$z$};
		\draw (1.3,2.5) node {$\overline{z}$};
		\foreach \Point in {(1,0.5),(1,2.5)}{
				\node at \Point {\textbullet};
				\draw (-0.25,0) node {Im};
				\draw (2,1.75) node {Re};
			}
	\end{tikzpicture}
\end{minipage}
\hfill
\begin{minipage}{0.65\textwidth}
	Propiedades:
	\begin{enumerate}
		\item $\overline{\overline{z}}=z$
		\item $\overline{z_1+z_2}=\overline{z_1}+\overline{z_2}$
		\item $\overline{z_1z_2}=\overline{z_1}\ \overline{z_2}$
		\item Si $z_2\neq 0$, $\overline{(\frac{z_1}{z_2})}=\frac{\overline{z_1}}{\overline{z_2}}$
		\item $|z|^2=z\overline{z}=Re(z)^2+Im(z)^2$. Por lo tanto, $|z|^2 \ge 0\ \forall z\neq 0$
		\item $z+\overline{z}=2 Re(z)$
		\item $z-\overline{z}=2i\ Im(z)$
	\end{enumerate}
	Observación, para dividir dos números complejos $\frac{z}{w}$, basta con multiplicar el numerador y denominador por el conjugado del denominador.
	$$
		\frac{z}{w}=\frac{z\overline{w}}{w\overline{w}}=\frac{z\overline{w}}{|w|^2}
	$$
\end{minipage}
\noindent \newpage
\subsection{Módulo}
\begin{minipage}{0.65\textwidth}
	\begin{definition}
		Definimos el módulo de un complejo $z=a+bi$ como el número real $|z|=\sqrt{a^2+b^2}$
	\end{definition}
	Propiedades del módulo. Sean $z_1$ y $z_2$ números complejos:
	\begin{enumerate}
		\item $|z|=0$ si, y sólo si, $z=0$
		\item $|z|=|\overline{z}|$
		\item $|z_1z_2|=|z_1||z_2|$
		\item Si $z\neq 0$, $|\frac{z_1}{z_2}|=\frac{|z_1|}{|z_2|}$
		\item \textbf{Desigualdad triangular:} $|z_1+z_2|\le |z_1|+|z_2|$
	\end{enumerate}
\end{minipage}
\hfill
\begin{minipage}{0.3\textwidth}
	\begin{tikzpicture}[x=0.75pt,y=0.75pt,yscale=-1,xscale=1]
		\draw  (0,108) -- (141,108)(17,0) -- (17,125) (134,103) -- (141,108) -- (134,113) (12,7) -- (17,0) -- (22,7)  ;
		\draw [-] (18,108) -- (90,45);
		\draw (90,45)[fill={rgb, 255:red, 0; green, 0; blue, 0}]circle [x radius= 3.35, y radius= 3.35];
		\draw (35.48,60.2) node [anchor=north west][inner sep=0.75pt]  [rotate=-319.32] [align=left] {$|z|$};
		\draw (95,30) node [anchor=north west][inner sep=0.75pt] [align=left] {z};
		\draw (0,0) node {Im};
		\draw (130,120) node {Re};
	\end{tikzpicture}
\end{minipage}
\subsection{Forma Polar}
Sabemos que cualquier complejo $z=a+bi$ puede ser considerado un punto $(a,b)$ y que cualquier punto de este tipo puede representarse con coordenadas polares $(r,\theta)$ con $r\ge0$.
\begin{definition}
	Cualquier complejo $z$ se puede representar como $z=r(cos(\theta)+i \ sen(\theta))=re^{i\theta}$, lo cual llamaremos forma polar. Siendo $r=|z|$ y $\theta=arg(z)$.
\end{definition}
\subsubsection{Argumento}
\begin{minipage}{0.65\textwidth}
	\begin{definition}
		Definimos el argumento de $z$ como la función\\\\ $\operatorname {arg}(z)={\begin{cases}\arctan \left({\frac  ba}\right)&\qquad a>0\\\arctan \left({\frac  ba}\right)+\pi &\qquad b\geq 0,a<0\\\arctan \left({\frac  ba}\right)-\pi &\qquad b<0,a<0\\+{\frac  {\pi }{2}}&\qquad b>0,a=0\\-{\frac  {\pi }{2}}&\qquad b<0,a=0\end{cases}}$
	\end{definition}
\end{minipage}
\hfill
\begin{minipage}{0.3\textwidth}
	\begin{tikzpicture}[x=0.75pt,y=0.75pt,yscale=-1,xscale=1]
		\draw (0,110) -- (135,110)(14,0) -- (14,123.5) (128.5,105) -- (135.5,110) -- (128.5,115) (9,7) -- (14,0) -- (19,7)  ;
		\draw    (14,110) -- (80,50) ;
		\draw [shift={(80,50)}][fill={rgb, 255:red, 0; green, 0; blue, 0 }](0, 0) circle [x radius= 3.35, y radius= 3.35]   ;
		\draw   (42,84) .. controls (48,89) and (52,97) .. (52,106) .. controls (52,107) and (51,110) .. (51,111) ;
		\draw (92,36.25) node [anchor=north west][inner sep=0.75pt]   [align=left] {$a+bi$};
		\draw (30,70) node [anchor=north west][inner sep=0.75pt]  [rotate=-310.79] [align=left] {$|z|$};
		\draw (56,84) node [anchor=north west][inner sep=0.75pt]   [align=left] {$\theta$};
		\draw (-13,0) node [anchor=north west][inner sep=0.75pt]   [align=left] {Im};
		\draw (118,122) node [anchor=north west][inner sep=0.75pt]   [align=left] {Re};
	\end{tikzpicture}
\end{minipage}
\subsubsection{Operaciones en Forma Polar}
\begin{definition}
	Sean $z_1=r_1(cos(\theta_1)+i \ sen(\theta_1))$ y $z_2=r_2(cos(\theta_2)+i \ sen(\theta_2))$, definimos su multiplicación como $z_1z_2=r_1r_2(cos(\theta_1+\theta_2)+i \ sen(\theta_1+\theta_2))$
\end{definition}
\begin{definition}
	Definimos la división de dos complejos como $\frac{z_1}{z_2}=\frac{r_1}{r_2}(cos(\theta_1-\theta_2)+i \ sen(\theta_1-\theta_2))$
\end{definition}
Observación: Si $z=r(cos(\theta)+ i \ sen(\theta))$ entonces $\frac{1}{z}=\frac{1}{r}(cos(\theta)+i \ sen(\theta))$
\begin{theorem}{Teorema de De Movire.}
	\\ Sea $z=r(cos(\theta)+i\ sen(\theta))$ y $n \in \mathbb{Z}^{+}$, entonces $z^n=(r(cos(\theta) + i \ sen(\theta)))^n=r^n(cos(n\theta)+i \ sen(n\theta))$
\end{theorem}
\subsection{Raíces de un numero complejo}
\subsubsection{Raíz cuadrada}
Si deseamos hallar $\sqrt{a+bi}$ una forma rápida de hacerlo es diciendo: $$\sqrt{a+bi}=c+di \rightarrow a+bi=(c+di)^2=c^2-d^2+2cdi\rightarrow \begin{cases}a=c^{2} -d^{2}\\b=2cd\end{cases}$$
\newpage \subsubsection{Raíces complejas}
Sea $z^n=r(cos(\theta)+i\ sen(\theta))$ y $n\in \mathbb{Z}^{+}$. Entonces, $z$ tiene $n$ raíces enésimas distintas.\\
Las raíces se hallan como: $$z_k=r^{\frac{1}{n}}\left(cos\left(\frac{\theta +2k\pi}{n}\right)+i\ sen\left(\frac{\theta + 2k\pi}{n}\right)\right)=r^{\frac{1}{n}}e^{\frac{\theta + 2k\pi}{n}i}$$
Donde $k=\{0,1,2,\dots,n-1\}$.
\newpage\section{Ecuaciones Diferenciales}
\begin{definition}
	Una ecuación diferencial es una igualdad en la cual la incógnita es una función desconocida y sus derivadas, $y=f(x)$ definida y derivable.
\end{definition}
Le llamamos ecuación de orden 1 si la única derivada de la función desconocida que aparece es la derivada primera, de orden 2 si la derivada de mayor orden que aparece es 2 y así sucesivamente.
\subsection{Ecuación diferencial de variables separables}
\begin{definition}
	Una ecuación diferencial se le llama de variables separables si es de la forma $y'=A(y)B(x)$
\end{definition}
Solución: $$\frac{y'(x)}{A(y(x))}=B(x)\rightarrow \int \frac{y'(x)}{A(y(x))} dx=\int B(x)dx+C\rightarrow \int \frac{dy}{A(y)}=\int B(x)dx+C$$
\subsection{Solución de una ecuación diferencial con condiciones o datos iniciales}
Por lo general existen infinitas soluciones a una misma ecuación diferencial, sin embargo si se dan datos iniciales apropiados, por lo general existe una única función solución que cumple los datos.\\\\
En una ecuación diferencial de primer orden se le llama dato inicial a una condición del tipo: $y(x_0)=0$.\\
Donde $x_0$ e $y_0$ son valores reales dados.\\\\
En una ecuación diferencial de segundo orden se le llaman datos iniciales a dos condiciones del tipo: $y(x_0)=y_0$, $y'(x_1)=y_1$.\\
Donde $x$ e $y$ son valores reales dados.\\\\
Para determinar la solución que verifica los datos iniciales, primero tendremos que haber hallado todas las soluciones (esto no quiere decir haber despejado la función).\\
Al conjunto de todas las soluciones se le llama solución general, la cual depende usualmente de una constante arbitraria si la ecuación es de primer orden, o de 2 si es de segundo orden.
\subsection{Ecuación diferencial lineal de primer orden homogénea}
\begin{definition}
	Se llama ecuación diferencial lineal de primer orden homogénea a una ecuación del tipo: $y'+a(x)y=0$
\end{definition}
Solución general:
$$y'=-a(x)y\rightarrow \frac{y'}{y}=-a(x)\rightarrow\int\frac{dy}{y}=-\int a(x)dx\rightarrow log(y)=-\int a(x)dx+C$$$$y=e^{-\int a(x)dx+C}=Ke^{-\int a(x)dx}$$
Siendo $K$ la variable arbitraria.
\newpage \subsection{Ecuación diferencial lineal de primer orden no homogénea}
\begin{definition}
	Se llama ecuación diferencial lineal de primer orden no homogénea a una ecuación del tipo: $y'+a(x)y=r(x)$, siendo $r(x)\neq 0$, si no, seria homogénea.
\end{definition}
Solución:
\begin{enumerate}
	\item Hallar la solución general, $y_h(x)$ de la ecuación diferencial lineal homogénea correspondiente, es decir la misma ecuación diferencial, pero sustituyendo $r(x)$ por la función nula.
	\item Hallar una solución particular de la ecuación $y_p(x)$ diferencial dada usando el método de variación de constante.
	\item Sumar $y(x)=y_h(x)+y_p(x)$
\end{enumerate}
\subsubsection{Método de variación de constante}
Para hallar la solución particular $y_p(x)$ de la ecuación diferencial, probaremos  con una función de cierto tipo como se describe:
\begin{enumerate}
	\item Tomaremos $y_p(x)=y_h(x)$, con la diferencia de que la constante arbitraria, ahora sera una función desconocida a determinar.
	\item Sustituimos $y_p(x)$ en la ecuación no homogénea dada, haciendo que la verifique y despejando la función. De todas las posibles funciones que se despejen (en general son infinitas), habrá que elegir solo una.
	\item Una vez hallada la función, la sustituimos en la expresión $y_p(x)$ para obtener la solución particular buscada.
\end{enumerate}
\textbf{Ejemplo:}\\
Sea $y'-cos(x)y=cos(x)$\\
Hallamos $y'-cos(x)y=0 \rightarrow y_h=Ke^{sen(x)}$\\
Decimos por el paso 2, que $y_p(x)=y_h(x)$ pero con la constante arbitraria como función,\\ así que
$$\begin{array}{l}
		y_p(x)=K(x)e^{sen(x)}                                            \\
		(K(x)e^{sen(x)})'-cos(x)K(x)e^{sen(x)}=cos(x)                    \\
		K'(x)e^{sen(x)}+K(x)cos(x)e^{sen(x)}-cos(x)K(x)e^{sen(x)}=cos(x) \\
		K'(x)e^{sen(x)}=cos(x)\rightarrow K'(x)=cos(x)e^{-sen(x)}        \\
		K(x)=\int cos(x)e^{-sen(x)}dx+C=-e^{-sen(x)}+C
	\end{array}$$
Ahora, elegimos la solución con $C=0$ y reemplazamos la función en $y_p(x)$
$$y_p(x)=K(x)e^{sen(x)}=-e^{-sen(x)}e^{sen(x)}=-e^{sen(x)-sen(x)}=-e^0=-1$$
Por lo tanto, $y(x)=y_h(x)+y_p(x)=Ke^{sen(x)}-1$
\newpage \subsection{Ecuación diferencial lineal de segundo orden a coeficientes constantes y homogénea}
\begin{definition}
	Una ecuación diferencial de segundo orden se llama lineal a coeficientes constantes y homogénea si es de la forma $y''+ay'+by=0$, donde $a$ y $b$ son constantes dadas independientes.
\end{definition}
\begin{theorem}{Estructura vectorial de las soluciones de la ecuación lineal homogénea.}\\
	Todas las funciones solución de la ecuación diferencial de segundo orden homogénea forman un espacio vectorial de dimensión 2.
\end{theorem}
Soluciones exponenciales: Buscaremos soluciones $y(x)=e^{\lambda x}$, donde $\lambda$ es una constante real a determinar, que sean solución de la ecuación diferencial $y''+ay'+by=0$. Sustituyendo en la ecuación diferencial $y=e^{\lambda x}$, $y'=\lambda e^{\lambda x}$, $y''=\lambda ^{2}e^{\lambda x}$, se obtiene: $(\lambda ^2+a\lambda+b)e^{\lambda x}=0\Leftrightarrow \lambda ^2+a\lambda+b=0$. Siendo $\lambda$ raíz de la ecuación de segundo grado, llamada \underline{ecuación caracteristica}.
\subsubsection{Solución general de la ecuación lineal homogénea de segundo orden}
Una vez encontradas las raíces de la ecuación característica, hay 3 casos.
\begin{enumerate}[A)]
	\item La ecuación característica tiene dos raíces distintas. La solución general es: $y(x)=C_1e^{\lambda_1 x}+C_2e^{\lambda_2 x}$
	\item La ecuación característica tiene una raíz doble. La solución general es: $y(x)=e^{\lambda x}(C_1+C_2 x)$
	\item La ecuación característica tiene dos raíces complejas conjugadas de la forma $\alpha\pm i\beta$. La solución general es: $y(x)=e^{\alpha x}(C_1 cos(\beta x)+C_2 sen(\beta x))$
\end{enumerate}
\subsection{Ecuación diferencial lineal de segundo orden a coeficientes constantes no homogénea}
\begin{definition}
	Una ecuación diferencial de segundo orden se llama lineal a coeficientes constantes y no homogénea si es del tipo: $y''+ay'+by=r(x)$.
\end{definition}
Solución:
\begin{enumerate}
	\item Hallar la solución general de $y_h(x)$ de la ecuación lineal homogénea correspondiente.
	\item Hallar una solución particular $y_p(x)$ de la ecuación no homogénea dada usando el método de coeficientes indeterminados.
	\item Sumar $y(x)=y_h(x)+y_p(x)$
\end{enumerate}
\subsubsection{Método de coeficientes indeterminados}
\begin{enumerate}
	\item Si $r(x)=e^{kx}P(x)$, donde $P(x)$ es un polinomio de grado $n$, probar $y_(x)=e^{kx}Q(x)$, donde $Q(x)$ es un polinomio de grado $n$ con coeficientes a determinar sustituyendo $y_p(x)$ en la ecuación diferencial.
	\item Si $r(x)=e^{kx}P(x)cos(mx)$ o $r(x)=e^{kx}P(x)sen(mx)$, donde $P$ es un polinomio de grado $n$, probar $y_p(x)=e^{kx}Q(x)cos(mx)+e^{kx}R(x)sen(mx)$, donde $Q$ y $R$ son polinomio de grado $n$.
\end{enumerate}
Si algún termino de $y_p(x)$ es también termino de $y_h(x)$, hay que multiplicar $y_p$ por $x$, o $x^2$ si es termino 2 veces.
\begin{theorem}
	Sea una ecuación diferencial $y''+ay'+by=r(x)$ donde $r(x)$ es una función conocida que puede descomponerse como suma $r(x)=r_1(x)+r_2(x)$.
	Se consideran las ecuaciones diferenciales auxiliares:\\
	$y_{1p} \rightarrow y''+ay'+by=r_1(x)$\\
	$y_{2p} \rightarrow y''+ay'+by=r_2(x)$\\
	Siendo $y_p(x)=y_{1p}+y_{2p}$.\\Este teorema puede aplicarse tantas veces como $r(x)$ pueda descomponerse, habiendo una suma de tres o mas sumandos en vez de dos como se mostró.
\end{theorem}
\newpage \section{Sucesiones y Series}
\subsection{Sucesiones}
\begin{definition}
	Las sucesiones son funciones $a: \mathbb{N} \rightarrow \mathbb{R}$, donde a cada natural, se le asocia un real $a_n$.
\end{definition}
Ejemplos:
\begin{enumerate}
	\item Sucesión armónica: $a_n=\frac{1}{n}$
	\item $a_0=1$, $a_1=1$, $a_n=a_{n-1}+a_{n-2}$
	\item Sucesión CTE: $a_n=c$ $\forall n \in \mathbb{N}$
	\item Sucesión identidad: $a_n=n$ $\forall n \in \mathbb{N}$
\end{enumerate}
\subsubsection{Convergencia}
\begin{definition}{Limite de una sucesión.}
	\\Una sucesión $a_n$ tiene el limite $L$ y se escribe como $$\\lim_{n \to \infty} a_n = L \text{ o } a_n \to L \text{ cuando } n \to \infty$$
	Si $\lim _{n\rightarrow \infty } a_{n} =L\begin{cases}
			=L\ \text{ (finito)} & \text{En este caso converge }(\mathbb{C}) \\
			=\infty              & \text{En este caso diverge }(\mathbb{D})  \\
			                     & \text{En este caso oscila }(\mathbb{O})
		\end{cases}$\\\\
	Y para todo $\epsilon > 0$ hay un correspondiente entero $N$ tal que si $n>N$ entonces $|a_n-L|<\epsilon$
\end{definition}
Ejemplos
\begin{enumerate}
	\item $\lim_{n \to + \infty} \frac{1}{n}=0$.
	\item $a_n=C$ $\forall n \rightarrow \lim_{n \to + \infty} a_n=c$
	\item $a_n=n$ $\forall n \in \mathbb{N} \rightarrow \lim_{n \to + \infty} a_n=+\infty$
	\item $a_n=(-1)^n$ $\forall n \in \mathbb{N}$
\end{enumerate}
\begin{theorem}
	Si $\lim_{x \to \infty} f(x)=L$ y $f(n)=a_n$ cuando $n$ es un entero, entonces $\lim_{n \to \infty} a_n = L$.
\end{theorem}
Propiedades:
\begin{enumerate}
	\item Si $\lim_{n \to + \infty} a_n=L$, $\lim_{n \to + \infty} b_n=M$ $\Rightarrow (a_n+b_n)_{n \in \mathbb{N}}$ es convergente y $\lim_{n \to + \infty} a_n+b_n=\lim_{n \to + \infty}a_n+\lim_{n \to + \infty} b_n=L+M$.
	\item $(\lambda a_n)_{n \in \mathbb{N}}$ es convergente si $\lim_{n \to +\infty} (\lambda a_n)=\lambda L$.
	\item $(a_n b_n)_{n\in\mathbb{N}}$ es convergente si $\lim_{n \to +\infty} a_n b_n=LM$
	\item Si $b_n=0$ $\forall n\in\mathbb{N}$, $M \ne 0$, $(\frac{a_n}{b_n})_{n\in\mathbb{N}}$ es convergente y $\lim_{n \to + \infty} \frac{a_n}{b_n}=\frac{L}{m}$
\end{enumerate}
\begin{theorem}
	Si $\lim_{n \to \infty} |a_n|=0$, entonces $\lim_{n \to \infty} a_n=0$
\end{theorem}
\begin{theorem}
	Si $\lim_{n \to \infty} a_n = L$ y la función $f$ es continua en $L$, entonces $$\lim_{n \to \infty} f(a_n)=f(L)$$
\end{theorem}\newpage
\subsubsection{Monotonía}
\begin{definition}
	Una sucesión $a_n$ se le llama monótona creciente si $a_n\le a_{n+1}$ para toda $n\ge 1$.\\
	Y se le denomina monótona decreciente si $a_n\ge a_{n+1}$.
\end{definition}
\begin{corollary}
	Si $a_{n+1}>a_n$ $\forall n\in\mathbb{N}$ es monótona estrictamente creciente.\\
	Si $a_{n+1}<a_n$ $\forall n\in\mathbb{N}$ es monótona estrictamente decreciente.
\end{corollary}
\subsubsection{Acotación}
\begin{definition}
	Una sucesión $a_n$ esta acotada superiormente si existe un numero $M$ tal que $a_n \le M$ para toda $n\ge 1$.\\
	Esta acotada inferiormente si existe un numero $m$ tal que $m\le a_n$ para toda $n\ge 1$.\\
	Si esta acotada superior e inferiormente, entonces $a_n$ es una sucesión acotada.
\end{definition}
\begin{theorem}{Teorema de la sucesión monótona.}
	\\Toda sucesión monótona y acotada es convergente.
\end{theorem}
Obs: Si $(a_n)_{n\in\mathbb{R}}$ es monótona decreciente y acotada, entonces $\lim a_n=\text{Inf}\{a_1,a_2,\dots\}$.
\subsubsection{Sub-sucesión}
\begin{definition}
	Dada una sucesión $a_n$ y otra extricamente creciente $(x_n):\mathbb{N}\to\mathbb{N}$, llamaremos sub-sucesión de $a_n$ a la sucesión $a\circ x:\mathbb{N}\to\mathbb{R}$, y lo denotamos como $a_{x_n}$.
\end{definition}
\begin{theorem}
	Si $\lim a_n=L$, entonces toda sub-sucesión de $a_n$ converge a $L$.
\end{theorem}
\subsubsection{Punto de acumulación}
\begin{definition}
	Sea una sucesión $a_n$, y su sub-sucesión $a_{n_k}$, $h$ es un punto de aglomeración si $a_{n_k}\to h$.
\end{definition}
\begin{theorem}{Teorema de Bolzano Weirstrass.}
	\\Todo conjunto infinito y acotado tiene (al menos) un punto de acumulación.
\end{theorem}
\begin{corollary}
	El punto de acumulación no tiene por que pertenecer al conjunto.
\end{corollary}
\begin{theorem}
	Toda sub-sucesión $a_n$ acotada tiene una sub-sucesión convergente.
\end{theorem}
\begin{corollary}
	Sea $\{a_n\}$ una sucesión de términos positivos ($a_n>0$ $\forall n$).\\
	Si $\lim \ \frac{a_{n+1}}{a_{n}} =L\begin{cases}< 1 & \text{Entonces $a_n$ converge a $0$}\\ >1 & \text{Entonces $a_n$ diverge}\end{cases}$
\end{corollary}
Obs: Si $L=1$ no se puede decir nada.
\begin{theorem}
	Una función $f: \mathbb{R}\rightarrow\mathbb{R}$ es continua en $a\in\mathbb{R}$.\\
	$\Leftrightarrow$ para toda sucesión ${a_n}$ tal que $\lim a_n=a$ se tiene que $\lim f(a_n)=f(a)$.
	$$(\forall {a_n}, a_n\rightarrow a \Rightarrow f(a_n)=f(a))$$
\end{theorem}
\newpage
\subsection{Series}
\begin{definition}
	En general, si se trata de sumar los términos de de una sucesión infinita $\{a_n\}_{n=1}^{\infty}$, se obtiene una expresión de la forma $$a_1+a_2+a_3+\cdots+a_n+\cdots$$
	que se denomina serie y se denota con el símbolo $$\sum_{n=1}^{\infty} a_n \text{ o } \sum a_n$$
\end{definition}
\begin{definition}
	Dada una serie $\sum_{n=1}^{\infty} a_n$, sea $s_n$ la n-enésima suma parcial: $$s_n=\sum_{i=1}^{n} a_i$$
	$$\lim _{n \to \infty} s_{n} =\lim _{n \to \infty} (a_{1} ,a_{2} ,\dotsc ,a_{n} )=\lim _{n \to \infty} \left(\sum _{k=1}^{n} a_{k} \right)\begin{cases}=L \text{ (finito)} & \text{Decimos que la serie converge (}\mathbb{C}\text{) y}\sum _{n=1}^{\infty } a_{n} =L  \\=\infty  & \text{Decimos que la serie diverge (}\mathbb{D}\text{)}\\ & \text{La serie oscila (}\mathbb{O}\text{)}\end{cases}$$
\end{definition}
Obs: $\sum_{n=0}^{\infty} a_n = \sum_{n=0}^{k-1} a_n + \sum_{n=k}^{\infty} a_n$
\subsubsection{Serie geométrica}
La serie geométrica $$\sum_{n=0}^{\infty} aq^n = \sum_{n=1}^{\infty} aq^{n-1}$$
es convergente si $|q|<1$ y su suma es
$$\sum_{n=0}^{\infty} aq^n=\sum_{n=1}^{\infty} aq^{n-1}=\frac{a}{1-q}$$
Si $|q|\ge 1$, la serie geométrica es divergente.\\
En el caso $$\sum_{n=n_0}^{\infty} aq^n=\frac{aq^{n_0}}{1-q}$$
\begin{theorem}
	Si $\sum a_n$ converge, entonces $a_n \to 0$.\\\\
	Cuidado que es una condición necesaria pero no suficiente. Concluir la convergencia de la serie a partir de que $\lim a_n=0$ es un grave error.
\end{theorem}
\subsubsection{Serie armónica}
\begin{theorem}{Convergencia de serie-p.}
	\\Sea una serie $\sum _{n=1}^{\infty }\frac{1}{n^{p}} =\begin{cases}p >1 & \mathbb{C}\\p\le 1 & \mathbb{D}\end{cases}$
\end{theorem}
\newpage
\subsubsection{Serie telescópica}
Una serie telescópica es aquella serie cuyas sumas parciales poseen un numero fijo de términos tras su cancelación.\\
Es decir, sea $\sum a_n$ donde $a_n=b_{n+1}-b_n$ siendo $b_n$ otra sucesión.\\
Entonces $s_n=b_{n+1}-b_0$ y $\lim s_n=\lim b_{n+1}-b_0$.\\
Un ejemplo clásico es la serie telescópica de Mengoli, que se define por $\sum_{n=1}^{\infty} \frac{1}{n(n+1)}$, y puede calcularse según
$$\begin{aligned}\sum _{n=1}^{\infty }{\frac {1}{n(n+1)}}                                                                                                                                                                                                   & {}=\sum _{n=1}^{\infty }\left({\frac {1}{n}}-{\frac {1}{n+1}}\right)\\{} & {}
               =\lim _{N\to \infty }\sum _{n=1}^{N}\left({\frac {1}{n}}-{\frac {1}{n+1}}\right)\\{}                                                                                                                                 & {}
               =\lim _{N\to \infty }\left\lbrack {\left(1-{\frac {1}{2}}\right)+\left({\frac {1}{2}}-{\frac {1}{3}}\right)+\cdots +\left({\frac {1}{N}}-{\frac {1}{N+1}}\right)}\right\rbrack \\{}                                  & {}
               =\lim _{N\to \infty }\left\lbrack {1+\left(-{\frac {1}{2}}+{\frac {1}{2}}\right)+\left(-{\frac {1}{3}}+{\frac {1}{3}}\right)+\cdots +\left(-{\frac {1}{N}}+{\frac {1}{N}}\right)-{\frac {1}{N+1}}}\right\rbrack \\{} & {}
               =\lim _{N\to \infty }\left\lbrack {1-{\frac {1}{N+1}}}\right\rbrack =1.\end{aligned}$$
\subsubsection{Series de términos positivos.}
\begin{definition}
	Una serie $\sum a_n$ se dice de términos positivos, siempre que $a_n>0$ $\forall n \in \mathbb{N}$.
\end{definition}
Observar que en este caso, la sucesión de sumas parciales $s_n$ es monótona creciente, por lo tanto la serie $\sum a_n$ puede ser convergente o divergente, pero nunca oscilar.
\begin{theorem}{Criterio de comparación.}
	\\Sean $\sum a_n$ y $\sum b_n$ series de términos positivos, tales que $a_n\le b_n$ $\forall n>n_0$. Entonces:
	\begin{enumerate}
		\item Si $\sum b_n$ $\mathbb{C}$ entonces $\sum a_n$ $\mathbb{C}$.
		\item Si $\sum a_n$ $\mathbb{D}$ entonces $\sum b_n$ $\mathbb{D}$.
	\end{enumerate}
	En cualquier otro caso, no puedo afirmar nada.
\end{theorem}
\begin{theorem}{Criterio de equivalencia.}
	\\Sean $\sum a_n$ y $\sum b_n$ dos series de términos positivos.
	\begin{enumerate}
		\item Si $\lim \frac{a_n}{b_n}=L >0$ finito, entonces las dos series son de la misma clase.
		\item Si $\lim \frac{a_n}{b_n}=0$ y $\sum b_n$ $\mathbb{C}$, entonces $\sum a_n$ $\mathbb{C}$.
		\item Si $\lim \frac{a_n}{b_n}=\infty$ y $\sum b_n$ $\mathbb{D}$ entonces $\sum a_n$ $\mathbb{D}$.
	\end{enumerate}
	En cualquier otro caso, no puedo concluir.
\end{theorem}
\begin{theorem}{Criterio del cociente.}
	\\Sea $\sum a_n$ una serie de términos positivos, tal que existe $\lim_{n \to \infty} \frac{a_{n+1}}{a_n}=L$. Entonces:
	\begin{enumerate}
		\item Si $L<1\Rightarrow \sum a_n$ $\mathbb{C}$
		\item Si $L>1\Rightarrow \sum a_n$ $\mathbb{D}$
	\end{enumerate}
	En otro caso el criterio no decide.
\end{theorem}
\begin{theorem}{Criterio de Cauchy.}
	\\Sea $\sum a_n$ una serie de términos positivos, tal que existe $\lim_{n \to \infty} \sqrt[n]{a_n}=L$. Entonces:
	\begin{enumerate}
		\item Si $L<1\Rightarrow\sum a_n$ $\mathbb{C}$
		\item Si $L>1\Rightarrow\sum a_n$ $\mathbb{D}$
	\end{enumerate}
	En otro caso el criterio no decide.
\end{theorem}\newpage
\subsubsection{Series alternadas.}
\begin{definition}
	A una serie se le dice alternada si tiene sus términos alternativamente positivos y negativos. Su expresión general es de la forma $\sum_{n=1}^{\infty} \left(-1  \right)^n a_n$ y $a_n>0$.
\end{definition}
\begin{definition}
	Decimos que una serie $\sum a_n$ es absolutamente convergente si y solo si $\sum |a_n|$ es convergente.
\end{definition}
\begin{theorem}
	Toda serie absolutamente convergente es convergente.
\end{theorem}
\begin{theorem}{Convergencia dominada.}
	\\Sea $a_n$, $b_n$ y $c_n$ tal que $a_n<b_n<c_n$ $\forall n$.\\
	Si $\sum c_n$ $\mathbb{C}$ y $\sum a_n$ $\mathbb{C}$ entonces $\sum b_n$ $\mathbb{C}$. Ademas, vale que $\sum a_n \le \sum b_n \le \sum c_n$.
\end{theorem}
\begin{theorem}{Criterio de Leibnitz}
	\\Si $a_n$ es una sucesión estrictamente decreciente que tiende a cero, entonces la serie alternada $\sum \left( -1 \right)^n a_n$ es convergente.
\end{theorem}
\newpage\section{Integrales impropias}
\subsection{Integrales impropias de $1^{ra}$ especie}
\begin{definition}
	Sea $f(x)$ definida en el intervalo $[a,\infty)$.
	\\Para toda $t>a$ la función $f(x)$ es integrable en $[a,t]$, y si $\lim_{t \to \infty} \int_{a}^{t} f(x) dx$ existe, decimos que la integral impropia esta definida y $\int_{a}^{\infty} f(x) dx=\lim_{t \to \infty} \int_{a}^{t} f(x) dx$.
	\begin{enumerate}
		\item Si $\lim_{t \to \infty} \int_{a}^{t} f(x) dx=L$ decimos que converge.
		\item Si $\lim_{t \to \infty} \int_{a}^{t} f(x) dx=\pm\infty$ decimos que diverge.
		\item Si $\lim_{t \to \infty} \int_{a}^{t} f(x) dx\ \nexists$ decimos que oscila.
	\end{enumerate}
\end{definition}
\begin{example}
	Sea $t>1$ $$\int _{1}^{t}\frac{1}{x^{p}} dx=\begin{cases}
			\left[ -\frac{1}{( 1-p) x^{p-1}}\right]_{1}^{t} & \text{Si} \ p\neq 1 \\
			[ log( x)]_{1}^{t}                              & \text{Si} \ p=1
		\end{cases} =\begin{cases}
			\frac{1}{p-1}-\frac{t^{1-p}}{(p-1)} & \text{Si} \ s\neq 1 \\
			log( t)                             & \text{Si} \ s=1
		\end{cases}$$
	Como $\lim_{t \to \infty} log(t)=\infty$ diverge.
	\\Por otro lado tenemos que $1-p<0$ o que $p<1$ por lo tanto $\lim_{t \to \infty} t^{1-p}=\infty$, y si $1-p<0$ o $p>1$ nos da que $\lim_{t \to \infty}t^{1-p}=0$.
	\\En conclusión, $\int_{1}^{\infty} \frac{1}{x^p} dx$ diverge si $p\le 1$ y converge si $p>1$.
\end{example}
\begin{definition}
	De manera análoga también podemos definir integrales impropias de la forma:
	\begin{enumerate}
		\item $\int_{-\infty}^{b} f(x) dx=\lim_{t \to -\infty}\int_{t}^{b} f(x) dx$
		\item $\int_{-\infty}^{\infty} f(x) dx=\int_{-\infty}^{a} f(x) dx + \int_{a}^{\infty} f(x) dx$
	\end{enumerate}
\end{definition}
\begin{corollary}
    Sea $f$ tal que $\int_{a}^{\infty} f(x) dx$ converge, y existe $\lim_{x \to \infty} f(x)=L$, entonces $L=0$. Aunque no podemos deducir que converge si su limite es $0$.
\end{corollary}
\begin{corollary}{Álgebra de integrales impropias de primera especie.}
	\\Sean $\int_{a}^{\infty} f(x) dx$ y $\int_{a}^{\infty} g(x) dx$ dos integrales impropias convergentes, y sea $\lambda\in\mathbb{R}$, se cumple que:
	\begin{enumerate}
		\item La integral $\int_{a}^{\infty} f(x)+ g(x)\ dx$ converge, y ademas $$\int_{a}^{\infty} f(x)+g(x)\ dx=\int_{a}^{\infty} f(x) dx+\int_{a}^{\infty} g(x) dx$$
		\item La integral $\int_{a}^{\infty} \lambda f(x) dx$ converge y ademas $$\int_{a}^{\infty} \lambda f(x) dx=\lambda \int_{a}^{\infty} f(x) dx$$
	\end{enumerate}
\end{corollary}
\begin{corollary}
	Aplicando la proposición anterior, si $\int_{a}^{\infty} f(x) dx$ converge y $\int_{a}^{\infty} g(x) dx$ diverge, entonces $\int_{a}^{\infty} f(x)+g(x)\ dx$ diverge.
\end{corollary}
\begin{corollary}
	Si $\int_{a}^{\infty} f(x) dx$ y $\int_{a}^{\infty} g(x) dx$ divergen, no podemos decir nada de su suma.
\end{corollary}
\begin{theorem}
	Sea $f(x)$ definida en $[a,\infty)$ una función no negativa e integrable en $[a,b]$ para $b\ge a$. Consideremos la función $F(b)=\int_{a}^{b} f(x) dx$ para $b\in [a,\infty)$. Entonces las siguientes afirmaciones se cumplen:
	\begin{enumerate}
		\item La función $F$ en $[a,\infty)$ es monótona creciente.
		\item $\int_{a}^{\infty} f(x) dx$ converge si y solo si, $F$ esta acotada superiormente.
	\end{enumerate}
\end{theorem}
\newpage\begin{theorem}{Teorema de comparación}
	\\Sea $0\le f(x)\le g(x)$ en $[a,\infty)$
	\begin{enumerate}
		\item Si $\int_{a}^{\infty} g(x) dx$ converge, entonces $\int_{a}^{\infty} f(x) dx$ también.
		\item Si $\int_{a}^{\infty} f(x) dx$ diverge, entonces $\int_{a}^{\infty} g(x) dx$ también.
	\end{enumerate}
\end{theorem}
\begin{theorem}{Criterio de equivalencia}
	\\Sean $f,g$ funciones definidas en $[a,\infty)$ que cumplen:
	\begin{itemize}
		\item $f(x)\ge 0$ y $g(x)>0\ \forall x\ge a$. Es decir, $f$ es no negativa y $g$ positiva en $[a,\infty)$.
		\item $f$ y $g$ son integrables en $[a,b]$ para $b\ge a$.
	\end{itemize}
	Consideremos $\lim_{x \to \infty} \frac{f(x)}{g(x)}=L$.
	\\Entonces cumplen las afirmaciones:
	\begin{enumerate}
		\item Si $L>0$, se tiene que $\int_{a}^{\infty} f(x) dx$  y $\int_{a}^{\infty} g(x) dx$ son del mismo tipo, es decir ambos convergen o divergen.
		      \\Si $L>0$, decimos que las funciones $f$ y $g$ son equivalentes.
		\item Para el caso $L=0$:
		      \begin{itemize}
			      \item $\int_{a}^{\infty} g(x) dx$ converge $\Rightarrow \int_{a}^\infty f(x) dx$ converge.
			      \item $\int_{a}^{\infty} f(x) dx$ diverge $\Rightarrow \int_{a}^{\infty} g(x) dx$ diverge.
		      \end{itemize}
		\item Para el caso $L=\infty$:
		      \begin{itemize}
			      \item $\int_{a}^{\infty} g(x) dx$ diverge $\Rightarrow \int_{a}^{\infty} f(x) dx$ diverge.
			      \item $\int_{a}^{\infty} f(x) dx$ converge $\Rightarrow \int_{a}^{\infty} g(x) dx$ converge.
		      \end{itemize}
	\end{enumerate}
\end{theorem}
\begin{definition}
	Sea $f(x)$ definida en $[a,\infty)$ una función integrable en $[a,b]$ para $b\ge a$. Decimos que $\int_{a}^{\infty} f(x) dx$ es absolutamente convergente si $\int_{a}^{\infty} |f(x)| dx$ converge.
\end{definition}
\begin{theorem}
	Sea $f(x)$ definida en $[a,\infty)$ una función integrable en $[a,b]$ para $b\ge a$ tal que $\int_{a}^{\infty} f(x) dx$ es absolutamente convergente. Entonces $\int_{a}^{\infty} f(x) dx$ converge.
\end{theorem}
\begin{definition}
	Sea $f(x)$ definida en $[a,\infty)$ una función integrable en $[a,b]$ para $b\ge a$, diremos que $\int_{a}^{\infty} f(x) dx$ es condicionalmente convergente si $\int_{a}^{\infty} f(x) dx$ converge pero $\int_{a}^{\infty} |f(x)| dx$ diverge.
\end{definition}
\begin{theorem}{Criterio serie-integral.}
	Sea $f(x)$ definido en $[a,\infty)$ una función positiva, decreciente e integrable en $[a,b]$ para $b\ge a$. Sea $n_0$ el primer entero no negativo mayor o igual que $a$. Para cada $n\ge n_0$, sean $$s_n=\sum_{k=n_0}^{n} f(k) \text{ y } t_n=\int_{n_0}^{n} f(x) dx$$
	Entonces, ambas sucesiones son del mismo tipo.
\end{theorem}
\newpage\subsection{Integrales impropias de $2^{da}$ especie}
\begin{definition}
	Sea $f(x)$ una función definida en $[t,b]$ para $a<t<b$, pero no esta definida en $[a,b]$ (o definida en $(a,b]$), entonces $\int_{a}^{b} f(x) dx$ no esta definida.
	\\Puede pasar que $\int_{t}^{b} f(x) dx$ este definida y existe $\lim_{t \to a^+} \int_{t}^{b} f(x) dx$, entonces  decimos que su valor es $$\int_{a^+}^{b} f(x) dx=\lim_{t \to a^+} \int_{t}^{b} f(x) dx$$
	\\De misma manera, puede pasar que $f(x)$ es continua en el intervalo $[a,t]$ para $a<t<b$ y existe $\lim_{t \to b^-}\int_{a}^{t} f(x) dx$, entonces $$\int_{a}^{b^-} f(x) dx=\lim_{t \to b^-} \int_{a}^{t} f(x) dx$$
\end{definition}
\begin{example}
	Veamos nuevamente la función $\frac{1}{x^p}$ con $p\in\mathbb{R}$, veamos para cuales valores de $p$, la integral impropia $\int_{0}^{1} \frac{1}{x^p} dx$ converge.
	$$\int _{0}^{1}\frac{1}{x^{p}} dx=\lim _{t\rightarrow 0}\int _{t}^{1}\frac{1}{x^{p}} dx=\begin{cases}
			[ log( t)]_{t}^{1}                       & \text{Si} \ p=1     \\
			\left[\frac{x^{1-p}}{1-p}\right]_{t}^{1} & \text{Si} \ p\neq 1
		\end{cases} =\begin{cases}
			-log( t)              & \text{Si} \ p=1     \\
			\frac{1-t^{1-p}}{1-p} & \text{Si} \ p\neq 1
		\end{cases} =\begin{cases}
			\infty        & \text{Si} \ p=1     \\
			\frac{1}{1-p} & \text{Si} \ 0< p< 1 \\
			\infty        & \text{Si} \  p>1
		\end{cases}$$
	Por lo tanto, $\int_{0}^{1} \frac{1}{x^p} dx$ converge si $p<1$, y diverge si $p\ge 1$.
\end{example}
\begin{corollary}
	Sea $f(x)$ una función definida en el intervalo $(a,b)$, e integrable en $[c,d]$ para $a<c\le d<b$. Si $\int_{a^+}^{b^-} f(x) dx$ converge, entonces $$\int_{a^+}^{b^-} f(x) dx=\int_{a^+}^{d} f(x) dx + \int_{d}^{b^-} f(x) dx$$
	para $\forall d\in (a,b)$.
\end{corollary}
\begin{corollary}{Álgebra de integrales impropias de segunda especie}
	Sean $\int_{a^+}^{b} f(x) dx$ y $\int_{a^+}^{b} g(x) dx$ dos integrales impropias convergentes y $\lambda\in\mathbb{R}$. Entonces:
	\begin{enumerate}
		\item La integral $\int_{a^+}^{b} f(x)+g(x)\ dx$ converge y ademas $$\int_{a^+}^{b} f(x)+g(x)\ dx=\int_{a^+}^{b} f(x) dx+\int_{a^+}^{b} g(x) dx$$
		\item La integral $\int_{a^+}^{b} \lambda f(x) dx$ converge y ademas $$\int_{a^+}^{b} \lambda f(x) dx=\lambda \int_{a^+}^{b} f(x) dx$$
	\end{enumerate}
\end{corollary}
\begin{theorem}
	Sea $f(x)$ definida en $(a,b]$ una función no negativa e integrable en $[c,b]$, para $c\in (a,b]$. Consideremos la función $F(c)=\int_{c}^{b} f(x) dx$. Entonces cumple:
	\begin{enumerate}
		\item La función $F$ en $(a,b]$ es monótona decreciente.
		\item $\int_{a^+}^{b} f(x) dx$ converge si y solo si $F$ esta acotada superiormente.
	\end{enumerate}
\end{theorem}
\newpage\begin{theorem}{Criterio de comparación}
    \\Sean $f,g$ definidas en $(a,b]$ funciones que cumplen:
    \begin{itemize}
        \item $0\le f(x)\le g(x)\ \forall x\in (a,b]$
        \item $f$ y $g$ son integrables en $[c,b]$ para $c\in (a,b]$.
    \end{itemize}
    Entonces cumplen:
    \begin{enumerate}
        \item Si $\int_{a^+}^{b} g(x) dx$ converge, entonces $\int_{a^+}^{b} g(x) dx$ converge.
        \item Si $\int_{a^+}^{b} f(x) dx$ diverge, entonces $\int_{a^+}^{b} g(x) dx$ diverge.
    \end{enumerate}
\end{theorem}
\begin{theorem}{Criterio de equivalentes}
    \\Sean $f,g$ funciones definidas en $(a,b]$ que cumplen:
    \begin{itemize}
        \item $f(x)\ge 0$ y $g(x)>0\ \forall x\in (a,b]$.
        \item $f$ y $g$ son integrables en $[c,b]\ \forall c\in (a,b]$.
    \end{itemize}
    Consideremos el limite $$\lim_{x \to a^+} \frac{f(x)}{g(x)}=L$$
    Entonces se cumple lo siguiente:
    \begin{enumerate}
        \item Si $L>0$ se tiene que $\int_{a^+}^{b} f(x) dx$ y $\int_{a^+}^{b} g(x) dx$ son de la misma clase.
        \item Si $L=0$
            \begin{itemize}
                \item $\int_{a^+}^{b} g(x) dx$ converge $\Rightarrow \int_{a^+}^{b} f(x) dx$ converge.
                \item $\int_{a^+}^{b} f(x) dx$ diverge $\Rightarrow \int_{a^+}^{b} g(x) dx$ diverge.
            \end{itemize}
        \item Si $L=\infty$
            \begin{itemize}
                \item $\int_{a^+}^{b} g(x) dx$ diverge $\Rightarrow \int_{a^+}^{b} f(x) dx$ diverge.
                \item $\int_{a^+}^{b} f(x) dx$ converge $\Rightarrow \int_{a^+}^{b} g(x) dx$ converge.
            \end{itemize}
    \end{enumerate}
\end{theorem}
\subsection{Integrales mixtas}
Cuando en una integral aparece mas de un punto problemático, debemos partir la integral en una suma de integrales que contengan solamente uno de esos puntos, y decimos que la integral original es convergente si y solo si cada uno de los sumandos lo es.\\
De esta forma por ejemplo si $f$ es continua, la integral impropia $\int_{-\infty}^{+\infty} f(x) dx$ debemos escribirla como suma de $\int_{-\infty}^{a} f(x) dx$ y $\int_{a}^{+\infty} f(x) dx$, y debemos clasificar estas dos integrales. Es fácil ver que el resultado no depende de $a$.
\begin{definition}
    Sea $f(x)$ definida en $(a,\infty)$ una función integrable en todo subintervalo $[b,c]\subseteq (a,\infty)$.
    \\A la expresión $\int_{a^+}^{\infty} f(x) dx$ se le conoce como integral impropia mixta.
\end{definition}
\newpage\section{Topología y Sucesiones en $\mathbb{R}^n$}
\subsection{Topología}
\begin{definition}{Norma}
	\\Una norma en $\mathbb{R}^n$ es una función $||\ ||:\mathbb{R}^n\to\mathbb{R}$ que cumple:
	\begin{itemize}
		\item $||x||\ge 0\ \forall x\in\mathbb{R}^n$
		\item $||x||=0\Leftrightarrow x=\vec{0}$
		\item $||\lambda x||=|\lambda|||x||\ \forall x\in\mathbb{R}^n\ \forall\lambda\in\mathbb{R}$
		\item $||x+y||\le ||x||+||y||\ \forall x,y\in\mathbb{R}^n$
	\end{itemize}
\end{definition}
\begin{definition}
	Una distancia en $\mathbb{R}^n$ es una función $d:\mathbb{R}^n\times\mathbb{R}^n\to\mathbb{R}$ tal que:
	\begin{itemize}
		\item $d(x,y)\ge 0\ \forall x,y\in\mathbb{R}^n$
		\item $d(x,y)=0\Leftrightarrow x=y$
		\item $d(x,y)=d(y,x)\ \forall x,y\mathbb{R}^n$
		\item $d(x,y)\le d(x,z)+d(z,y)\ \forall x,y,z\in\mathbb{R}^n$
	\end{itemize}
\end{definition}
\begin{definition}
	Sea $||\ ||$ una norma en $\mathbb{R}^n$. La distancia inducida por $||\ ||$ se define como: $$d(x,y)=||x-y||$$
\end{definition}
Normas en $\mathbb{R}^2$
\begin{itemize}
	\item $||x||_2=||(x,y)||_2=\sqrt{x^2+y^2}$ (norma euclidiana)
	\item $||x||_1=||(x,y)||=|x|+|y|$ (norma 1 o del taxista)
	\item $||x||_{\infty}=||(x,y)||_{\infty}=max\{|x|,|y|\}$ (norma infinito)
\end{itemize}
\begin{definition}{ \ }
	\\Sea $d$ una distancia en $\mathbb{R}^n$. Sea $a\in\mathbb{R}^n$ y $\delta >0$.
	\\La bola abierta de centro $a$ y radio $\delta$ es el conjunto $$B(a,\delta)=\{x\in\mathbb{R}^n :\ d(a,x)<\delta \}$$
	La bola cerrada de centro $a$ y radio $\delta$ es el conjunto $$B[a,\delta]=\{x\in\mathbb{R}^n :\ d(a,x)\le\delta \}$$
	La esfera de centro $a$ y radio $\delta$ es el conjunto $$E[a,\delta]=\{x\in\mathbb{R}^n:\ d(a,x)=\delta \}$$
\end{definition}
\begin{corollary}
	$B(a,\delta) \cup E[a,\delta]=B[a,\delta]$.
\end{corollary}
\begin{definition}
    Llamamos bola reducida de centro $a$ y radio $\delta$ al conjunto $B^*(a,\delta)=B(a,\delta)-\{a\}$, es decir, la bola sin el centro.
\end{definition}
\begin{definition}
    Sea $A\subset\mathbb{R}^n$. Definimos al complemento de $A$ como $A^C$ a el conjunto que contiene todos los elementos que no están en $A$.
\end{definition}
\newpage\begin{definition}
    Sea $A\subset\mathbb{R}^n$ un conjunto, $A^C$ su complemento, entonces:
    \begin{enumerate}
        \item Decimos que $x_0\in\mathbb{R}^n$ es un punto interior del conjunto A, si existe una bola de centro $x_0$ y radio $r>0$ tal que $B(x_0,r)\subset A$.\\
            Llamamos ${\r A}$ al conjunto de puntos interiores de $A$.
        \item Decimos que $x_0\in\mathbb{R}^n$ es un punto exterior del conjunto $A$ si existe una bola de centro $x_0$ y radio $r>0$ tal que $B(x_0,r)\subset A^c$.
            \\Llamamos $Ext(A)$ al conjunto de puntos exteriores de $A$.
        \item Decimos que $x_0 \in\mathbb{R}^n$ es un \underline{punto frontera} de $A$ si para todo $r>0$, $B(x_0,r)\cap A\ne \emptyset$ y $B(x_o,r)\cap A^c \ne \emptyset$.
            \\Llamamos $\partial A$ al conjunto de puntos frontera de $A$.
    \end{enumerate}
\end{definition}
\begin{corollary}
    La bola abierta $B(a,\delta)$ es un conjunto abierto.
\end{corollary}
\begin{corollary}
    Si $A$ y $B$ son dos conjuntos abiertos, entonces $A\cap B$ es abierto.
\end{corollary}
\begin{theorem}
    Sea $A$ un conjunto abierto, entonces su unión con cualquier otro conjunto es un conjunto abierto.
\end{theorem}
\begin{corollary}
    Si $A$ es un conjunto de $\mathbb{R}^n$, entonces:
    \\$int(A)\cap Ext(A)\cup Fr(A)=\mathbb{R}^n$
    \\$$ \begin{cases}
        int(A)\cap Ext(A)=\emptyset
        \\ int(A)\cap Fr(A)=\emptyset
        \\ Ext(A)\cap Fr(A)=\emptyset
    \end{cases}$$
\end{corollary}
\begin{definition}
    Sea $A\subset\mathbb{R}^n$ un conjunto:
    \begin{enumerate}
        \item Decimos que $A$ es un \underline{conjunto abierto} si todos sus puntos son interiores, es decir $A={\r A}$.
        \item Decimos que $A$ es un conjunto cerrado si $A^c$ es un conjunto abierto.
        \item La clausura de $A$ es el conjunto $$\overline{A}=A\cup\partial(A)$$
        \item $A$ es un \underline{conjunto acotado} si existe un $K>0$ tal que $A\subset B(\vec{0},K)$, es decir, si podemos incluir al conjunto en una bola.
        \item $A$ es un \underline{conjunto compacto} si $A$ es cerrado y acotado.
        \item $x_0\in\mathbb{R}^n$ es un \underline{punto de acumulación} de $A$ si para todo $r>0$, $B^*(x_0,r)\cap A\ne\emptyset$
            \\Al conjunto de los puntos de acumulación lo \underline{llamamos derivado de $A$}, y se nota $A'$.
        \item Un conjunto $A$ es cerrado si y solo si $A$ contiene todos sus puntos de acumulación.
    \end{enumerate}
\end{definition}
\newpage\subsection{Sucesiones en $\mathbb{R}^n$}
\begin{definition}
    Una sucesión es una función $a:\mathbb{N}\to\mathbb{R}^n$
\end{definition}
\begin{definition}{Límite}
    \\Decimos que la sucesión $a_n$ tiene límite $L\in\mathbb{R}^n$, y lo denotamos $\lim_{n \to \infty}a_n=L\ \Leftrightarrow\ \forall\epsilon>0,\ \exists n_0\in\mathbb{N}$ tal que $\forall n\ge 0$ se tiene que $a_n\in B(L,\epsilon)$.
\end{definition}
\begin{corollary}
    Sea $a_k$ una sucesión en $\mathbb{R}^n$ y $L=(L_1,L_2,\dots,L_n)$. Entonces: $$\lim_{k \to \infty}a_k=L\ \Leftrightarrow\ \lim_{k \to \infty}a_{k_i}=L_i\ \forall i=1,2,\dots,n$$
    Es decir, una sucesión en $\mathbb{R}^n$ converge a $L$ si y sólo si cada una de sus coordenadas converge a la coordenada correspondiente de $L$
\end{corollary}
\begin{corollary}{Propiedades}
        Si $x_n\to p_1$ e $y_n\to p_2$ entonces
        \begin{enumerate}
            \item $x_n+y_n=p_1+p_2$
            \item $\lambda x_n\to\lambda p_1\ \forall\lambda\in\mathbb{R}$
            \item $||x_n||\to ||p_1||$
        \end{enumerate}
        No tiene sentido decir $x_n\times y_n\to p_1\times p_2$.
\end{corollary}
\begin{definition}
    Sea $A\subset\mathbb{R}^m$ un conjunto. Decimos que $A$ es cerrado por sucesiones $\Leftrightarrow\forall\{x_n\}\subset A$ sucesión tal que $x_n\to p$ se tiene que $p\in A$.
\end{definition}
\begin{corollary}
    $A$ es cerrado $\Leftrightarrow A$ es cerrado por sucesiones.
\end{corollary}
\begin{corollary}
    Toda sucesión acotada en $\mathbb{R}^m$ tiene al menos una sub-sucesión convergente. 
\end{corollary}
\begin{theorem}
    Sea $A\subset\mathbb{R}^m$ un conjunto compacto (cerrado + acotado) y sea $\{x_n\}$ una sucesión en $A$. Entonces $\{x_n\}$ tiene una sub-sucesión convergente a $p\in A$.
\end{theorem}
\begin{definition}
    Sea $a_k$ una sucesión en $\mathbb{R}^n$, y $k_j$ una sucesión estrictamente creciente de números naturales. Entonces la sucesión $a_{k_j}$ es una sub-sucesión de $a_k$.
\end{definition}
\begin{theorem}
    Toda sucesión acotada de $\mathbb{R}^n$ tiene una sub-sucesión convergente.
\end{theorem}
\begin{corollary}
    Si $K\subset\mathbb{R}^n$ es un conjunto compacto, y $a_k$ es una sucesión de elementos de $K$, entonces $a_k$ posee una sub-sucesión convergente, y ademas su limite es un elemento de $K$.
\end{corollary}\newpage
\section{Continuidad}
\subsection{Funciones en $\mathbb{R}^n$}
\begin{definition}{Función escalar}
	\\Una función escalar es una función cuyo dominio esta incluido en $\mathbb{R}^n$ y codominio $\mathbb{R}$.
\end{definition}
\begin{example}{ \ }
    \begin{itemize}
        \item $f:\mathbb{R}^2\to\mathbb{R}\ /\ f(x,y)=xy$
        \item $T:\mathbb{R}^n\to\mathbb{R}\ /\ T(x_1,x_2,\cdots,x_n)=x_1+x_2+\cdots+x_n$
        \item $f:\mathbb{R}^2\to\mathbb{R}\ /\ f(x,y)=x^2+xy$
    \end{itemize}
\end{example}
\begin{definition}{Función vectorial}
	\\Una función vectorial es una función con dominio en $\mathbb{R}^n$ y codominio en $\mathbb{R}^m$.\\
    \\Para $f:A\subseteq\mathbb{R}^n\to\mathbb{R}^m\ /\ f(x_1,x_2,\cdots,x_n)=(f_1(x_1,x_2,\cdots,x_n),\cdots,f_m(x_1,\cdots,x_n))$ y se nota: $$f=(f_1,f_2,\cdots,f_m)$$
Cada $f_i:A\subseteq\mathbb{R}^n\to\mathbb{R}$ es una función escalar $\forall i=1,\cdots,m$ y se llama en este contexto función coordenada.
\\Sea $f:A\subset\mathbb{R}^n\to\mathbb{R}$, el grafico de $f$ es el conjunto $$G(f)=\{(x_1,x_2,\cdots,x_n,x_n+1)\in\mathbb{R}^n:\ (x_1,x_2,\cdots,x_n)\in A,\ x_{n+1}=f(x_1,x_2,\cdots,x_n)\}$$
\end{definition}
\begin{definition}{Conjuntos de nivel}
    \\Los conjuntos de nivel son subconjuntos del dominio, cuyos puntos tienen la misma imagen por función.
    \\Dado un $k\in\mathbb{R}$, el conjunto de nivel $k$ es $C_k=\{x\in\mathbb{R}^n : f(x)=k\}$.
\end{definition}
\newpage\subsection{Limites y continuidad}
\begin{definition}
    Dado un conjunto $D\subset\mathbb{R}^n$, una función $f:D\to\mathbb{R}$, y $a\in\mathbb{R}^n$ un punto de acumulación de $D$, decimos que $$\lim_{x \to a}f(x)=L\ \Leftrightarrow\ \forall\epsilon>0,\ \exists\delta>0\ \text{tal que }\forall x\in B^*(a,\delta)\cap D\ \text{se cumple }f(x)\in B(L,\epsilon)$$
\end{definition}
\begin{definition}
    Dado un conjunto $D\subset\mathbb{R}^n$, una función $f:D\in\mathbb{R}$, y $a\in\mathbb{R}^n$, y un punto de $D$, decimos que $f$ es continua en $a$ si y solo si: $$\forall\epsilon>0,\ \exists\delta>0\ \text{tal que }\forall x\in B(a,\delta)\cap D\ \text{se cumple }f(x)\in B(f(a),\epsilon)$$
\end{definition}
\begin{corollary}
    El punto $a$ debe estar en el dominio (en particular calculamos $f(a)$), pero no necesariamente debe ser un punto de acumulación, de esta forma podemos distinguir dos casos:
    \begin{enumerate}
        \item Si $a$ es un punto de acumulación de $D$, entonces la definición de continuidad coincide con la de limite, con $L=f(a)$. Es decir $$f\text{ es continua en }a\Leftrightarrow \lim_{x \to a}f(x)=f(a)$$
        \item Si $a$ no es un punto de acumulación de $D$, entonces es un punto aislado. Es decir, existe un radio $\delta>0$ tal que no hay puntos de $D$ en $B(a,\delta)$. Entonces, una función $f$ es siempre sera continua en los puntos aislados del dominio.
    \end{enumerate}
\end{corollary}
\begin{theorem}
    Sea un conjunto $D\subset\mathbb{R}^n$, una función $f:D\in\mathbb{R}$, y $a\in\mathbb{R}^n$ un punto de acumulación de $D$, entonces: $$ \begin{array}{l}
\begin{array}{ c l }
\lim _{x\rightarrow a} f( x) =L\Leftrightarrow  & \text{para toda sucesión } x_{k} \ \text{de elementos de } D\backslash \{a\} \ \text{tal que }\lim _{k\rightarrow \infty } x_{k} =a\\
 & \text{tenemos que }\lim _{k\rightarrow \infty } f( x_{k}) =L
\end{array}\\
\end{array}$$
\end{theorem}
\begin{theorem}
    Sea un conjunto de $D\subset\mathbb{R}^n$, una función $f:D\to\mathbb{R}$, y $a\in\mathbb{R}^n$ un punto de $D$, entonces $$\begin{array}{ c l }
f\ \text{es continua en } a\Leftrightarrow  & \text{para toda sucesión } x_{k} \ \text{de elementos de } D\ \text{tal que }\lim _{k\rightarrow \infty } x_{k} =a\\
 & \text{tenemos que }\lim _{k\rightarrow \infty } f( x_{k}) =f( a)
\end{array}$$
\end{theorem}
\begin{definition}{Limites reiterados}
    \\Sea $f:\mathbb{R}^n\to\mathbb{R}$, dado un punto $(a_1,a_2,\cdots,a_n)\in\mathbb{R}^n$, decimos que los limites reiterados de $f$ como:$$\lim_{x_1 \to a_1}\left( \lim_{x_2 \to a_2}\left(\cdots\left( \lim_{x_n \to a_n} f(x_1,x_2,\cdots,x_n)\right)\cdots \right) \right) $$
\end{definition}
\begin{corollary}
    Si $\lim_{(x_1,\cdots,x_n) \to (a_1,\cdots,a_n)}f(x_1,\cdots,x_n)=L$ entonces todos los limites reiterados de la función $f$ en el punto $(a_1,a_2,\cdots,a_n)$ existen y valen $L$.
\end{corollary}
\begin{definition}{Limites direccionales}
    \\Sea $f:\mathbb{R}^n\to\mathbb{R}$ y sea $g:\mathbb{R}^{n-1}\to\mathbb{R}$, definimos el limite direccional de $f$ en la dirección de $g$ en el punto $(a_1,a_2,\cdots,a_n)\in\mathbb{R}^n$ como:$$\begin{array}{ l }
\lim _{( x_{1} ,x_{2} ,\cdots ,x_{n})\rightarrow ( a_{1} ,a_{2} ,\cdots ,a_{n})} f( x_{1} ,x_{2} ,\cdots x_{n})\\
x_{j} =g\underbrace{( x_{1} ,x_{2} ,\cdots ,x_{n})}_{\text{No aparece } x_{j}}
\end{array}$$
\end{definition}
\begin{example}
    Una generalización en $\mathbb{R}^2$ seria: $$\begin{array}{ c }
\lim _{( x,y)\rightarrow ( x_{0} ,y_{0})} f( x,y) =\lim _{x\rightarrow x_{0}} f( x,g( x))\\
g( x) =y
\end{array}$$
\end{example}
\begin{corollary}
    Si existe el limite de $f$ en $(a_1,\cdots,a_n)$ y vale $L$, entonces existe el limite direccional a través de cualquier dirección $g$ y también vale $L$.
\end{corollary}
\begin{corollary}
    Es importante saber que los limites direccionales no permiten garantizar la existencia del limite. Son útiles para hallar un candidato a limite, o para demostrar que el limite no existe.
\end{corollary}
\begin{definition}{Limites por coordenadas polares}
    \\El calculo de limites con coordenadas polares consiste en aplicar a las variables $x$ e $y$ las ecuaciones de equivalencia entre la representación mediante coordenadas cartesianas y polares, según la grafica, con $x=r\cos(\theta)$ e $y=r\sen(\theta)$, y definimos el acercamiento al punto $(0,0)$ mediante la resolución del limite en la variable $r$:$$\lim_{(x,y) \to x_0,y_)}f(x,y)=\lim_{r \to 0}f(r,\theta)$$
    Al dejar la variable angular $\theta$ sin tratamiento en el calculo, se contemplan todas las trayectorias posibles de acercamiento.
\end{definition}
\begin{corollary}
    Si el limite en coordenadas polares queda dependiendo de $\theta$, es decir de la dirección, podemos concluir que no existe el limite, y en caso que $\theta$ no quede acotada, no podemos concluir nada.
\end{corollary}
\begin{example}
    Sea $f(x,y)=\frac{x^2y}{x^2+y^2}$ calcular su limite en $(0,0)$.
    $$\lim_{(x,y) \to (0,0)}\frac{x^2y}{x^2+y^2}=\lim_{r \to 0}\frac{(r\cos(\theta))^2r\sin(\theta)}{r^2\cos^2(\theta)+r^2\sen^2(\theta)}=\lim_{r \to 0}r\cos^2(\theta)\sin(\theta)=0$$Pues $G(\theta)=cos^2(\theta)\sin(\theta)$ esta acotado.
\end{example}
\begin{corollary}
    Sean $f$ y $g$ dos funciones continuas en $a$, entonces:
    \begin{itemize}
        \item $f+g$ es continua en $a$.
        \item $f g$ es continua en $a$.
        \item Si $g(a)\neq 0$ entonces $\frac{f}{g}$ es continua en $a$.
    \end{itemize}
\end{corollary}
\begin{corollary}
    Sea $f:\mathbb{R}^n\to\mathbb{R}^n$ una función continua en $a\in\mathbb{R}^n$, y $g:\mathbb{R}\to\mathbb{R}$ una función continua en $f(a)\in\mathbb{R}$, entonces la composición $g\circ f$ es continua en $a$.
\end{corollary}
\begin{theorem}{Weierstrass}
    \\Sea $f:C\to\mathbb{R}$, $C\subset\mathbb{R}^n$ compacto, y $f$ continua, entonces $f$ alcanza mínimo y máximo en $C$.
    \\Es decir, $\exists x_m,x_n\in C$ tal que $f(x_m)\le f(x)$ y $f(x_n)\ge f(x)\ \forall x\in C$.
\end{theorem}
\newpage\section{Diferenciabilidad en varias variables}
\subsection{Derivadas parciales y direccionales}
\begin{definition}
    Sea $f:\mathbb{R}^2\to\mathbb{R}$ y $(x_0,y_0)$ un punto de $\mathbb{R}^2$, definimos la derivada parcial de $f$ respecto a la variable $x$ en el punto $(x_0,y_0)$ como el siguiente limite (si existe):$$\frac{\partial f}{\partial x}(x_0,y_0)=\lim_{h \to 0}\frac{f(x_0+h,y_0)-f(x_0,y_0)}{h}$$
    De manera análoga se define para la variable $y$.
\end{definition}
\begin{corollary}
    Esta definición, nos permite derivar como en una función de una variable, tomando la otra como una variable, es decir, si queremos derivar una función respecto $x$, solo tenemos que derivar normalmente tomando $y$ como una constante.
\end{corollary}
\begin{example}
    Si queremos hallar $\frac{\partial f}{\partial x}(1,1)$ para la función $f(x,y)=x^2y+5y$, primero derivamos respecto a $x$, donde nos queda la función $\frac{\partial f}{\partial x}(x,y)=2x$, por lo que $\frac{\partial f}{\partial x}(1,1)=2\times 1=2$.
\end{example}
\begin{definition}
    Sea $f:\mathbb{R}^2\to\mathbb{R}$ una función, $(x_0,y_0)$ un punto en $\mathbb{R}^2$, y un vector dirección $v=(v_1,v_2)$. Entonces definimos la derivada direccional de $f$ respecto a la dirección de $v$ en el punto $(x_0,y_0)$ como el siguiente límite, si existe:$$\frac{\partial f}{\partial v}(x_0,y_0)=\lim_{h \to 0}\frac{f(x_0+hv_1,y_0+hv_2)-f(x_0,y_0)}{h}$$
\end{definition}
\begin{corollary}
    $$\frac{\partial f}{\partial x}(x_0,y_0)=\frac{\partial f}{\partial (1,0)} (x_0,y_0)$$
    $$\frac{\partial f}{\partial y} (x_0,y_0)=\frac{\partial f}{\partial (0,1)} (x_0,y_0)$$
\end{corollary}
\begin{example}
    Calculemos la derivada direccional respecto a la dirección $v=(1,1)$ de la función $f(x,y)=x^2+y^2$, entonces:$$\frac{\partial f}{\partial v}(x_0,y_0)=\lim_{h \to 0}\frac{f(x_0+hv_1,y_0+hv_2)-f(x_0,y_0)}{h}=\lim_{h \to 0}\frac{(x_0+h)^2+(y_0+h)^2-x_0^2-y_0^2}{h}$$
    $$=\lim_{h \to 0}\frac{x_0^2+2x_0h+h^2+y_0^2+2y_0h+h^2-x_0^2-y_0^2}{h}=\lim_{h \to 0}\frac{2x_0h+h^2+2y_0h+h^2}{h}=2x_0+2y_0$$
\end{example}
\newpage\subsection{Diferenciabilidad}
\begin{definition}
    Sea $f:\mathbb{R}^n\to\mathbb{R}$, y $a\in\mathbb{R}^n$, decimos que la función $f$ es diferenciable en $a$ si y solo si existe una transformación lineal $df_a:\mathbb{R}^n\to\mathbb{R}$ tal que $$f(a+h)=f(a)+df_a(h)+r(h)$$
    Con $r:\mathbb{R}^n\to\mathbb{R}$ una función que cumple $\lim_{h \to 0} \frac{r(h)}{||h||}=0$.
\end{definition}
\begin{definition}
    Sea $f:\mathbb{R}^2\to\mathbb{R}$ y $a=(x_0,y_0)\in\mathbb{R}^2$, decimos que la función $f$ es diferenciable en $(x_0,y_0)$ si y solo si existen dos reales $A$ y $B$ tal que $$f(x_0+\triangle x,y_0+\triangle y)=f(x_0,y_0)+A\triangle x+B\triangle y+r(\triangle x,\triangle y)$$
    Con $r:\mathbb{R}^2\to\mathbb{R}$ una función que cumple $\lim_{(\triangle x,\triangle y) \to (0,0)}\frac{r(\triangle x,\triangle y)}{||(\triangle x,\triangle y)||}=0$.
\end{definition}
\begin{theorem}
    Sea $f:\mathbb{R}^2\to\mathbb{R}$ una función diferenciable en un punto $(x_0,y_0)$, entonces:
    \begin{itemize}
        \item $f$ es continua en $(x_0,y_0)$.
        \item Existen todas las derivadas parciales en $(x_0,y_0)$ y valen $\frac{\partial f}{\partial x}(x_0,y_0)=A$ y $\frac{\partial f}{\partial y}(x_0,y_0)=B$.
        \item Existen todas las derivadas direccionales de $f$ en $(x_0,y_0)$ y si $v=(v_1,v_2)\in\mathbb{R}^2$ entonces se cumple:$$\frac{\partial f}{\partial v}(x_0,y_0)=\frac{\partial f}{\partial x}(x_0,y_0)v_1+\frac{\partial f}{\partial y_0}v_2$$
    \end{itemize}
\end{theorem}
\begin{definition}
    Dada $f:\mathbb{R}^2\to\mathbb{R}$ diferenciable en $(x_0,y_0)$, definimos el vector gradiente de $f$ en $(x_0,y_0)$ como $\triangledown f( x_{0} ,y_{0}) =\begin{pmatrix}
\frac{\partial f}{\partial x}( x_{0} ,y_{0}) & \frac{\partial f}{\partial y}( x_{0} ,y_{0})
\end{pmatrix}$
\end{definition}
\begin{corollary}
    Con esta notación, podemos expresar la derivada direccional como el producto interno del gradiente con la dirección: $$\frac{\partial f}{\partial v}(x_0,y_0)=\langle \triangledown f(x_0,y_0),v\rangle$$
\end{corollary}
\begin{corollary}
    Consideremos un vector $v$ de norma uno, y las derivadas direccionales de $f$ respecto a $v$. Como tenemos que $\frac{\partial f}{\partial v}(x_0,y_0)=\langle\triangledown f(x_0,y_0),v\rangle$, entonces la derivada direccional se maximiza cuando los vectores $v$ y $\triangledown f(x_0,y_0)$ son colineales, es decir, la dirección del gradiente es la dirección de máximo crecimiento de $f$. Ademas, el gradiente es siempre perpendicular a las curvas de nivel.
\end{corollary}

\tikzset{every picture/.style={line width=0.75pt}} %set default line width to 0.75pt
\begin{center}
\begin{tikzpicture}[x=0.75pt,y=0.75pt,yscale=-1,xscale=1]
%uncomment if require: \path (0,300); %set diagram left start at 0, and has height of 300

%Curve Lines [id:da7458171219907485]
\draw [color={rgb, 255:red, 208; green, 2; blue, 27 }  ,draw opacity=1 ]   (191,97.5) .. controls (236.54,48) and (230.13,55.35) .. (279.49,42.88) ;
\draw [shift={(281,42.5)}, rotate = 165.7] [color={rgb, 255:red, 208; green, 2; blue, 27 }  ,draw opacity=1 ][line width=0.75]    (10.93,-3.29) .. controls (6.95,-1.4) and (3.31,-0.3) .. (0,0) .. controls (3.31,0.3) and (6.95,1.4) .. (10.93,3.29)   ;
%Curve Lines [id:da15762690638125232]
\draw [color={rgb, 255:red, 208; green, 2; blue, 27 }  ,draw opacity=1 ]   (282,29.5) .. controls (229.53,30.49) and (199.6,57.94) .. (174.75,95.36) ;
\draw [shift={(174,96.5)}, rotate = 303.34] [color={rgb, 255:red, 208; green, 2; blue, 27 }  ,draw opacity=1 ][line width=0.75]    (10.93,-3.29) .. controls (6.95,-1.4) and (3.31,-0.3) .. (0,0) .. controls (3.31,0.3) and (6.95,1.4) .. (10.93,3.29)   ;
%Curve Lines [id:da38294230817063457]
\draw [color={rgb, 255:red, 208; green, 2; blue, 27 }  ,draw opacity=1 ]   (375,42.5) .. controls (427.21,46.44) and (444.48,56.2) .. (465.06,92.81) ;
\draw [shift={(466,94.5)}, rotate = 241.07] [color={rgb, 255:red, 208; green, 2; blue, 27 }  ,draw opacity=1 ][line width=0.75]    (10.93,-3.29) .. controls (6.95,-1.4) and (3.31,-0.3) .. (0,0) .. controls (3.31,0.3) and (6.95,1.4) .. (10.93,3.29)   ;
%Curve Lines [id:da7803289982688932]
\draw [color={rgb, 255:red, 126; green, 211; blue, 33 }  ,draw opacity=1 ]   (492,95.5) .. controls (476.08,62.66) and (452.24,29.83) .. (378.12,26.55) ;
\draw [shift={(377,26.5)}, rotate = 2.29] [color={rgb, 255:red, 126; green, 211; blue, 33 }  ,draw opacity=1 ][line width=0.75]    (10.93,-3.29) .. controls (6.95,-1.4) and (3.31,-0.3) .. (0,0) .. controls (3.31,0.3) and (6.95,1.4) .. (10.93,3.29)   ;
%Curve Lines [id:da8217702805694439]
\draw [color={rgb, 255:red, 208; green, 2; blue, 27 }  ,draw opacity=1 ]   (232,134.5) .. controls (311.2,142.42) and (377.66,151.32) .. (417.79,122.39) ;
\draw [shift={(419,121.5)}, rotate = 143.13] [color={rgb, 255:red, 208; green, 2; blue, 27 }  ,draw opacity=1 ][line width=0.75]    (10.93,-3.29) .. controls (6.95,-1.4) and (3.31,-0.3) .. (0,0) .. controls (3.31,0.3) and (6.95,1.4) .. (10.93,3.29)   ;
%Curve Lines [id:da46790486851183033]
\draw [color={rgb, 255:red, 126; green, 211; blue, 33 }  ,draw opacity=1 ]   (419,105.5) .. controls (340.39,91.57) and (311.29,93.48) .. (232.2,107.29) ;
\draw [shift={(231,107.5)}, rotate = 350.07] [color={rgb, 255:red, 126; green, 211; blue, 33 }  ,draw opacity=1 ][line width=0.75]    (10.93,-3.29) .. controls (6.95,-1.4) and (3.31,-0.3) .. (0,0) .. controls (3.31,0.3) and (6.95,1.4) .. (10.93,3.29)   ;

% Text Node
\draw    (282,19.25) -- (375,19.25) -- (375,53.25) -- (282,53.25) -- cycle  ;
\draw (328.5,36.25) node   [align=left] {\begin{minipage}[lt]{60.52pt}\setlength\topsep{0pt}
\begin{center}
$\displaystyle f$ continua
\end{center}

\end{minipage}};
% Text Node
\draw    (137,98.25) -- (230,98.25) -- (230,156.25) -- (137,156.25) -- cycle  ;
\draw (183.5,127.25) node   [align=left] {\begin{minipage}[lt]{60.52pt}\setlength\topsep{0pt}
\begin{center}
Existen $\displaystyle \frac{\partial \ f}{\partial \ v}$
\end{center}

\end{minipage}};
% Text Node
\draw    (420,98.25) -- (529,98.25) -- (529,132.25) -- (420,132.25) -- cycle  ;
\draw (474.5,115.25) node   [align=left] {\begin{minipage}[lt]{71.4pt}\setlength\topsep{0pt}
\begin{center}
$\displaystyle f$ diferenciable
\end{center}
\end{minipage}};
\end{tikzpicture}
\end{center}
\begin{definition}{Plano tangente}
    \\Sea entonces una función $f$ diferenciable en el punto $(x_0,y_0)$, entonces la ecuación del plano tangente por $(x_0,y_0)$ es $$z=f(x_0,y_0)+\frac{\partial f}{\partial x}(x_0,y_0)(x-x_0)+\frac{\partial f}{\partial y}(x_0,y_0)(y-y_0)$$
\end{definition}
\begin{theorem}{Condición suficiente de diferenciabilidad}
    \\Sea $f:\mathbb{R}^2\to\mathbb{R}$ y $(x_0,y_0)\in\mathbb{R}^2$, existen las derivadas parciales en una bola de centro $(x_0,y_0)$, y son continuas en $(x_0,y_0)$. Entonces $f$ es diferenciable en $(x_0,y_0)$.
\end{theorem}
\begin{corollary}
    Notare como $f_x=\frac{\partial f}{\partial x}$ e $f_y=\frac{\partial f}{\partial y}$.
\end{corollary}
\begin{theorem}{Regla de la cadena I}
    \\Sea $f:\mathbb{R}^2\to\mathbb{R}$ una función diferenciable en $(x_0,y_0)$, $I\subset\mathbb{R}$ y $\alpha:I\to\mathbb{R}^2$, $\alpha=(x(t),y(t))$, tal que $\alpha(t_0)=(x_0,y_0)$ y las dos funciones coordenadas $x(t)$ e $y(t)$ son derivables en $t_0$.\\
    Entonces la composición $g(t)=(f\circ\alpha)(t)=f(\alpha(t))$ es derivable en $t_0$, y su derivada vale $$g'(t_0)=f_x(x_0,y_0)x'(t_0)+f_y(x_0,y_0)y'(t_0)=\langle\triangledown f(x_0,y_0),\alpha'(t_0)\rangle$$
\end{theorem}
\begin{theorem}{Regla de la cadena II}
    \\Sea $g:\mathbb{R}^2\to\mathbb{R}^2$, de forma $g(x,y)=(g_1(x,y),g_2(x,y))$, y $f:\mathbb{R}^2\to\mathbb{R}$, que escribiremos como $f(u,v)$. Supongamos que $g(x_0,y_0)=(u_0,v_0)$, que $f$ es diferenciable en $(u_0,v_0)$, y que tanto $g_1$ como $g_2$ son diferenciable en $(x_0,y_0)$, entonces la función $h=(f\circ g)$ tiene derivadas parciales que verifican:$$\frac{\partial h}{\partial x}(x_0,y_0)=\frac{\partial f}{\partial u}(u_0,v_0)\frac{\partial g_1}{\partial x}(x_0,y_0)+\frac{\partial f}{\partial v}(u_0,v_0)\frac{\partial g_2}{\partial x}(x_0,y_0)$$
    $$\frac{\partial h}{\partial y}(x_0,y_0)=\frac{\partial f}{\partial u}(u_0,v_0)\frac{\partial g_1}{\partial y}(x_0,y_0)+\frac{\partial f}{\partial v}(u_0,v_0)\frac{\partial g_2}{\partial y}(x_0,y_0)$$
\end{theorem}
\subsection{Derivadas de orden superior}
\begin{definition}{Derivadas de segundo orden}
\begin{itemize}
    \item $\frac{\partial}{\partial x}(\frac{\partial f}{\partial x})=\frac{\partial^2 f}{\partial x^2}=f_{xx}$
    \item $\frac{\partial}{\partial y}(\frac{\partial f}{\partial x})=\frac{\partial^2 f}{\partial y\partial x}=f_{yx}$
    \item $\frac{\partial}{\partial x}(\frac{\partial f}{\partial y})=\frac{\partial^2 f}{\partial x\partial y}=f_{xy}$
    \item $\frac{\partial}{\partial y}(\frac{\partial f}{\partial y})=\frac{\partial^2 f}{\partial y^2}=f_{yy}$
\end{itemize}
\end{definition}
\begin{definition}
    Sea $f$ una función con derivadas parciales de segundo orden en $(x_0,y_0)$.
    \\La matriz Hessiana de $f$ en $(x_0,y_0)$ es 
\end{definition}
\begin{theorem}{Teorema de Schwarz}
    \\Si $f_{xy}$ y $f_{yx}$ existen en una bola $B((x_0,y_0),\delta)$ y son continuas en $(x_0,y_0)$ entonces $f_{xy}(x_0,y_0)=f_{yx}(x_0,y_0)$.
\end{theorem}
\begin{definition}
    Decimos que $f:\mathbb{R}^2\to\mathbb{R}$ es de clase $C^k$ si existen y son continuas sus derivadas de orden $k$.
\end{definition}
\begin{corollary}
    Si $f\in C^1\Rightarrow\ f$ es diferenciable.\\
    Si $f\in C^2\Rightarrow\ f_{xy}=f_{yx}$.
\end{corollary}
\newpage\subsection{Funciones de $\mathbb{R}^n$ a $\mathbb{R}^m$}
\begin{definition}
    Una función $f:\mathbb{R}^n\to\mathbb{R}^m$ es de la forma $$f(x_1,x_2,\cdots,x_n)=(f_1(x_1,x_2,\cdots,x_n),f_2(x_1,x_2,\cdots,x_n),\cdots,f_m(x_1,x_2,\cdots,x_n))$$
\end{definition}
\begin{definition}
    Decimos que $f:\mathbb{R}^n\to\mathbb{R}^m$ es diferenciable en $a=(a_1,a_2,\cdots,a_n)\in\mathbb{R}^n$ si y solo si existe una transformación lineal $df_a:\mathbb{R}^n\to\mathbb{R}^m$ tal que $f(a+h)=f(a)+df_a(h)+r(h)$ con $\lim_{h \to 0}\frac{r(h)}{||h||}=0$.
\end{definition}
\begin{corollary}
    La matriz asociada de $df_a$ tiene como entradas las derivadas parciales (matriz Jacobiana).
\end{corollary}
\begin{definition}
    Definimos la matriz Jacobiana como $$J_f(a)=\begin{pmatrix}
\frac{\partial f_{1}}{\partial x_{1}}( a) & \frac{\partial f_{1}}{\partial x_{2}}( a) & \dotsc  & \frac{\partial f_{1}}{\partial x_{n}}( a)\\
\frac{\partial f_{2}}{\partial x_{1}}( a) & \frac{\partial f_{2}}{\partial x_{2}}( a) & \dotsc  & \frac{\partial f_{1}}{\partial x_{n}}( a)\\
\vdots  & \vdots  & \ddots  & \vdots \\
\frac{\partial f_{m}}{\partial x_{1}}( a) & \frac{\partial f_{m}}{\partial x_{2}}( a) & \dotsc  & \frac{\partial f_{m}}{\partial x_{n}}( a)
\end{pmatrix}=\begin{pmatrix}
\triangledown f_{1}( a)\\
\triangledown f_{2}( a)\\
\vdots \\
\triangledown f_{m}( a)
\end{pmatrix}$$
\end{definition}
\begin{theorem}{Regla de la cadena III}
    \\Sea $g:\mathbb{R}^k\to\mathbb{R}^n$, diferenciable en $a\in\mathbb{R}^k$, y $f:\mathbb{R}^n\to\mathbb{R}^m$, diferenciable en $g(a)$. Entonces $f\circ g:\mathbb{R}^k\to\mathbb{R}^m$ es diferenciable en $a$ y ademas su diferencial es $d(f\circ g)_a=df_{g(a)}\circ dg_a$.
\end{theorem}
\begin{corollary}
    Observar que la diferencial de la composición es la composición de los diferenciales, por lo tanto, la matriz jacobiana de $f\circ g$ es el producto de las matrices Jacobianas de $f$ y $g$: $J_{f\circ g}(a)=J_f(g(a))\cdot J_g(a)$.
\end{corollary}
\subsection{Desarrollo de Taylor}
\begin{definition}
Sea $f:\mathbb{R}\to\mathbb{R}$ con derivadas continuas hasta el orden $k+1$ en $a\in\mathbb{R}$. Entonces:\\
$f(x)=f(a)+f'(a)(x-a)+\frac{f''(a)}{2}(x-a)^2+\cdots+\frac{f^{(k)}(a)}{k!}(x-a)^k+r(x-a)$ con $\lim_{x \to a}\frac{r(x-a)}{(x-a)^k}=0$.
\end{definition}
\begin{theorem}{Teorema de Taylor}
    Sea $f:\mathbb{R}^2\to\mathbb{R}$ de clase $C^{k+1}$ (derivadas parciales de orden $k+1$ continuas) y $a\in\mathbb{R}^2$.
    \\Entonces:$$f(a+h)=f(a)+df_a(h_1,h_2)+\frac{d^2f_a}{2}(h_1,h_2)+\cdots+\frac{d^kf_a}{k!}(h_1,h_2)+r(h_1,h_2)$$
    Con $\lim_{(h_1,h_2) \to (0,0)}\frac{r(h_1,h_2)}{(\sqrt{h_1^2+h_2^2})^k}=0$
    \\O: $$f(x)=f(a)+df_a(x-a)+\frac{d^2f_a}{2}(x-a)+\cdots+\frac{d^{(k)}f_a}{k!}(x-a)+r(x-a)$$
    Con $\lim_{x \to a}\frac{r(x-a)}{||x-a||^k}=0$
\end{theorem}
\begin{corollary}
    Sea $a=(x_0,y_0)$, la aproximación lineal es: $f(x,y)\approx f(x_0,y_0)+J_f(x_0,y_0)\begin{pmatrix} x-x_0\\y-y_0 \end{pmatrix} $.
    \\La aproximación cuadrática es $f(x,y)\approx f(x_0,y_0)+J_f(x_0,y_0) \begin{pmatrix} x-x_0 \\ y-y_0 \end{pmatrix}+\frac{1}{2} \begin{pmatrix} x-x_0 & y-y_0 \end{pmatrix} H_f(x_0,y_0) \begin{pmatrix} x-x_0 \\ y-y_0 \end{pmatrix}  $.
\end{corollary}\newpage
\subsection{Integrales dobles}
\begin{corollary}{Propiedades}
    \\Sea $D$ compacto en $\mathbb{R}^2$:
    \begin{enumerate}
        \item $\int\int_D\alpha f(x,y)+\beta g(x,y)dxdy=\alpha\int\int_Df(x,y)dxdy+\beta\int\int_Dg(x,y)dxdy$.
        \item $\int\int_{D_1\cup D_2}f(x,y)dxdy=\int\int_{D_1}f(x,y)dxdy+\int\int_{D_2}f(x,y)dxdy$\\Siendo $D_1$ y $D_2$ disjuntos, es decir $D_1\cap D_2=\emptyset$.
        \item Si $f(x,y)\ge0\ \forall(x,y)\in D$ entonces $\int\int_Df(x,y)dxdy\ge0$.
        \item Si $f(x,y)\ge g(x,y)\ \forall(x,y)\in D$ entonces $\int\int_Df(x,y)dxdy\ge\int\int_Dg(x,y)dxdy$.
    \end{enumerate}
\end{corollary}
\begin{theorem}{Teorema de Fubini 1}
    \\Sea $D=[a,b]\times[c,d]$ un rectangulo en $\mathbb{R}^2$. Entonces:$$\int\int_Df(x,y)dxdy=\int_{c}^{d}\left( \int_{a}^{b} f(x,y) dx \right) dy=\int_{a}^{b} \left( \int_{c}^{d} f(x,y) dy \right)  dx$$
\end{theorem}
\begin{theorem}{Teorema de Fubini 2}
    \\Sea $D=\{(x,y)\in\mathbb{R}^2:\ a\le x\le b,\ \alpha(x)\le y\le \beta(x)\}$. Entonces: $$\int\int_Df(x,y)dxdy=\int_{a}^{b} \left( \int_{\alpha(x)}^{\beta(x)} f(x,y) dy \right)  dx=\int_{c}^{d} \left( \int_{\gamma(y)}^{\delta(y)} f(x,y) dx \right)  dy$$
\end{theorem}

\end{document}
