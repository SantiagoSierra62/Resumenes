\documentclass[10pt]{article}
\usepackage[spanish]{babel}
\usepackage{amsmath, amssymb, amsthm, tikz}
\usepackage[margin=1in]{geometry}
\usepackage[shortlabels]{enumitem}
\usetikzlibrary{babel}

\theoremstyle{definition}
\newtheorem{definition}{Definición}[section]
\newtheorem{theorem}{Teorema}[section]
\newtheorem{corollary}{Corolario}[theorem]

\title{Cálculo Diferencial e Integral en varias variables}
\author{Santiago Sierra}

\begin{document}
\maketitle \tableofcontents \newpage
\section{Números Complejos}
\begin{definition}
	Un numero complejo es un numero de forma $z=a+bi$ y $a,b\in \mathbb{R}$, donde $i^2=-1$, conocemos los números reales a y b como parte real e imaginaria respectivamente del numero $z$.
	$$
		Re(z)=a \ \ \ \ Im(z)=b
	$$
\end{definition}
Se le llama $i$ a la unidad imaginaria.
Esta expresión que describimos se le llama forma binómica del numero.
\begin{definition}
	Dos números complejos $z, w$ son iguales si y solo si
	$$
		Re(z)=Re(w) \ \text{y} \ Im(z)=Im(w)
	$$
\end{definition}
\subsection{Suma y Producto de números complejos}
Dados dos números complejos $z=a+bi$ y $w=c+di$ definimos la suma de $z+w$ y el producto $zw$ mediante:
\begin{gather*}
	\begin{array}{l}
		z+w=(a+bi)+(c+di)=(a+c)+(b+d)i \\
		zw=(a+bi)(c+di)=ac+adi+bci+bdi^2=(ac-bd)+(ad+bci)i
	\end{array}
\end{gather*}
Ejemplo:
$$
	\begin{array}{l}
		(1-i)+(4+7i)=(1+4)+(-1+7)i=5+6i \\
		(-1+3i)(2-5i)=(-1)(2-5i)+(3i)(2-5i)=(-2+5i)+(6i-15i^2)=(-2+5i)+(15+6i)=13+11i
	\end{array}
$$
\textbf{Propiedades.} Sean $z$,$w$,$v\in\mathbb{C}$
\begin{enumerate}
	\item Conmutativas: $z+w=w+z$ y $zw=wz$
	\item Asociativas: $(z+w)+v=z+(w+v)$ y $(zw)v=z(wv)$
	\item Cada numero complejo $z=a+bi$ tiene un elemento opuesto, $-z=-a-bi$, tal que $z+(-z)=0$
	\item Distributiva (del producto respecto a la suma) $z(w+v)=zw+zv$.
\end{enumerate}
\subsection{Conjugado de un complejo}
\begin{definition}
	Sea $z=a+bi$ un numero complejo. Se define el conjugado de $z$ y se representa por $\overline{z}$, como el numero complejo $\overline{z}=a-bi$.
\end{definition}
Geométricamente, un complejo $z=a+bi$ se representa por el punto $P=(a,b)$, y su conjugado $\overline{z}=a-bi$ por el punto $P'=(a,-b)$\\\\
\begin{minipage}{0.3\textwidth}
	\begin{tikzpicture}[x=50pt,y=50,yscale=-1,xscale=1]
		\draw[<-] (0,0) -- (0,3);
		\draw[->] (-0.5,1.5) -- (2,1.5);
		\draw[-] (0,1.5) -- (1,2.5);
		\draw[-] (0,1.5) -- (1,0.5);
		\draw[dotted] (1,2.5) -- (1,0.5);
		\draw (1.3,0.5) node {$z$};
		\draw (1.3,2.5) node {$\overline{z}$};
		\foreach \Point in {(1,0.5),(1,2.5)}{
				\node at \Point {\textbullet};
				\draw (-0.25,0) node {Im};
				\draw (2,1.75) node {Re};
			}
	\end{tikzpicture}
\end{minipage}
\hfill
\begin{minipage}{0.65\textwidth}
	Propiedades:
	\begin{enumerate}
		\item $\overline{\overline{z}}=z$
		\item $\overline{z_1+z_2}=\overline{z_1}+\overline{z_2}$
		\item $\overline{z_1z_2}=\overline{z_1}\ \overline{z_2}$
		\item Si $z_2\neq 0$, $\overline{(\frac{z_1}{z_2})}=\frac{\overline{z_1}}{\overline{z_2}}$
		\item $|z|^2=z\overline{z}=Re(z)^2+Im(z)^2$. Por lo tanto, $|z|^2 \ge 0\ \forall z\neq 0$
		\item $z+\overline{z}=2 Re(z)$
		\item $z-\overline{z}=2i\ Im(z)$
	\end{enumerate}
	Observación, para dividir dos números complejos $\frac{z}{w}$, basta con multiplicar el numerador y denominador por el conjugado del denominador.
	$$
		\frac{z}{w}=\frac{z\overline{w}}{w\overline{w}}=\frac{z\overline{w}}{|w|^2}
	$$
\end{minipage}
\noindent \newpage
\subsection{Módulo}
\begin{minipage}{0.65\textwidth}
	\begin{definition}
		Definimos el módulo de un complejo $z=a+bi$ como el número real $|z|=\sqrt{a^2+b^2}$
	\end{definition}
	Propiedades del módulo. Sean $z_1$ y $z_2$ números complejos:
	\begin{enumerate}
		\item $|z|=0$ si, y sólo si, $z=0$
		\item $|z|=|\overline{z}|$
		\item $|z_1z_2|=|z_1||z_2|$
		\item Si $z\neq 0$, $|\frac{z_1}{z_2}|=\frac{|z_1|}{|z_2|}$
		\item \textbf{Desigualdad triangular:} $|z_1+z_2|\le |z_1|+|z_2|$
	\end{enumerate}
\end{minipage}
\hfill
\begin{minipage}{0.3\textwidth}
	\begin{tikzpicture}[x=0.75pt,y=0.75pt,yscale=-1,xscale=1]
		\draw  (0,108) -- (141,108)(17,0) -- (17,125) (134,103) -- (141,108) -- (134,113) (12,7) -- (17,0) -- (22,7)  ;
		\draw [-] (18,108) -- (90,45);
		\draw (90,45)[fill={rgb, 255:red, 0; green, 0; blue, 0}]circle [x radius= 3.35, y radius= 3.35];
		\draw (35.48,60.2) node [anchor=north west][inner sep=0.75pt]  [rotate=-319.32] [align=left] {$|z|$};
		\draw (95,30) node [anchor=north west][inner sep=0.75pt] [align=left] {z};
		\draw (0,0) node {Im};
		\draw (130,120) node {Re};
	\end{tikzpicture}
\end{minipage}
\subsection{Forma Polar}
Sabemos que cualquier complejo $z=a+bi$ puede ser considerado un punto $(a,b)$ y que cualquier punto de este tipo puede representarse con coordenadas polares $(r,\theta)$ con $r\ge0$.
\begin{definition}
	Cualquier complejo $z$ se puede representar como $z=r(cos(\theta)+i \ sen(\theta))=re^{i\theta}$, lo cual llamaremos forma polar. Siendo $r=|z|$ y $\theta=arg(z)$.
\end{definition}
\subsubsection{Argumento}
\begin{minipage}{0.65\textwidth}
	\begin{definition}
		Definimos el argumento de $z$ como la función\\\\ $\operatorname {arg}(z)={\begin{cases}\arctan \left({\frac  ba}\right)&\qquad a>0\\\arctan \left({\frac  ba}\right)+\pi &\qquad b\geq 0,a<0\\\arctan \left({\frac  ba}\right)-\pi &\qquad b<0,a<0\\+{\frac  {\pi }{2}}&\qquad b>0,a=0\\-{\frac  {\pi }{2}}&\qquad b<0,a=0\end{cases}}$
	\end{definition}
\end{minipage}
\hfill
\begin{minipage}{0.3\textwidth}
	\begin{tikzpicture}[x=0.75pt,y=0.75pt,yscale=-1,xscale=1]
		\draw (0,110) -- (135,110)(14,0) -- (14,123.5) (128.5,105) -- (135.5,110) -- (128.5,115) (9,7) -- (14,0) -- (19,7)  ;
		\draw    (14,110) -- (80,50) ;
		\draw [shift={(80,50)}][fill={rgb, 255:red, 0; green, 0; blue, 0 }](0, 0) circle [x radius= 3.35, y radius= 3.35]   ;
		\draw   (42,84) .. controls (48,89) and (52,97) .. (52,106) .. controls (52,107) and (51,110) .. (51,111) ;
		\draw (92,36.25) node [anchor=north west][inner sep=0.75pt]   [align=left] {$a+bi$};
		\draw (30,70) node [anchor=north west][inner sep=0.75pt]  [rotate=-310.79] [align=left] {$|z|$};
		\draw (56,84) node [anchor=north west][inner sep=0.75pt]   [align=left] {$\theta$};
		\draw (-13,0) node [anchor=north west][inner sep=0.75pt]   [align=left] {Im};
		\draw (118,122) node [anchor=north west][inner sep=0.75pt]   [align=left] {Re};
	\end{tikzpicture}
\end{minipage}
\subsubsection{Operaciones en Forma Polar}
\begin{definition}
	Sean $z_1=r_1(cos(\theta_1)+i \ sen(\theta_1))$ y $z_2=r_2(cos(\theta_2)+i \ sen(\theta_2))$, definimos su multiplicación como $z_1z_2=r_1r_2(cos(\theta_1+\theta_2)+i \ sen(\theta_1+\theta_2))$
\end{definition}
\begin{definition}
	Definimos la división de dos complejos como $\frac{z_1}{z_2}=\frac{r_1}{r_2}(cos(\theta_1-\theta_2)+i \ sen(\theta_1-\theta_2))$
\end{definition}
Observación: Si $z=r(cos(\theta)+ i \ sen(\theta))$ entonces $\frac{1}{z}=\frac{1}{r}(cos(\theta)+i \ sen(\theta))$
\begin{theorem}{Teorema de De Movire.}
	\\ Sea $z=r(cos(\theta)+i\ sen(\theta))$ y $n \in \mathbb{Z}^{+}$, entonces $z^n=(r(cos(\theta) + i \ sen(\theta)))^n=r^n(cos(n\theta)+i \ sen(n\theta))$
\end{theorem}
\subsection{Raíces de un numero complejo}
\subsubsection{Raíz cuadrada}
Si deseamos hallar $\sqrt{a+bi}$ una forma rápida de hacerlo es diciendo: $$\sqrt{a+bi}=c+di \rightarrow a+bi=(c+di)^2=c^2-d^2+2cdi\rightarrow \begin{cases}a=c^{2} -d^{2}\\b=2cd\end{cases}$$
\newpage \subsubsection{Raíces complejas}
Sea $z^n=r(cos(\theta)+i\ sen(\theta))$ y $n\in \mathbb{Z}^{+}$. Entonces, $z$ tiene $n$ raíces enésimas distintas.\\
Las raíces se hallan como: $$z_k=r^{\frac{1}{n}}\left(cos\left(\frac{\theta +2k\pi}{n}\right)+i\ sen\left(\frac{\theta + 2k\pi}{n}\right)\right)=r^{\frac{1}{n}}e^{\frac{\theta + 2k\pi}{n}i}$$
Donde $k=\{0,1,2,\dots,n-1\}$.
\section{Ecuaciones Diferenciales}
\begin{definition}
	Una ecuación diferencial es una igualdad en la cual la incógnita es una función desconocida y sus derivadas, $y=f(x)$ definida y derivable.
\end{definition}
Le llamamos ecuación de orden 1 si la única derivada de la función desconocida que aparece es la derivada primera, de orden 2 si la derivada de mayor orden que aparece es 2 y así sucesivamente.
\subsection{Ecuación diferencial de variables separables}
\begin{definition}
	Una ecuación diferencial se le llama de variables separables si es de la forma $y'=A(y)B(x)$
\end{definition}
Solución: $$\frac{y'(x)}{A(y(x))}=B(x)\rightarrow \int \frac{y'(x)}{A(y(x))} dx=\int B(x)dx+C\rightarrow \int \frac{dy}{A(y)}=\int B(x)dx+C$$
\subsection{Solución de una ecuación diferencial con condiciones o datos iniciales}
Por lo general existen infinitas soluciones a una misma ecuación diferencial, sin embargo si se dan datos iniciales apropiados, por lo general existe una única función solución que cumple los datos.\\\\
En una ecuación diferencial de primer orden se le llama dato inicial a una condición del tipo: $y(x_0)=0$.\\
Donde $x_0$ e $y_0$ son valores reales dados.\\\\
En una ecuación diferencial de segundo orden se le llaman datos iniciales a dos condiciones del tipo: $y(x_0)=y_0$, $y'(x_1)=y_1$.\\
Donde $x$ e $y$ son valores reales dados.\\\\
Para determinar la solución que verifica los datos iniciales, primero tendremos que haber hallado todas las soluciones (esto no quiere decir haber despejado la función).\\
Al conjunto de todas las soluciones se le llama solución general, la cual depende usualmente de una constante arbitraria si la ecuación es de primer orden, o de 2 si es de segundo orden.
\subsection{Ecuación diferencial lineal de primer orden homogénea}
\begin{definition}
	Se llama ecuación diferencial lineal de primer orden homogénea a una ecuación del tipo: $y'+a(x)y=0$
\end{definition}
Solución general:
$$y'=-a(x)y\rightarrow \frac{y'}{y}=-a(x)\rightarrow\int\frac{dy}{y}=-\int a(x)dx\rightarrow log(y)=-\int a(x)dx+C$$$$y=e^{-\int a(x)dx+C}=Ke^{-\int a(x)dx}$$
Siendo $K$ la variable arbitraria.
\newpage \subsection{Ecuación diferencial lineal de primer orden no homogénea}
\begin{definition}
	Se llama ecuación diferencial lineal de primer orden no homogénea a una ecuación del tipo: $y'+a(x)y=r(x)$, siendo $r(x)\neq 0$, si no, seria homogénea.
\end{definition}
Solución:
\begin{enumerate}
	\item Hallar la solución general, $y_h(x)$ de la ecuación diferencial lineal homogénea correspondiente, es decir la misma ecuación diferencial, pero sustituyendo $r(x)$ por la función nula.
	\item Hallar una solución particular de la ecuación $y_p(x)$ diferencial dada usando el método de variación de constante.
	\item Sumar $y(x)=y_h(x)+y_p(x)$
\end{enumerate}
\subsubsection{Método de variación de constante}
Para hallar la solución particular $y_p(x)$ de la ecuación diferencial, probaremos  con una función de cierto tipo como se describe:
\begin{enumerate}
	\item Tomaremos $y_p(x)=y_h(x)$, con la diferencia de que la constante arbitraria, ahora sera una función desconocida a determinar.
	\item Sustituimos $y_p(x)$ en la ecuación no homogénea dada, haciendo que la verifique y despejando la función. De todas las posibles funciones que se despejen (en general son infinitas), habrá que elegir solo una.
	\item Una vez hallada la función, la sustituimos en la expresión $y_p(x)$ para obtener la solución particular buscada.
\end{enumerate}
\textbf{Ejemplo:}\\
Sea $y'-cos(x)y=cos(x)$\\
Hallamos $y'-cos(x)y=0 \rightarrow y_h=Ke^{sen(x)}$\\
Decimos por el paso 2, que $y_p(x)=y_h(x)$ pero con la constante arbitraria como función,\\ así que
$$\begin{array}{l}
		y_p(x)=K(x)e^{sen(x)}                                            \\
		(K(x)e^{sen(x)})'-cos(x)K(x)e^{sen(x)}=cos(x)                    \\
		K'(x)e^{sen(x)}+K(x)cos(x)e^{sen(x)}-cos(x)K(x)e^{sen(x)}=cos(x) \\
		K'(x)e^{sen(x)}=cos(x)\rightarrow K'(x)=cos(x)e^{-sen(x)}        \\
		K(x)=\int cos(x)e^{-sen(x)}dx+C=-e^{-sen(x)}+C
	\end{array}$$
Ahora, elegimos la solución con $C=0$ y reemplazamos la función en $y_p(x)$
$$y_p(x)=K(x)e^{sen(x)}=-e^{-sen(x)}e^{sen(x)}=-e^{sen(x)-sen(x)}=-e^0=-1$$
Por lo tanto, $y(x)=y_h(x)+y_p(x)=Ke^{sen(x)}-1$
\newpage \subsection{Ecuación diferencial lineal de segundo orden a coeficientes constantes y homogénea}
\begin{definition}
	Una ecuación diferencial de segundo orden se llama lineal a coeficientes constantes y homogénea si es de la forma $y''+ay'+by=0$, donde $a$ y $b$ son constantes dadas independientes.
\end{definition}
\begin{theorem}{Estructura vectorial de las soluciones de la ecuación lineal homogénea.}\\
	Todas las funciones solución de la ecuación diferencial de segundo orden homogénea forman un espacio vectorial de dimensión 2.
\end{theorem}
Soluciones exponenciales: Buscaremos soluciones $y(x)=e^{\lambda x}$, donde $\lambda$ es una constante real a determinar, que sean solución de la ecuación diferencial $y''+ay'+by=0$. Sustituyendo en la ecuación diferencial $y=e^{\lambda x}$, $y'=\lambda e^{\lambda x}$, $y''=\lambda ^{2}e^{\lambda x}$, se obtiene: $(\lambda ^2+a\lambda+b)e^{\lambda x}=0\Leftrightarrow \lambda ^2+a\lambda+b=0$. Siendo $\lambda$ raíz de la ecuación de segundo grado, llamada \underline{ecuación caracteristica}.
\subsubsection{Solución general de la ecuación lineal homogénea de segundo orden}
Una vez encontradas las raíces de la ecuación característica, hay 3 casos.
\begin{enumerate}[A)]
	\item La ecuación característica tiene dos raíces distintas. La solución general es: $y(x)=C_1e^{\lambda_1 x}+C_2e^{\lambda_2 x}$
	\item La ecuación característica tiene una raíz doble. La solución general es: $y(x)=e^{\lambda x}(C_1+C_2 x)$
	\item La ecuación característica tiene dos raíces complejas conjugadas de la forma $\alpha\pm i\beta$. La solución general es: $y(x)=e^{\alpha x}(C_1 cos(\beta x)+C_2 sen(\beta x))$
\end{enumerate}
\subsection{Ecuación diferencial lineal de segundo orden a coeficientes constantes no homogénea}
\begin{definition}
	Una ecuación diferencial de segundo orden se llama lineal a coeficientes constantes y no homogénea si es del tipo: $y''+ay'+by=r(x)$.
\end{definition}
Solución:
\begin{enumerate}
	\item Hallar la solución general de $y_h(x)$ de la ecuación lineal homogénea correspondiente.
	\item Hallar una solución particular $y_p(x)$ de la ecuación no homogénea dada usando el método de coeficientes indeterminados.
	\item Sumar $y(x)=y_h(x)+y_p(x)$
\end{enumerate}
\subsubsection{Método de coeficientes indeterminados}
\begin{enumerate}
	\item Si $r(x)=e^{kx}P(x)$, donde $P(x)$ es un polinomio de grado $n$, probar $y_(x)=e^{kx}Q(x)$, donde $Q(x)$ es un polinomio de grado $n$ con coeficientes a determinar sustituyendo $y_p(x)$ en la ecuación diferencial.
	\item Si $r(x)=e^{kx}P(x)cos(mx)$ o $r(x)=e^{kx}P(x)sen(mx)$, donde $P$ es un polinomio de grado $n$, probar $y_p(x)=e^{kx}Q(x)cos(mx)+e^{kx}R(x)sen(mx)$, donde $Q$ y $R$ son polinomio de grado $n$.
\end{enumerate}
Si algún termino de $y_p(x)$ es también termino de $y_h(x)$, hay que multiplicar $y_p$ por $x$, o $x^2$ si es termino 2 veces.
\begin{theorem}
	Sea una ecuación diferencial $y''+ay'+by=r(x)$ donde $r(x)$ es una función conocida que puede descomponerse como suma $r(x)=r_1(x)+r_2(x)$.
	Se consideran las ecuaciones diferenciales auxiliares:\\
	$y_{1p} \rightarrow y''+ay'+by=r_1(x)$\\
	$y_{2p} \rightarrow y''+ay'+by=r_2(x)$\\
	Siendo $y_p(x)=y_{1p}+y_{2p}$.\\Este teorema puede aplicarse tantas veces como $r(x)$ pueda descomponerse, habiendo una suma de tres o mas sumandos en vez de dos como se mostró.
\end{theorem}
\newpage \section{Sucesiones y Series}
\subsection{Sucesiones}
\begin{definition}
	Las sucesiones son funciones $a: \mathbb{N} \rightarrow \mathbb{R}$, donde a cada natural, se le asocia un real $a_n$.
\end{definition}
Ejemplos:
\begin{enumerate}
	\item Sucesión armónica: $a_n=\frac{1}{n}$
	\item $a_0=1$, $a_1=1$, $a_n=a_{n-1}+a_{n-2}$
	\item Sucesión CTE: $a_n=c$ $\forall n \in \mathbb{N}$
	\item Sucesión identidad: $a_n=n$ $\forall n \in \mathbb{N}$
\end{enumerate}
\subsubsection{Convergencia}
\begin{definition}{Limite de una sucesión.}
	\\Una sucesión $a_n$ tiene el limite $L$ y se escribe como $$\\lim_{n \to \infty} a_n = L \text{ o } a_n \to L \text{ cuando } n \to \infty$$
	Si $\lim _{n\rightarrow \infty } a_{n} =L\begin{cases}
			=L\ \text{ (finito)} & \text{En este caso converge }(\mathbb{C}) \\
			=\infty              & \text{En este caso diverge }(\mathbb{D})  \\
			                     & \text{En este caso oscila }(\mathbb{O})
		\end{cases}$\\\\
	Y para todo $\epsilon > 0$ hay un correspondiente entero $N$ tal que si $n>N$ entonces $|a_n-L|<\epsilon$
\end{definition}
Ejemplos
\begin{enumerate}
	\item $\lim_{n \to + \infty} \frac{1}{n}=0$.
	\item $a_n=C$ $\forall n \rightarrow \lim_{n \to + \infty} a_n=c$
	\item $a_n=n$ $\forall n \in \mathbb{N} \rightarrow \lim_{n \to + \infty} a_n=+\infty$
	\item $a_n=(-1)^n$ $\forall n \in \mathbb{N}$
\end{enumerate}
\begin{theorem}
	Si $\lim_{x \to \infty} f(x)=L$ y $f(n)=a_n$ cuando $n$ es un entero, entonces $\lim_{n \to \infty} a_n = L$.
\end{theorem}
Propiedades:
\begin{enumerate}
	\item Si $\lim_{n \to + \infty} a_n=L$, $\lim_{n \to + \infty} b_n=M$ $\Rightarrow (a_n+b_n)_{n \in \mathbb{N}}$ es convergente y $\lim_{n \to + \infty} a_n+b_n=\lim_{n \to + \infty}a_n+\lim_{n \to + \infty} b_n=L+M$.
	\item $(\lambda a_n)_{n \in \mathbb{N}}$ es convergente si $\lim_{n \to +\infty} (\lambda a_n)=\lambda L$.
	\item $(a_n b_n)_{n\in\mathbb{N}}$ es convergente si $\lim_{n \to +\infty} a_n b_n=LM$
	\item Si $b_n=0$ $\forall n\in\mathbb{N}$, $M \ne 0$, $(\frac{a_n}{b_n})_{n\in\mathbb{N}}$ es convergente y $\lim_{n \to + \infty} \frac{a_n}{b_n}=\frac{L}{m}$
\end{enumerate}
\begin{theorem}
	Si $\lim_{n \to \infty} |a_n|=0$, entonces $\lim_{n \to \infty} a_n=0$
\end{theorem}
\begin{theorem}
	Si $\lim_{n \to \infty} a_n = L$ y la función $f$ es continua en $L$, entonces $$\lim_{n \to \infty} f(a_n)=f(L)$$
\end{theorem}\newpage
\subsubsection{Monotonía}
\begin{definition}
	Una sucesión $a_n$ se le llama monótona creciente si $a_n\le a_{n+1}$ para toda $n\ge 1$.\\
	Y se le denomina monótona decreciente si $a_n\ge a_{n+1}$.
\end{definition}
\begin{corollary}
	Si $a_{n+1}>a_n$ $\forall n\in\mathbb{N}$ es monótona estrictamente creciente.\\
	Si $a_{n+1}<a_n$ $\forall n\in\mathbb{N}$ es monótona estrictamente decreciente.
\end{corollary}
\subsubsection{Acotación}
\begin{definition}
	Una sucesión $a_n$ esta acotada superiormente si existe un numero $M$ tal que $a_n \le M$ para toda $n\ge 1$.\\
	Esta acotada inferiormente si existe un numero $m$ tal que $m\le a_n$ para toda $n\ge 1$.\\
	Si esta acotada superior e inferiormente, entonces $a_n$ es una sucesión acotada.
\end{definition}
\begin{theorem}{Teorema de la sucesión monótona.}
	\\Toda sucesión monótona y acotada es convergente.
\end{theorem}
Obs: Si $(a_n)_{n\in\mathbb{R}}$ es monótona decreciente y acotada, entonces $\lim a_n=\text{Inf}\{a_1,a_2,\dots\}$.
\subsubsection{Sub-sucesión}
\begin{definition}
	Dada una sucesión $a_n$ y otra extricamente creciente $(x_n):\mathbb{N}\to\mathbb{N}$, llamaremos sub-sucesión de $a_n$ a la sucesión $a\circ x:\mathbb{N}\to\mathbb{R}$, y lo denotamos como $a_{x_n}$.
\end{definition}
\begin{theorem}
	Si $\lim a_n=L$, entonces toda sub-sucesión de $a_n$ converge a $L$.
\end{theorem}
\subsubsection{Punto de acumulación}
\begin{definition}
	Sea una sucesión $a_n$, y su sub-sucesión $a_{n_k}$, $h$ es un punto de aglomeración si $a_{n_k}\to h$.
\end{definition}
\begin{theorem}{Teorema de Bolzano Weirstrass.}
	\\Todo conjunto infinito y acotado tiene (al menos) un punto de acumulación.
\end{theorem}
\begin{corollary}
	El punto de acumulación no tiene por que pertenecer al conjunto.
\end{corollary}
\begin{theorem}
	Toda sub-sucesión $a_n$ acotada tiene una sub-sucesión convergente.
\end{theorem}
\begin{corollary}
	Sea $\{a_n\}$ una sucesión de términos positivos ($a_n>0$ $\forall n$).\\
	Si $\lim \ \frac{a_{n+1}}{a_{n}} =L\begin{cases}< 1 & \text{Entonces $a_n$ converge a $0$}\\ >1 & \text{Entonces $a_n$ diverge}\end{cases}$
\end{corollary}
Obs: Si $L=1$ no se puede decir nada.
\begin{theorem}
	Una función $f: \mathbb{R}\rightarrow\mathbb{R}$ es continua en $a\in\mathbb{R}$.\\
	$\Leftrightarrow$ para toda sucesión ${a_n}$ tal que $\lim a_n=a$ se tiene que $\lim f(a_n)=f(a)$.
	$$(\forall {a_n}, a_n\rightarrow a \Rightarrow f(a_n)=f(a))$$
\end{theorem}
\newpage
\subsection{Series}
\begin{definition}
	En general, si se trata de sumar los términos de de una sucesión infinita $\{a_n\}_{n=1}^{\infty}$, se obtiene una expresión de la forma $$a_1+a_2+a_3+\cdots+a_n+\cdots$$
	que se denomina serie y se denota con el símbolo $$\sum_{n=1}^{\infty} a_n \text{ o } \sum a_n$$
\end{definition}
\begin{definition}
	Dada una serie $\sum_{n=1}^{\infty} a_n$, sea $s_n$ la n-enésima suma parcial: $$s_n=\sum_{i=1}^{n} a_i$$
	$$\lim _{n \to \infty} s_{n} =\lim _{n \to \infty} (a_{1} ,a_{2} ,\dotsc ,a_{n} )=\lim _{n \to \infty} \left(\sum _{k=1}^{n} a_{k} \right)\begin{cases}=L \text{ (finito)} & \text{Decimos que la serie converge (}\mathbb{C}\text{) y}\sum _{n=1}^{\infty } a_{n} =L  \\=\infty  & \text{Decimos que la serie diverge (}\mathbb{D}\text{)}\\ & \text{La serie oscila (}\mathbb{O}\text{)}\end{cases}$$
\end{definition}
Obs: $\sum_{n=0}^{\infty} a_n = \sum_{n=0}^{k-1} a_n + \sum_{n=k}^{\infty} a_n$
\subsubsection{Serie geométrica}
La serie geométrica $$\sum_{n=0}^{\infty} aq^n = \sum_{n=1}^{\infty} aq^{n-1}$$
es convergente si $|q|<1$ y su suma es
$$\sum_{n=0}^{\infty} aq^n=\sum_{n=1}^{\infty} aq^{n-1}=\frac{a}{1-q}$$
Si $|q|\ge 1$, la serie geométrica es divergente.\\
En el caso $$\sum_{n=n_0}^{\infty} aq^n=\frac{aq^{n_0}}{1-q}$$
\begin{theorem}
	Si $\sum a_n$ converge, entonces $a_n \to 0$.\\\\
	Cuidado que es una condición necesaria pero no suficiente. Concluir la convergencia de la serie a partir de que $\lim a_n=0$ es un grave error.
\end{theorem}
\subsubsection{Serie armónica}
\begin{theorem}{Convergencia de serie-p.}
	\\Sea una serie $\sum _{n=1}^{\infty }\frac{1}{n^{p}} =\begin{cases}p >1 & \mathbb{C}\\p\le 1 & \mathbb{D}\end{cases}$
\end{theorem}
\newpage
\subsubsection{Serie telescópica}
Una serie telescópica es aquella serie cuyas sumas parciales poseen un numero fijo de términos tras su cancelación.\\
Es decir, sea $\sum a_n$ donde $a_n=b_{n+1}-b_n$ siendo $b_n$ otra sucesión.\\
Entonces $s_n=b_{n+1}-b_0$ y $\lim s_n=\lim b_{n+1}-b_0$.\\
Un ejemplo clásico es la serie telescópica de Mengoli, que se define por $\sum_{n=1}^{\infty} \frac{1}{n(n+1)}$, y puede calcularse según
$$\begin{aligned}\sum _{n=1}^{\infty }{\frac {1}{n(n+1)}}                                                                                                                                                                                                   & {}=\sum _{n=1}^{\infty }\left({\frac {1}{n}}-{\frac {1}{n+1}}\right)\\{} & {}
               =\lim _{N\to \infty }\sum _{n=1}^{N}\left({\frac {1}{n}}-{\frac {1}{n+1}}\right)\\{}                                                                                                                                 & {}
               =\lim _{N\to \infty }\left\lbrack {\left(1-{\frac {1}{2}}\right)+\left({\frac {1}{2}}-{\frac {1}{3}}\right)+\cdots +\left({\frac {1}{N}}-{\frac {1}{N+1}}\right)}\right\rbrack \\{}                                  & {}
               =\lim _{N\to \infty }\left\lbrack {1+\left(-{\frac {1}{2}}+{\frac {1}{2}}\right)+\left(-{\frac {1}{3}}+{\frac {1}{3}}\right)+\cdots +\left(-{\frac {1}{N}}+{\frac {1}{N}}\right)-{\frac {1}{N+1}}}\right\rbrack \\{} & {}
               =\lim _{N\to \infty }\left\lbrack {1-{\frac {1}{N+1}}}\right\rbrack =1.\end{aligned}$$
\subsubsection{Series de términos positivos.}
\begin{definition}
	Una serie $\sum a_n$ se dice de términos positivos, siempre que $a_n>0$ $\forall n \in \mathbb{N}$.
\end{definition}
Observar que en este caso, la sucesión de sumas parciales $s_n$ es monótona creciente, por lo tanto la serie $\sum a_n$ puede ser convergente o divergente, pero nunca oscilar.
\begin{theorem}{Criterio de comparación.}
	\\Sean $\sum a_n$ y $\sum b_n$ series de términos positivos, tales que $a_n\le b_n$ $\forall n>n_0$. Entonces:
	\begin{enumerate}
		\item Si $\sum b_n$ $\mathbb{C}$ entonces $\sum a_n$ $\mathbb{C}$.
		\item Si $\sum a_n$ $\mathbb{D}$ entonces $\sum b_n$ $\mathbb{D}$.
	\end{enumerate}
	En cualquier otro caso, no puedo afirmar nada.
\end{theorem}
\begin{theorem}{Criterio de equivalencia.}
	\\Sean $\sum a_n$ y $\sum b_n$ dos series de términos positivos.
	\begin{enumerate}
		\item Si $\lim \frac{a_n}{b_n}=L >0$ finito, entonces las dos series son de la misma clase.
		\item Si $\lim \frac{a_n}{b_n}=0$ y $\sum b_n$ $\mathbb{C}$, entonces $\sum a_n$ $\mathbb{C}$.
		\item Si $\lim \frac{a_n}{b_n}=\infty$ y $\sum b_n$ $\mathbb{D}$ entonces $\sum a_n$ $\mathbb{D}$.
	\end{enumerate}
	En cualquier otro caso, no puedo concluir.
\end{theorem}
\begin{theorem}{Criterio del cociente.}
	\\Sea $\sum a_n$ una serie de términos positivos, tal que existe $\lim_{n \to \infty} \frac{a_{n+1}}{a_n}=L$. Entonces:
	\begin{enumerate}
		\item Si $L<1\Rightarrow \sum a_n$ $\mathbb{C}$
		\item Si $L>1\Rightarrow \sum a_n$ $\mathbb{D}$
	\end{enumerate}
	En otro caso el criterio no decide.
\end{theorem}
\begin{theorem}{Criterio de Cauchy.}
	\\Sea $\sum a_n$ una serie de términos positivos, tal que existe $\lim_{n \to \infty} \sqrt[n]{a_n}=L$. Entonces:
	\begin{enumerate}
		\item Si $L<1\Rightarrow\sum a_n$ $\mathbb{C}$
		\item Si $L>1\Rightarrow\sum a_n$ $\mathbb{D}$
	\end{enumerate}
	En otro caso el criterio no decide.
\end{theorem}\newpage
\subsubsection{Series alternadas.}
\begin{definition}
	A una serie se le dice alternada si tiene sus términos alternativamente positivos y negativos. Su expresión general es de la forma $\sum_{n=1}^{\infty} \left(-1  \right)^n a_n$ y $a_n>0$.
\end{definition}
\begin{definition}
	Decimos que una serie $\sum a_n$ es absolutamente convergente si y solo si $\sum |a_n|$ es convergente.
\end{definition}
\begin{theorem}
	Toda serie absolutamente convergente es convergente.
\end{theorem}
\begin{theorem}{Convergencia dominada.}
	\\Sea $a_n$, $b_n$ y $c_n$ tal que $a_n<b_n<c_n$ $\forall n$.\\
	Si $\sum c_n$ $\mathbb{C}$ y $\sum a_n$ $\mathbb{C}$ entonces $\sum b_n$ $\mathbb{C}$. Ademas, vale que $\sum a_n \le \sum b_n \le \sum c_n$.
\end{theorem}
\begin{theorem}{Criterio de Leibnitz}
	\\Si $a_n$ es una sucesión estrictamente decreciente que tiende a cero, entonces la serie alternada $\sum \left( -1 \right)^n a_n$ es convergente.
\end{theorem}
\section{Integrales impropias}
\begin{definition}
	Consideramos $f:[a,+\infty)\to\mathbb{R}$ continua.\\
	Esto es que $f$ es continua en $[a,x]$ $\forall x>a$.\\
	Queremos darle sentido $\int_{a}^{+\infty} f(x) dx=\lim_{x \to \infty} \int_{a}^{x} f(t) dt$.
    $$\int_{a}^{+\infty} f(x) dx=\lim_{x \to \infty} \int_{a}^{x} f(x) dx =
	    \begin{cases}
		    L \text{ (finito)} & \to \int_{a}^{+\infty} f(x) dx \text{ } \mathbb{C} \text{ y } \int_{a}^{+\infty} f(x) dx=L \\
		    \infty             & \to \int_{a}^{+\infty} f(x) dx \text{ } \mathbb{D}                                         \\
		    \nexists           & \to \int_{a}^{+\infty} f(x) dx \text{ } \mathbb{O}
	    \end{cases}$$
\end{definition}
\begin{theorem}
	Sea $f:[a,+\infty)\to\mathbb{R}$ continua y $b>a$.\\
	Entonces $\int_{a}^{+\infty} f(x) dx$ y $\int_{b}^{+\infty} f(x) dx$ tienen igual comportamiento, y en caso de converger se cumple: $$\int_{a}^{+\infty} f(x) dx=\int_{a}^{b} f(x) dx+\int_{b}^{+\infty} f(x) dx$$
\end{theorem}
\begin{theorem}{Condición necesaria de convergencia.}
	\\Sea $f:[a,\infty)\to\mathbb{R}$ continua, si $\int_{a}^{+\infty} f(x) dx$ $\mathbb{C}$ y existe $\lim_{x \to \infty} f(x)$, entonces $\lim_{x \to \infty} f(x)=0$
\end{theorem}
\subsection{Integrales impropias de $1^{ra}$ especie.}
\begin{theorem}{Criterio Integral.}
	\\Sea $f:[n_0,+\infty)\to\mathbb{R}$ continua, monótona decreciente y no negativa.
	\\Defino $\forall n\ge n_0$ $a_n=f(n)$.
	\\\underline{Entonces}:	\begin{enumerate}
		\item $\sum_{n=n_0}^{+\infty} a_n$ $\mathbb{C} \Leftrightarrow \int_{n_0}^{+\infty} f(x) dx$ $\mathbb{C}$.
		\item $\sum_{n=n_0}^{+\infty} a_n$ $\mathbb{D} \Leftrightarrow \int_{n_0}^{+\infty} f(x) dx$ $\mathbb{D}$.
	\end{enumerate}
\end{theorem}
\begin{corollary}
    Si $\int_{a}^{\infty} f(t) dt$ y $\int_{a}^{\infty} g(t) dt$ convergen, entonces $\int_{a}^{\infty} \left( \alpha f(t)+\beta g(t) \right)  dt$ converge.
\end{corollary}\newpage
\begin{corollary}
	Si existe $\lim_{x \to +\infty} f(x)\neq 0$ entonces $\int_{a}^{+\infty} f(x) dx$ no converge.
\end{corollary}
\begin{theorem}{Criterio Serie-Integral.}
	\\Sean $f,g:[a,+\infty)\to\mathbb{R}$ continua tal que $0\le f(x)\le g(x)$ $\forall x\ge a$
	\begin{enumerate}
		\item Si $\int_{a}^{+\infty} f(x) dx$ $\mathbb{D}$ entonces $\int_{a}^{+\infty} g(x) dx$ $\mathbb{D}$
		\item Si $\int_{a}^{+\infty} g(x) dx$ $\mathbb{C}$ entonces $\int_{a}^{+\infty} f(x) dx$ $\mathbb{C}$
	\end{enumerate}
\end{theorem}
\begin{corollary}
    Sean $f$ y $g$ funciones con $f(t)\ge 0$, $g(t)\ge 0\ \forall t$, y $\lim_{x \to \infty} \frac{f(t)}{g(t)}=L>0$.\\
    Entonces $\int_{a}^{\infty} f(t) dt$ y $\int_{a}^{\infty} g(t) dt$ son de la misma clase.
\end{corollary}
\begin{definition}
    Decimos que la integral impropia $\int_{a}^{+\infty} f(x) dx$ es absolutamente convergente si y solo si $\int_{a}^{+\infty} |f(x)| dx$ es convergente.
\end{definition}
\begin{theorem}
    Si $\int_{a}^{+\infty} f(x) dx$ es absolutamente convergente, entonces es convergente.
\end{theorem}
\begin{definition}
	Sea $f:(-\infty ,b]$ continua:
	\\$\int_{-\infty}^{b} f(x) dx=\lim_{x \to -\infty} \int_{x}^{b} f(t) dt$
\end{definition}
\begin{definition}
	Sea $f:\mathbb{R}\to\mathbb{R}$ continua:
	$$\int_{-\infty}^{+\infty} f(x) dx=\int_{-\infty}^{0} f(x) dx+\int_{0}^{+\infty} f(x) dx$$
	Decimos que
	$$\int_{-\infty}^{+\infty} f(x) dx \text{ } \mathbb{C} \Leftrightarrow \begin{matrix} \int_{-\infty}^{0} f(x) dx \text{ } \mathbb{C} \\ \text{y} \\ \int_{0}^{+\infty} f(x) dx \text{ } \mathbb{C} \end{matrix} $$
\end{definition}
\subsection{Integrales impropias de $2^{da}$ especie}
\begin{definition}
    Sea $f:(a,b]\to\mathbb{R}$ continua, y $F(x)=\int_{x}^{b} f(t) dt$. Entonces si el $\lim_{x \to a^{+}}F(x)=L$ finito, decimos que la integral impropia $\int_{a}^{b} f(t) dt$ es convergente, y su valor es $L$. Si por el contrario el limite es infinito o no existe, decimos que la integral impropia diverge u oscila, respectivamente.
\end{definition}
Importante, recordemos el comportamiento de la integral impropia $\int \frac{1}{x^{\alpha}} dx$.
$$\int _{0}^{1}\frac{1}{x^{\alpha }} dx\ \begin{array}{ l }
\text{Converge si} \ \alpha < 1\\
\text{Diverge si} \ \alpha \geq 1
\end{array} \ \int _{1}^{\infty }\frac{1}{x^{\alpha }} \ \begin{array}{ l }
\text{Converge si} \ \alpha  >1\\
\text{Diverge si} \ \alpha \leq 1
\end{array}$$
\subsection{Integrales mixtas}
Cuando en una integral aparece mas de un punto problemático, debemos partir la integral en una suma de integrales que contengan solamente uno de esos puntos, y decimos que la integral original es convergente si y solo si cada uno de los sumandos lo es.\\
De esta forma por ejemplo si $f$ es continua, la integral impropia $\int_{-\infty}^{+\infty} f(x) dx$ debemos escribirla como suma de $\int_{-\infty}^{a} f(x) dx$ y $\int_{a}^{+\infty} f(x) dx$, y debemos clasificar estas dos integrales. Es fácil ver que el resultado no depende de $a$.
\end{document}
