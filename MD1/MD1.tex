\documentclass{report}

%%%%%%%%%%%%%%%%%%%%%%%%%%%%%%%%%
% PACKAGE IMPORTS
%%%%%%%%%%%%%%%%%%%%%%%%%%%%%%%%%


\usepackage[tmargin=2cm,rmargin=1in,lmargin=1in,margin=0.85in,bmargin=2cm,footskip=.2in]{geometry}
\usepackage{amsmath,amsfonts,amsthm,amssymb,mathtools}
\usepackage[spanish]{babel}
\usepackage[varbb]{newpxmath}
\usepackage{xfrac}
\usepackage[makeroom]{cancel}
\usepackage{mathtools}
\usepackage{bookmark}
\usepackage{enumitem}
\usepackage{hyperref,theoremref}
\hypersetup{
	pdftitle={Assignment},
	colorlinks=true, linkcolor=doc!90,
	bookmarksnumbered=true,
	bookmarksopen=true
}
\usepackage[most,many,breakable]{tcolorbox}
\usepackage{xcolor}
\usepackage{varwidth}
\usepackage{varwidth}
\usepackage{etoolbox}
%\usepackage{authblk}
\usepackage{nameref}
\usepackage{forest}
\usepackage{multicol,array}
\usepackage{tikz-cd}
\usepackage[ruled,vlined,linesnumbered]{algorithm2e}
\usepackage{comment} % enables the use of multi-line comments (\ifx \fi) 
\usepackage{import}
\usepackage{xifthen}
\usepackage{pdfpages}
\usepackage{transparent}

\newcommand\mycommfont[1]{\footnotesize\ttfamily\textcolor{blue}{#1}}
\SetCommentSty{mycommfont}
\newcommand{\incfig}[1]{%
    \def\svgwidth{\columnwidth}
    \import{./figures/}{#1.pdf_tex}
}

\usepackage{tikzsymbols}
\renewcommand\qedsymbol{$\Laughey$}


%\usepackage{import}
%\usepackage{xifthen}
%\usepackage{pdfpages}
%\usepackage{transparent}


%%%%%%%%%%%%%%%%%%%%%%%%%%%%%%
% SELF MADE COLORS
%%%%%%%%%%%%%%%%%%%%%%%%%%%%%%



\definecolor{myg}{RGB}{56, 140, 70}
\definecolor{myb}{RGB}{45, 111, 177}
\definecolor{myr}{RGB}{199, 68, 64}
\definecolor{mytheorembg}{HTML}{F2F2F9}
\definecolor{mytheoremfr}{HTML}{00007B}
\definecolor{mylenmabg}{HTML}{FFFAF8}
\definecolor{mylenmafr}{HTML}{983b0f}
\definecolor{mypropbg}{HTML}{f2fbfc}
\definecolor{mypropfr}{HTML}{191971}
\definecolor{myexamplebg}{HTML}{F2FBF8}
\definecolor{myexamplefr}{HTML}{88D6D1}
\definecolor{myexampleti}{HTML}{2A7F7F}
\definecolor{mydefinitbg}{HTML}{E5E5FF}
\definecolor{mydefinitfr}{HTML}{3F3FA3}
\definecolor{notesgreen}{RGB}{0,162,0}
\definecolor{myp}{RGB}{197, 92, 212}
\definecolor{mygr}{HTML}{2C3338}
\definecolor{myred}{RGB}{127,0,0}
\definecolor{myyellow}{RGB}{169,121,69}
\definecolor{myexercisebg}{HTML}{F2FBF8}
\definecolor{myexercisefg}{HTML}{88D6D1}


%%%%%%%%%%%%%%%%%%%%%%%%%%%%
% TCOLORBOX SETUPS
%%%%%%%%%%%%%%%%%%%%%%%%%%%%

\setlength{\parindent}{1cm}
%================================
% THEOREM BOX
%================================

\tcbuselibrary{theorems,skins,hooks}
\newtcbtheorem[number within=section]{Theorem}{Teorema}
{%
	enhanced,
	breakable,
	colback = mytheorembg,
	frame hidden,
	boxrule = 0sp,
	borderline west = {2pt}{0pt}{mytheoremfr},
	sharp corners,
	detach title,
	before upper = \tcbtitle\par\smallskip,
	coltitle = mytheoremfr,
	fonttitle = \bfseries\sffamily,
	description font = \mdseries,
	separator sign none,
	segmentation style={solid, mytheoremfr},
}
{th}

\tcbuselibrary{theorems,skins,hooks}
\newtcbtheorem[number within=chapter]{theorem}{Teorema}
{%
	enhanced,
	breakable,
	colback = mytheorembg,
	frame hidden,
	boxrule = 0sp,
	borderline west = {2pt}{0pt}{mytheoremfr},
	sharp corners,
	detach title,
	before upper = \tcbtitle\par\smallskip,
	coltitle = mytheoremfr,
	fonttitle = \bfseries\sffamily,
	description font = \mdseries,
	separator sign none,
	segmentation style={solid, mytheoremfr},
}
{th}


\tcbuselibrary{theorems,skins,hooks}
\newtcolorbox{Theoremcon}
{%
	enhanced
	,breakable
	,colback = mytheorembg
	,frame hidden
	,boxrule = 0sp
	,borderline west = {2pt}{0pt}{mytheoremfr}
	,sharp corners
	,description font = \mdseries
	,separator sign none
}

%================================
% Corollery
%================================
\tcbuselibrary{theorems,skins,hooks}
\newtcbtheorem[number within=section]{Corollary}{Corolario}
{%
	enhanced
	,breakable
	,colback = myp!10
	,frame hidden
	,boxrule = 0sp
	,borderline west = {2pt}{0pt}{myp!85!black}
	,sharp corners
	,detach title
	,before upper = \tcbtitle\par\smallskip
	,coltitle = myp!85!black
	,fonttitle = \bfseries\sffamily
	,description font = \mdseries
	,separator sign none
	,segmentation style={solid, myp!85!black}
}
{th}
\tcbuselibrary{theorems,skins,hooks}
\newtcbtheorem[number within=chapter]{corollary}{Corolario}
{%
	enhanced
	,breakable
	,colback = myp!10
	,frame hidden
	,boxrule = 0sp
	,borderline west = {2pt}{0pt}{myp!85!black}
	,sharp corners
	,detach title
	,before upper = \tcbtitle\par\smallskip
	,coltitle = myp!85!black
	,fonttitle = \bfseries\sffamily
	,description font = \mdseries
	,separator sign none
	,segmentation style={solid, myp!85!black}
}
{th}


%================================
% LENMA
%================================

\tcbuselibrary{theorems,skins,hooks}
\newtcbtheorem[number within=section]{Lenma}{Lenma}
{%
	enhanced,
	breakable,
	colback = mylenmabg,
	frame hidden,
	boxrule = 0sp,
	borderline west = {2pt}{0pt}{mylenmafr},
	sharp corners,
	detach title,
	before upper = \tcbtitle\par\smallskip,
	coltitle = mylenmafr,
	fonttitle = \bfseries\sffamily,
	description font = \mdseries,
	separator sign none,
	segmentation style={solid, mylenmafr},
}
{th}

\tcbuselibrary{theorems,skins,hooks}
\newtcbtheorem[number within=chapter]{lenma}{Lenma}
{%
	enhanced,
	breakable,
	colback = mylenmabg,
	frame hidden,
	boxrule = 0sp,
	borderline west = {2pt}{0pt}{mylenmafr},
	sharp corners,
	detach title,
	before upper = \tcbtitle\par\smallskip,
	coltitle = mylenmafr,
	fonttitle = \bfseries\sffamily,
	description font = \mdseries,
	separator sign none,
	segmentation style={solid, mylenmafr},
}
{th}


%================================
% PROPOSITION
%================================

\tcbuselibrary{theorems,skins,hooks}
\newtcbtheorem[number within=section]{Prop}{Proposición}
{%
	enhanced,
	breakable,
	colback = mypropbg,
	frame hidden,
	boxrule = 0sp,
	borderline west = {2pt}{0pt}{mypropfr},
	sharp corners,
	detach title,
	before upper = \tcbtitle\par\smallskip,
	coltitle = mypropfr,
	fonttitle = \bfseries\sffamily,
	description font = \mdseries,
	separator sign none,
	segmentation style={solid, mypropfr},
}
{th}

\tcbuselibrary{theorems,skins,hooks}
\newtcbtheorem[number within=chapter]{prop}{Proposición}
{%
	enhanced,
	breakable,
	colback = mypropbg,
	frame hidden,
	boxrule = 0sp,
	borderline west = {2pt}{0pt}{mypropfr},
	sharp corners,
	detach title,
	before upper = \tcbtitle\par\smallskip,
	coltitle = mypropfr,
	fonttitle = \bfseries\sffamily,
	description font = \mdseries,
	separator sign none,
	segmentation style={solid, mypropfr},
}
{th}


%================================
% CLAIM
%================================

\tcbuselibrary{theorems,skins,hooks}
\newtcbtheorem[number within=section]{claim}{Claim}
{%
	enhanced
	,breakable
	,colback = myg!10
	,frame hidden
	,boxrule = 0sp
	,borderline west = {2pt}{0pt}{myg}
	,sharp corners
	,detach title
	,before upper = \tcbtitle\par\smallskip
	,coltitle = myg!85!black
	,fonttitle = \bfseries\sffamily
	,description font = \mdseries
	,separator sign none
	,segmentation style={solid, myg!85!black}
}
{th}



%================================
% Exercise
%================================

\tcbuselibrary{theorems,skins,hooks}
\newtcbtheorem[number within=section]{Exercise}{Exercise}
{%
	enhanced,
	breakable,
	colback = myexercisebg,
	frame hidden,
	boxrule = 0sp,
	borderline west = {2pt}{0pt}{myexercisefg},
	sharp corners,
	detach title,
	before upper = \tcbtitle\par\smallskip,
	coltitle = myexercisefg,
	fonttitle = \bfseries\sffamily,
	description font = \mdseries,
	separator sign none,
	segmentation style={solid, myexercisefg},
}
{th}

\tcbuselibrary{theorems,skins,hooks}
\newtcbtheorem[number within=chapter]{exercise}{Exercise}
{%
	enhanced,
	breakable,
	colback = myexercisebg,
	frame hidden,
	boxrule = 0sp,
	borderline west = {2pt}{0pt}{myexercisefg},
	sharp corners,
	detach title,
	before upper = \tcbtitle\par\smallskip,
	coltitle = myexercisefg,
	fonttitle = \bfseries\sffamily,
	description font = \mdseries,
	separator sign none,
	segmentation style={solid, myexercisefg},
}
{th}

%================================
% EXAMPLE BOX
%================================

\newtcbtheorem[number within=section]{Example}{Ejemplo}
{%
	colback = myexamplebg
	,breakable
	,colframe = myexamplefr
	,coltitle = myexampleti
	,boxrule = 1pt
	,sharp corners
	,detach title
	,before upper=\tcbtitle\par\smallskip
	,fonttitle = \bfseries
	,description font = \mdseries
	,separator sign none
	,description delimiters parenthesis
}
{ex}

\newtcbtheorem[number within=chapter]{example}{Ejemplo}
{%
	colback = myexamplebg
	,breakable
	,colframe = myexamplefr
	,coltitle = myexampleti
	,boxrule = 1pt
	,sharp corners
	,detach title
	,before upper=\tcbtitle\par\smallskip
	,fonttitle = \bfseries
	,description font = \mdseries
	,separator sign none
	,description delimiters parenthesis
}
{ex}

%================================
% DEFINITION BOX
%================================

\newtcbtheorem[number within=section]{Definition}{Definición}{enhanced,
	before skip=2mm,after skip=2mm, colback=red!5,colframe=red!80!black,boxrule=0.5mm,
	attach boxed title to top left={xshift=1cm,yshift*=1mm-\tcboxedtitleheight}, varwidth boxed title*=-3cm,
	boxed title style={frame code={
					\path[fill=tcbcolback]
					([yshift=-1mm,xshift=-1mm]frame.north west)
					arc[start angle=0,end angle=180,radius=1mm]
					([yshift=-1mm,xshift=1mm]frame.north east)
					arc[start angle=180,end angle=0,radius=1mm];
					\path[left color=tcbcolback!60!black,right color=tcbcolback!60!black,
						middle color=tcbcolback!80!black]
					([xshift=-2mm]frame.north west) -- ([xshift=2mm]frame.north east)
					[rounded corners=1mm]-- ([xshift=1mm,yshift=-1mm]frame.north east)
					-- (frame.south east) -- (frame.south west)
					-- ([xshift=-1mm,yshift=-1mm]frame.north west)
					[sharp corners]-- cycle;
				},interior engine=empty,
		},
	fonttitle=\bfseries,
	title={#2},#1}{def}
\newtcbtheorem[number within=chapter]{definition}{Definición}{enhanced,
	before skip=2mm,after skip=2mm, colback=red!5,colframe=red!80!black,boxrule=0.5mm,
	attach boxed title to top left={xshift=1cm,yshift*=1mm-\tcboxedtitleheight}, varwidth boxed title*=-3cm,
	boxed title style={frame code={
					\path[fill=tcbcolback]
					([yshift=-1mm,xshift=-1mm]frame.north west)
					arc[start angle=0,end angle=180,radius=1mm]
					([yshift=-1mm,xshift=1mm]frame.north east)
					arc[start angle=180,end angle=0,radius=1mm];
					\path[left color=tcbcolback!60!black,right color=tcbcolback!60!black,
						middle color=tcbcolback!80!black]
					([xshift=-2mm]frame.north west) -- ([xshift=2mm]frame.north east)
					[rounded corners=1mm]-- ([xshift=1mm,yshift=-1mm]frame.north east)
					-- (frame.south east) -- (frame.south west)
					-- ([xshift=-1mm,yshift=-1mm]frame.north west)
					[sharp corners]-- cycle;
				},interior engine=empty,
		},
	fonttitle=\bfseries,
	title={#2},#1}{def}



%================================
% Solution BOX
%================================

\makeatletter
\newtcbtheorem{question}{Question}{enhanced,
	breakable,
	colback=white,
	colframe=myb!80!black,
	attach boxed title to top left={yshift*=-\tcboxedtitleheight},
	fonttitle=\bfseries,
	title={#2},
	boxed title size=title,
	boxed title style={%
			sharp corners,
			rounded corners=northwest,
			colback=tcbcolframe,
			boxrule=0pt,
		},
	underlay boxed title={%
			\path[fill=tcbcolframe] (title.south west)--(title.south east)
			to[out=0, in=180] ([xshift=5mm]title.east)--
			(title.center-|frame.east)
			[rounded corners=\kvtcb@arc] |-
			(frame.north) -| cycle;
		},
	#1
}{def}
\makeatother

%================================
% SOLUTION BOX
%================================

\makeatletter
\newtcolorbox{solution}{enhanced,
	breakable,
	colback=white,
	colframe=myg!80!black,
	attach boxed title to top left={yshift*=-\tcboxedtitleheight},
	title=Solution,
	boxed title size=title,
	boxed title style={%
			sharp corners,
			rounded corners=northwest,
			colback=tcbcolframe,
			boxrule=0pt,
		},
	underlay boxed title={%
			\path[fill=tcbcolframe] (title.south west)--(title.south east)
			to[out=0, in=180] ([xshift=5mm]title.east)--
			(title.center-|frame.east)
			[rounded corners=\kvtcb@arc] |-
			(frame.north) -| cycle;
		},
}
\makeatother

%================================
% Question BOX
%================================

\makeatletter
\newtcbtheorem{qstion}{Question}{enhanced,
	breakable,
	colback=white,
	colframe=mygr,
	attach boxed title to top left={yshift*=-\tcboxedtitleheight},
	fonttitle=\bfseries,
	title={#2},
	boxed title size=title,
	boxed title style={%
			sharp corners,
			rounded corners=northwest,
			colback=tcbcolframe,
			boxrule=0pt,
		},
	underlay boxed title={%
			\path[fill=tcbcolframe] (title.south west)--(title.south east)
			to[out=0, in=180] ([xshift=5mm]title.east)--
			(title.center-|frame.east)
			[rounded corners=\kvtcb@arc] |-
			(frame.north) -| cycle;
		},
	#1
}{def}
\makeatother

\newtcbtheorem[number within=chapter]{wconc}{Wrong Concept}{
	breakable,
	enhanced,
	colback=white,
	colframe=myr,
	arc=0pt,
	outer arc=0pt,
	fonttitle=\bfseries\sffamily\large,
	colbacktitle=myr,
	attach boxed title to top left={},
	boxed title style={
			enhanced,
			skin=enhancedfirst jigsaw,
			arc=3pt,
			bottom=0pt,
			interior style={fill=myr}
		},
	#1
}{def}



%================================
% NOTE BOX
%================================

\usetikzlibrary{arrows,calc,shadows.blur}
\tcbuselibrary{skins}
\newtcolorbox{note}[1][]{%
	enhanced jigsaw,
	colback=gray!20!white,%
	colframe=gray!80!black,
	size=small,
	boxrule=1pt,
	title=\textbf{Nota:-},
	halign title=flush center,
	coltitle=black,
	breakable,
	drop shadow=black!50!white,
	attach boxed title to top left={xshift=1cm,yshift=-\tcboxedtitleheight/2,yshifttext=-\tcboxedtitleheight/2},
	minipage boxed title=1.5cm,
	boxed title style={%
			colback=white,
			size=fbox,
			boxrule=1pt,
			boxsep=2pt,
			underlay={%
					\coordinate (dotA) at ($(interior.west) + (-0.5pt,0)$);
					\coordinate (dotB) at ($(interior.east) + (0.5pt,0)$);
					\begin{scope}
						\clip (interior.north west) rectangle ([xshift=3ex]interior.east);
						\filldraw [white, blur shadow={shadow opacity=60, shadow yshift=-.75ex}, rounded corners=2pt] (interior.north west) rectangle (interior.south east);
					\end{scope}
					\begin{scope}[gray!80!black]
						\fill (dotA) circle (2pt);
						\fill (dotB) circle (2pt);
					\end{scope}
				},
		},
	#1,
}

%%%%%%%%%%%%%%%%%%%%%%%%%%%%%%
% SELF MADE COMMANDS
%%%%%%%%%%%%%%%%%%%%%%%%%%%%%%


\newcommand{\thm}[2]{\begin{Theorem}{#1}{}#2\end{Theorem}}
\newcommand{\cor}[2]{\begin{Corollary}{#1}{}#2\end{Corollary}}
\newcommand{\mlenma}[2]{\begin{Lenma}{#1}{}#2\end{Lenma}}
\newcommand{\mprop}[2]{\begin{Prop}{#1}{}#2\end{Prop}}
\newcommand{\clm}[3]{\begin{claim}{#1}{#2}#3\end{claim}}
\newcommand{\wc}[2]{\begin{wconc}{#1}{}\setlength{\parindent}{1cm}#2\end{wconc}}
\newcommand{\thmcon}[1]{\begin{Theoremcon}{#1}\end{Theoremcon}}
\newcommand{\ex}[2]{\begin{Example}{#1}{}#2\end{Example}}
\newcommand{\dfn}[2]{\begin{Definition}[colbacktitle=red!75!black]{#1}{}#2\end{Definition}}
\newcommand{\dfnc}[2]{\begin{definition}[colbacktitle=red!75!black]{#1}{}#2\end{definition}}
\newcommand{\qs}[2]{\begin{question}{#1}{}#2\end{question}}
\newcommand{\pf}[2]{\begin{myproof}[#1]#2\end{myproof}}
\newcommand{\nt}[1]{\begin{note}#1\end{note}}

\newcommand*\circled[1]{\tikz[baseline=(char.base)]{
		\node[shape=circle,draw,inner sep=1pt] (char) {#1};}}
\newcommand\getcurrentref[1]{%
	\ifnumequal{\value{#1}}{0}
	{??}
	{\the\value{#1}}%
}
\newcommand{\getCurrentSectionNumber}{\getcurrentref{section}}
\newenvironment{myproof}[1][\proofname]{%
	\proof[\bfseries #1: ]%
}{\endproof}

\newcommand{\mclm}[2]{\begin{myclaim}[#1]#2\end{myclaim}}
\newenvironment{myclaim}[1][\claimname]{\proof[\bfseries #1: ]}{}

\newcounter{mylabelcounter}

\makeatletter
\newcommand{\setword}[2]{%
	\phantomsection
	#1\def\@currentlabel{\unexpanded{#1}}\label{#2}%
}
\makeatother




\tikzset{
	symbol/.style={
			draw=none,
			every to/.append style={
					edge node={node [sloped, allow upside down, auto=false]{$#1$}}}
		}
}


% deliminators
\DeclarePairedDelimiter{\abs}{\lvert}{\rvert}
\DeclarePairedDelimiter{\norm}{\lVert}{\rVert}

\DeclarePairedDelimiter{\ceil}{\lceil}{\rceil}
\DeclarePairedDelimiter{\floor}{\lfloor}{\rfloor}
\DeclarePairedDelimiter{\round}{\lfloor}{\rceil}

\newsavebox\diffdbox
\newcommand{\slantedromand}{{\mathpalette\makesl{d}}}
\newcommand{\makesl}[2]{%
\begingroup
\sbox{\diffdbox}{$\mathsurround=0pt#1\mathrm{#2}$}%
\pdfsave
\pdfsetmatrix{1 0 0.2 1}%
\rlap{\usebox{\diffdbox}}%
\pdfrestore
\hskip\wd\diffdbox
\endgroup
}
\newcommand{\dd}[1][]{\ensuremath{\mathop{}\!\ifstrempty{#1}{%
\slantedromand\@ifnextchar^{\hspace{0.2ex}}{\hspace{0.1ex}}}%
{\slantedromand\hspace{0.2ex}^{#1}}}}
\ProvideDocumentCommand\dv{o m g}{%
  \ensuremath{%
    \IfValueTF{#3}{%
      \IfNoValueTF{#1}{%
        \frac{\dd #2}{\dd #3}%
      }{%
        \frac{\dd^{#1} #2}{\dd #3^{#1}}%
      }%
    }{%
      \IfNoValueTF{#1}{%
        \frac{\dd}{\dd #2}%
      }{%
        \frac{\dd^{#1}}{\dd #2^{#1}}%
      }%
    }%
  }%
}
\providecommand*{\pdv}[3][]{\frac{\partial^{#1}#2}{\partial#3^{#1}}}
%  - others
\DeclareMathOperator{\Lap}{\mathcal{L}}
\DeclareMathOperator{\Var}{Var} % varience
\DeclareMathOperator{\Cov}{Cov} % covarience
\DeclareMathOperator{\E}{E} % expected

% Since the amsthm package isn't loaded

% I prefer the slanted \leq
\let\oldleq\leq % save them in case they're every wanted
\let\oldgeq\geq
\renewcommand{\leq}{\leqslant}
\renewcommand{\geq}{\geqslant}

% % redefine matrix env to allow for alignment, use r as default
% \renewcommand*\env@matrix[1][r]{\hskip -\arraycolsep
%     \let\@ifnextchar\new@ifnextchar
%     \array{*\c@MaxMatrixCols #1}}


%\usepackage{framed}
%\usepackage{titletoc}
%\usepackage{etoolbox}
%\usepackage{lmodern}


%\patchcmd{\tableofcontents}{\contentsname}{\sffamily\contentsname}{}{}

%\renewenvironment{leftbar}
%{\def\FrameCommand{\hspace{6em}%
%		{\color{myyellow}\vrule width 2pt depth 6pt}\hspace{1em}}%
%	\MakeFramed{\parshape 1 0cm \dimexpr\textwidth-6em\relax\FrameRestore}\vskip2pt%
%}
%{\endMakeFramed}

%\titlecontents{chapter}
%[0em]{\vspace*{2\baselineskip}}
%{\parbox{4.5em}{%
%		\hfill\Huge\sffamily\bfseries\color{myred}\thecontentspage}%
%	\vspace*{-2.3\baselineskip}\leftbar\textsc{\small\chaptername~\thecontentslabel}\\\sffamily}
%{}{\endleftbar}
%\titlecontents{section}
%[8.4em]
%{\sffamily\contentslabel{3em}}{}{}
%{\hspace{0.5em}\nobreak\itshape\color{myred}\contentspage}
%\titlecontents{subsection}
%[8.4em]
%{\sffamily\contentslabel{3em}}{}{}  
%{\hspace{0.5em}\nobreak\itshape\color{myred}\contentspage}



%%%%%%%%%%%%%%%%%%%%%%%%%%%%%%%%%%%%%%%%%%%
% TABLE OF CONTENTS
%%%%%%%%%%%%%%%%%%%%%%%%%%%%%%%%%%%%%%%%%%%

\usepackage{tikz}
\definecolor{doc}{RGB}{0,60,110}
\usepackage{titletoc}
\contentsmargin{0cm}
\titlecontents{chapter}[3.7pc]
{\addvspace{30pt}%
	\begin{tikzpicture}[remember picture, overlay]%
		\draw[fill=doc!60,draw=doc!60] (-7,-.1) rectangle (-0.9,.5);%
		\pgftext[left,x=-3.5cm,y=0.2cm]{\color{white}\Large\sc\bfseries Chapter\ \thecontentslabel};%
	\end{tikzpicture}\color{doc!60}\large\sc\bfseries}%
{}
{}
{\;\titlerule\;\large\sc\bfseries Page \thecontentspage
	\begin{tikzpicture}[remember picture, overlay]
		\draw[fill=doc!60,draw=doc!60] (2pt,0) rectangle (4,0.1pt);
	\end{tikzpicture}}%
\titlecontents{section}[3.7pc]
{\addvspace{2pt}}
{\contentslabel[\thecontentslabel]{2pc}}
{}
{\hfill\small \thecontentspage}
[]
\titlecontents*{subsection}[3.7pc]
{\addvspace{-1pt}\small}
{}
{}
{\ --- \small\thecontentspage}
[ \textbullet\ ][]

\makeatletter
\renewcommand{\tableofcontents}{%
	\chapter*{%
	  \vspace*{-20\p@}%
	  \begin{tikzpicture}[remember picture, overlay]%
		  \pgftext[right,x=15cm,y=0.2cm]{\color{doc!60}\Huge\sc\bfseries \contentsname};%
		  \draw[fill=doc!60,draw=doc!60] (13,-.75) rectangle (20,1);%
		  \clip (13,-.75) rectangle (20,1);
		  \pgftext[right,x=15cm,y=0.2cm]{\color{white}\Huge\sc\bfseries \contentsname};%
	  \end{tikzpicture}}%
	\@starttoc{toc}}
\makeatother


% Things Lie
\newcommand{\kb}{\mathfrak b}
\newcommand{\kg}{\mathfrak g}
\newcommand{\kh}{\mathfrak h}
\newcommand{\kn}{\mathfrak n}
\newcommand{\ku}{\mathfrak u}
\newcommand{\kz}{\mathfrak z}
\DeclareMathOperator{\Ext}{Ext} % Ext functor
\DeclareMathOperator{\Tor}{Tor} % Tor functor
\newcommand{\gl}{\opname{\mathfrak{gl}}} % frak gl group
\renewcommand{\sl}{\opname{\mathfrak{sl}}} % frak sl group chktex 6

% More script letters etc.
\newcommand{\SA}{\mathcal A}
\newcommand{\SB}{\mathcal B}
\newcommand{\SC}{\mathcal C}
\newcommand{\SF}{\mathcal F}
\newcommand{\SG}{\mathcal G}
\newcommand{\SH}{\mathcal H}
\newcommand{\OO}{\mathcal O}

\newcommand{\SCA}{\mathscr A}
\newcommand{\SCB}{\mathscr B}
\newcommand{\SCC}{\mathscr C}
\newcommand{\SCD}{\mathscr D}
\newcommand{\SCE}{\mathscr E}
\newcommand{\SCF}{\mathscr F}
\newcommand{\SCG}{\mathscr G}
\newcommand{\SCH}{\mathscr H}

% Mathfrak primes
\newcommand{\km}{\mathfrak m}
\newcommand{\kp}{\mathfrak p}
\newcommand{\kq}{\mathfrak q}

% number sets
\newcommand{\RR}[1][]{\ensuremath{\ifstrempty{#1}{\mathbb{R}}{\mathbb{R}^{#1}}}}
\newcommand{\NN}[1][]{\ensuremath{\ifstrempty{#1}{\mathbb{N}}{\mathbb{N}^{#1}}}}
\newcommand{\ZZ}[1][]{\ensuremath{\ifstrempty{#1}{\mathbb{Z}}{\mathbb{Z}^{#1}}}}
\newcommand{\QQ}[1][]{\ensuremath{\ifstrempty{#1}{\mathbb{Q}}{\mathbb{Q}^{#1}}}}
\newcommand{\CC}[1][]{\ensuremath{\ifstrempty{#1}{\mathbb{C}}{\mathbb{C}^{#1}}}}
\newcommand{\PP}[1][]{\ensuremath{\ifstrempty{#1}{\mathbb{P}}{\mathbb{P}^{#1}}}}
\newcommand{\HH}[1][]{\ensuremath{\ifstrempty{#1}{\mathbb{H}}{\mathbb{H}^{#1}}}}
\newcommand{\FF}[1][]{\ensuremath{\ifstrempty{#1}{\mathbb{F}}{\mathbb{F}^{#1}}}}
% expected value
\newcommand{\EE}{\ensuremath{\mathbb{E}}}
\newcommand{\charin}{\text{ char }}
\DeclareMathOperator{\sign}{sign}
\DeclareMathOperator{\Aut}{Aut}
\DeclareMathOperator{\Inn}{Inn}
\DeclareMathOperator{\Syl}{Syl}
\DeclareMathOperator{\Gal}{Gal}
\DeclareMathOperator{\GL}{GL} % General linear group
\DeclareMathOperator{\SL}{SL} % Special linear group

%---------------------------------------
% BlackBoard Math Fonts :-
%---------------------------------------

%Captital Letters
\newcommand{\bbA}{\mathbb{A}}	\newcommand{\bbB}{\mathbb{B}}
\newcommand{\bbC}{\mathbb{C}}	\newcommand{\bbD}{\mathbb{D}}
\newcommand{\bbE}{\mathbb{E}}	\newcommand{\bbF}{\mathbb{F}}
\newcommand{\bbG}{\mathbb{G}}	\newcommand{\bbH}{\mathbb{H}}
\newcommand{\bbI}{\mathbb{I}}	\newcommand{\bbJ}{\mathbb{J}}
\newcommand{\bbK}{\mathbb{K}}	\newcommand{\bbL}{\mathbb{L}}
\newcommand{\bbM}{\mathbb{M}}	\newcommand{\bbN}{\mathbb{N}}
\newcommand{\bbO}{\mathbb{O}}	\newcommand{\bbP}{\mathbb{P}}
\newcommand{\bbQ}{\mathbb{Q}}	\newcommand{\bbR}{\mathbb{R}}
\newcommand{\bbS}{\mathbb{S}}	\newcommand{\bbT}{\mathbb{T}}
\newcommand{\bbU}{\mathbb{U}}	\newcommand{\bbV}{\mathbb{V}}
\newcommand{\bbW}{\mathbb{W}}	\newcommand{\bbX}{\mathbb{X}}
\newcommand{\bbY}{\mathbb{Y}}	\newcommand{\bbZ}{\mathbb{Z}}

%---------------------------------------
% MathCal Fonts :-
%---------------------------------------

%Captital Letters
\newcommand{\mcA}{\mathcal{A}}	\newcommand{\mcB}{\mathcal{B}}
\newcommand{\mcC}{\mathcal{C}}	\newcommand{\mcD}{\mathcal{D}}
\newcommand{\mcE}{\mathcal{E}}	\newcommand{\mcF}{\mathcal{F}}
\newcommand{\mcG}{\mathcal{G}}	\newcommand{\mcH}{\mathcal{H}}
\newcommand{\mcI}{\mathcal{I}}	\newcommand{\mcJ}{\mathcal{J}}
\newcommand{\mcK}{\mathcal{K}}	\newcommand{\mcL}{\mathcal{L}}
\newcommand{\mcM}{\mathcal{M}}	\newcommand{\mcN}{\mathcal{N}}
\newcommand{\mcO}{\mathcal{O}}	\newcommand{\mcP}{\mathcal{P}}
\newcommand{\mcQ}{\mathcal{Q}}	\newcommand{\mcR}{\mathcal{R}}
\newcommand{\mcS}{\mathcal{S}}	\newcommand{\mcT}{\mathcal{T}}
\newcommand{\mcU}{\mathcal{U}}	\newcommand{\mcV}{\mathcal{V}}
\newcommand{\mcW}{\mathcal{W}}	\newcommand{\mcX}{\mathcal{X}}
\newcommand{\mcY}{\mathcal{Y}}	\newcommand{\mcZ}{\mathcal{Z}}


%---------------------------------------
% Bold Math Fonts :-
%---------------------------------------

%Captital Letters
\newcommand{\bmA}{\boldsymbol{A}}	\newcommand{\bmB}{\boldsymbol{B}}
\newcommand{\bmC}{\boldsymbol{C}}	\newcommand{\bmD}{\boldsymbol{D}}
\newcommand{\bmE}{\boldsymbol{E}}	\newcommand{\bmF}{\boldsymbol{F}}
\newcommand{\bmG}{\boldsymbol{G}}	\newcommand{\bmH}{\boldsymbol{H}}
\newcommand{\bmI}{\boldsymbol{I}}	\newcommand{\bmJ}{\boldsymbol{J}}
\newcommand{\bmK}{\boldsymbol{K}}	\newcommand{\bmL}{\boldsymbol{L}}
\newcommand{\bmM}{\boldsymbol{M}}	\newcommand{\bmN}{\boldsymbol{N}}
\newcommand{\bmO}{\boldsymbol{O}}	\newcommand{\bmP}{\boldsymbol{P}}
\newcommand{\bmQ}{\boldsymbol{Q}}	\newcommand{\bmR}{\boldsymbol{R}}
\newcommand{\bmS}{\boldsymbol{S}}	\newcommand{\bmT}{\boldsymbol{T}}
\newcommand{\bmU}{\boldsymbol{U}}	\newcommand{\bmV}{\boldsymbol{V}}
\newcommand{\bmW}{\boldsymbol{W}}	\newcommand{\bmX}{\boldsymbol{X}}
\newcommand{\bmY}{\boldsymbol{Y}}	\newcommand{\bmZ}{\boldsymbol{Z}}
%Small Letters
\newcommand{\bma}{\boldsymbol{a}}	\newcommand{\bmb}{\boldsymbol{b}}
\newcommand{\bmc}{\boldsymbol{c}}	\newcommand{\bmd}{\boldsymbol{d}}
\newcommand{\bme}{\boldsymbol{e}}	\newcommand{\bmf}{\boldsymbol{f}}
\newcommand{\bmg}{\boldsymbol{g}}	\newcommand{\bmh}{\boldsymbol{h}}
\newcommand{\bmi}{\boldsymbol{i}}	\newcommand{\bmj}{\boldsymbol{j}}
\newcommand{\bmk}{\boldsymbol{k}}	\newcommand{\bml}{\boldsymbol{l}}
\newcommand{\bmm}{\boldsymbol{m}}	\newcommand{\bmn}{\boldsymbol{n}}
\newcommand{\bmo}{\boldsymbol{o}}	\newcommand{\bmp}{\boldsymbol{p}}
\newcommand{\bmq}{\boldsymbol{q}}	\newcommand{\bmr}{\boldsymbol{r}}
\newcommand{\bms}{\boldsymbol{s}}	\newcommand{\bmt}{\boldsymbol{t}}
\newcommand{\bmu}{\boldsymbol{u}}	\newcommand{\bmv}{\boldsymbol{v}}
\newcommand{\bmw}{\boldsymbol{w}}	\newcommand{\bmx}{\boldsymbol{x}}
\newcommand{\bmy}{\boldsymbol{y}}	\newcommand{\bmz}{\boldsymbol{z}}

%---------------------------------------
% Scr Math Fonts :-
%---------------------------------------

\newcommand{\sA}{{\mathscr{A}}}   \newcommand{\sB}{{\mathscr{B}}}
\newcommand{\sC}{{\mathscr{C}}}   \newcommand{\sD}{{\mathscr{D}}}
\newcommand{\sE}{{\mathscr{E}}}   \newcommand{\sF}{{\mathscr{F}}}
\newcommand{\sG}{{\mathscr{G}}}   \newcommand{\sH}{{\mathscr{H}}}
\newcommand{\sI}{{\mathscr{I}}}   \newcommand{\sJ}{{\mathscr{J}}}
\newcommand{\sK}{{\mathscr{K}}}   \newcommand{\sL}{{\mathscr{L}}}
\newcommand{\sM}{{\mathscr{M}}}   \newcommand{\sN}{{\mathscr{N}}}
\newcommand{\sO}{{\mathscr{O}}}   \newcommand{\sP}{{\mathscr{P}}}
\newcommand{\sQ}{{\mathscr{Q}}}   \newcommand{\sR}{{\mathscr{R}}}
\newcommand{\sS}{{\mathscr{S}}}   \newcommand{\sT}{{\mathscr{T}}}
\newcommand{\sU}{{\mathscr{U}}}   \newcommand{\sV}{{\mathscr{V}}}
\newcommand{\sW}{{\mathscr{W}}}   \newcommand{\sX}{{\mathscr{X}}}
\newcommand{\sY}{{\mathscr{Y}}}   \newcommand{\sZ}{{\mathscr{Z}}}


%---------------------------------------
% Math Fraktur Font
%---------------------------------------

%Captital Letters
\newcommand{\mfA}{\mathfrak{A}}	\newcommand{\mfB}{\mathfrak{B}}
\newcommand{\mfC}{\mathfrak{C}}	\newcommand{\mfD}{\mathfrak{D}}
\newcommand{\mfE}{\mathfrak{E}}	\newcommand{\mfF}{\mathfrak{F}}
\newcommand{\mfG}{\mathfrak{G}}	\newcommand{\mfH}{\mathfrak{H}}
\newcommand{\mfI}{\mathfrak{I}}	\newcommand{\mfJ}{\mathfrak{J}}
\newcommand{\mfK}{\mathfrak{K}}	\newcommand{\mfL}{\mathfrak{L}}
\newcommand{\mfM}{\mathfrak{M}}	\newcommand{\mfN}{\mathfrak{N}}
\newcommand{\mfO}{\mathfrak{O}}	\newcommand{\mfP}{\mathfrak{P}}
\newcommand{\mfQ}{\mathfrak{Q}}	\newcommand{\mfR}{\mathfrak{R}}
\newcommand{\mfS}{\mathfrak{S}}	\newcommand{\mfT}{\mathfrak{T}}
\newcommand{\mfU}{\mathfrak{U}}	\newcommand{\mfV}{\mathfrak{V}}
\newcommand{\mfW}{\mathfrak{W}}	\newcommand{\mfX}{\mathfrak{X}}
\newcommand{\mfY}{\mathfrak{Y}}	\newcommand{\mfZ}{\mathfrak{Z}}
%Small Letters
\newcommand{\mfa}{\mathfrak{a}}	\newcommand{\mfb}{\mathfrak{b}}
\newcommand{\mfc}{\mathfrak{c}}	\newcommand{\mfd}{\mathfrak{d}}
\newcommand{\mfe}{\mathfrak{e}}	\newcommand{\mff}{\mathfrak{f}}
\newcommand{\mfg}{\mathfrak{g}}	\newcommand{\mfh}{\mathfrak{h}}
\newcommand{\mfi}{\mathfrak{i}}	\newcommand{\mfj}{\mathfrak{j}}
\newcommand{\mfk}{\mathfrak{k}}	\newcommand{\mfl}{\mathfrak{l}}
\newcommand{\mfm}{\mathfrak{m}}	\newcommand{\mfn}{\mathfrak{n}}
\newcommand{\mfo}{\mathfrak{o}}	\newcommand{\mfp}{\mathfrak{p}}
\newcommand{\mfq}{\mathfrak{q}}	\newcommand{\mfr}{\mathfrak{r}}
\newcommand{\mfs}{\mathfrak{s}}	\newcommand{\mft}{\mathfrak{t}}
\newcommand{\mfu}{\mathfrak{u}}	\newcommand{\mfv}{\mathfrak{v}}
\newcommand{\mfw}{\mathfrak{w}}	\newcommand{\mfx}{\mathfrak{x}}
\newcommand{\mfy}{\mathfrak{y}}	\newcommand{\mfz}{\mathfrak{z}}

\usetikzlibrary{babel,positioning,chains,fit,shapes,calc}

\title{\Huge{Matemática Discreta 1}}
\author{\huge{Santiago Sierra}}
\date{}

\begin{document}

\definecolor{myblue}{RGB}{80,80,160}
\definecolor{mygreen}{RGB}{80,160,80}

\maketitle
\newpage% or \cleardoublepage
% \pdfbookmark[<level>]{<title>}{<dest>}
\pdfbookmark[section]{\contentsname}{toc}
\tableofcontents
\pagebreak

\chapter{Relaciones}
\section{Definición de Relaciones}
\dfn{}{Para los conjuntos $A,\ B \subseteq U$, el producto cartesiano de $A$ y $B$ se denota por $A\times B$, y $$A\times B=\{(a,b)\ /\ a\in A \wedge b\in B\}$$}
\cor{}{Notamos como $|A|$ al cardinal de un conjunto, que representa la cantidad de elementos en $A$.}
\cor{}{Si los conjuntos $A,\ B$ son finitos, se sigue de la regla del producto que $|A\times B|=|A|\cdot|B|$.\\
	Aunque generalmente no ocurre que $A\times B=B\times A$, tenemos que $|A\times B|=|B\times A|$.}
\dfn{}{Para los conjuntos $A,B\subseteq U$, cualquier subconjunto de $A\times B$ es una relación de $A$ en $B$.\\
	A los subconjuntos de $A\times A$ se les llama relaciones sobre $A$.}
\dfn{Relación binaria}{Una relación binaria es un conjunto de pares ordenados pertenecientes al producto cartesiano de dos conjuntos que cumple una propiedad $P(a,b)$ en particular, es decir: $$R=\{(a,b)\in A\times B\ /\ P(a,b)\}$$
	Notaremos como $aRb$ para indicar que $(a,b)\in R$ y $a\cancel{R}b$ para expresar que $(a,b)\notin R$}
\cor{}{En general, para conjuntos finitos $A,\ B$, existen $2^{|A\times B|}=2^{|A||B|}$ relaciones de $A$ en $B$, incluyendo la relación vacía y la propia relación $A\times B$.}
\dfn{Relación inversa}{Si $R$ es una relación sobre $A$, entonces $R^{-1}$ es una relación sobre $A$ definida por $xRy\Leftrightarrow yR^{-1}x,\ \forall x,y\in A$.
	\\Es decir, da vuelta el par ordenado.}
\cor{}{$(R^{-1})^{-1}=R$.}
\cor{}{$R\subseteq S\Rightarrow R^{-1}\subseteq S^{-1}$.}
\dfn{}{$\overline{R}=R^C=A\times A-R=\{(a,b)\in A\times A: (a,b)\notin R\}$.\\
	$$a\overline{R}b\Leftrightarrow a\cancel{R}b$$}
\newpage\section{Tipos de relaciones}
\dfn{}{Sea $R$ una relación en un conjunto $A$:
	\begin{itemize}
		\item La relación $R$ es reflexiva si: $$\forall a\in A,\ aRa$$
		\item La relación $R$ es irreflexiva si: $$\forall a\in A,\ a\cancel{R}a$$
		\item La relación $R$ es simétrica si: $$\forall a,b\in A, aRb\Leftrightarrow bRa$$
		\item La relación $R$ es anti-simétrica si: $$\begin{rcases}aRb\\bRa\end{rcases}\Rightarrow a=b$$
		\item La relación $R$ es asimétrica si: $$aRb\Rightarrow b\cancel{R}a$$
		      Una relación asimétrica no puede ser reflexiva ni simétrica, e implica la anti-simétrica.
		\item La relación $R$ es transitiva si: $$\begin{rcases}aRb\\bRc\end{rcases}\Rightarrow aRc$$
	\end{itemize}}
\cor{}{$R$ es simétrica $\Leftrightarrow R=R^{-1}$.}
\cor{}{Si $R$ es reflexiva, simétrica, etc, $R^{-1}$ es del mismo tipo.}
\dfn{}{Una relación $R$ sobre un conjunto $A$ es un orden parcial, o una relación de orden parcial, si $R$ es reflexiva, anti-simétrica y transitiva.}
\dfn{}{Una relación de equivalencia $R$ sobre un conjunto $A$ es una relación que es reflexiva, simétrica y transitiva.}
\newpage\section{Producto, unión e intersección de relaciones}
\dfn{}{Si $A,B,C$ son conjuntos y $R\subseteq A\times B$ y $S\subseteq B\times C$, entonces la relación compuesta $RS$ es una relación de $A$ en $C$ definida como $RS=\{(x,z) / x\in A \wedge z\in C \wedge \exists y\in B / (x,y)\in R \wedge (y,z)\in S \}$.}
\thm{}{Sean $A,B,C,D$ conjuntos y $R_1\subseteq A\times B,\ R_2\subseteq B\times C,\ R_3\subseteq C\times D$.\\Entonces $R_1(R_2R_3)=(R_1R_2)R_3$.}
\dfn{}{Dado un conjunto $A$ y una relación $R$ sobre $A$, definimos las potencias de $R$ en forma recursiva como:
	\begin{itemize}
		\item $R^1=R$.
		\item Para $n\in\mathbb{Z}^+,\ R^{n+1}=R\ R^n$.
	\end{itemize}}
\ex{}{Si $A=\{1,2,3,4\}$ y $R=\{(1,2),(1,3),(2,4),(3,2)\}$, entonces $R^2=\{(1,4),(1,2),(3,4)\}$, $R^3=\{(1,4)\}$ y para $n\ge4,\ R^n=\emptyset$.}
\newpage\section{Representación matricial y dígrafos}
\subsection{Representación matricial}
\dfn{}{Una matriz cero-uno $m\times n,\ E=(e_{ij})_{m\times n}$ es una disposición rectangular de números en $m$ filas y $n$ columnas, donde cada $e_{ij}$ para $1\le i\le m$, y $1\le j\le n$, denota la entrada de la i-esima columna de $E$ y cada una de dichas entradas es $0$ o $1$.}
\dfn{}{Si $R$ es una relación entre $A$ y $B$, entonces $R$ puede ser representado por la matriz $M$ cuyos indices de fila y columna indexan los elementos de $a$ y $b$, respectivamente, de manera que las entradas de $M$ quedan definidas por:$$m_{ij}=\begin{cases}1 & aRb\\0 & a\cancel{R}b\end{cases}$$}
\cor{}{Sea $A$ un conjunto y $R$ una relación sobre $A$, si $M(R)$ es la matriz de la relación, entonces:
	\begin{itemize}
		\item $M(R)=0$, la matriz con todos los elementos iguales a 0, si y sólo si $R=\emptyset$.
		\item $M(R)=1$, la matriz con todos los elementos iguales a 1, si y sólo si $R=A\times A$.
		\item $M(R^m)=[M(R)]^m,\ m\in\mathbb{Z}$.
	\end{itemize}}
\cor{}{Sean $R$ y $S$ relaciones, y $M(R),\ M(S)$ sus matrices respectivamente,\\ entonces $M(R)\cdot M(S)=M(RS)$, pero ningún termino excede el $1$.}
\dfn{}{Sean $E$ y $F$ dos matrices cero-uno $m\times n$. Decimos que $E$ es menor que $F$ y escribimos $E\le F$, si $e_{ij}\le f_{ij}$, para todos $1\le i\le m,\ 1\le j\le n$.\\
	En caso que exista al menos un elemento que no cumpla esto, las matrices no son comparables}
\ex{}{$$\begin{pmatrix} 0 & 1\\1 & 0 \end{pmatrix}\le \begin{pmatrix} 1 & 1\\1 & 0 \end{pmatrix}$$
	$$\begin{pmatrix} 1 & 1\\1 & 0 \end{pmatrix} \cancel{\le} \begin{pmatrix} 0 & 1\\ 1 & 1 \end{pmatrix}\ \
		\begin{pmatrix} 0&1\\1&1 \end{pmatrix} \cancel{\le} \begin{pmatrix} 1&1\\1&0 \end{pmatrix} $$}\hfill\break
\dfn{Intersección o producto coordenada a coordenada}{Si $M,\ N\in\mathcal{M}_{r\times s}\Rightarrow M\cap N\in\mathcal{M}_{r\times s} / (M\cap N)=M_{ij}\cdot N_{ij}$.}
\thm{}{Dado un conjunto $A$ con y una relación $R$ sobre $A$, sea $M$ la matriz de relación para $R$, entonces:
	\begin{itemize}
		\item $R$ es reflexiva $\Leftrightarrow Id_{|A|}\le M$. En el caso contrario, es irreflexiva.
		\item $R$ es simétrica $\Leftrightarrow M=M^t$.
		\item $R$ es transitiva $\Leftrightarrow MM=M^2\le M$.
		\item $R$ es anti-simétrica $\Leftrightarrow M\cap M^t\le Id_{|A|}$.\\
		      Recordar, esta matriz se forma operando los elementos correspondientes de $M$ y $M^t$ con las reglas $0\cap0=0\cap1=1\cap0=0$ y $1\cap1=1$ (lo mismo que multiplicación coordenada a coordenada).
	\end{itemize}}
\cor{}{Sean las relaciones $R,\ S$, y $M(R),\ M(S)$ las matrices de las relaciones, entonces:
	\begin{itemize}
		\item $M(R)^t=M(R^{-1})$.
		\item $M(R)+M(S)=M(R\cup S)$.
		\item $M(R)\cap M(S)=M(R\cap S)$.
	\end{itemize}}
\cor{Conteo de Relaciones}{Sea $|A|=n$:
	\begin{itemize}
		\item Cantidad de relaciones reflexivas: $2^{n^2-n}$\\Tenemos 2 posibilidades, $0$ o $1$, en la diagonal siempre tiene que haber 1, y la matriz va a tener un total de $n^2$ entradas, y les restamos $n$ que son las entradas que ya están ocupadas por la diagonal.
		\item Cantidad de relaciones simétricas: $2^{\frac{n^2+n}{2}}$\\Devuelta, 2 posibilidades, la diagonal no importan las entradas si son $0$ o $1$, solo que sea simétrica, por lo que podemos elegir las entradas de la triangular superior y se determinan las del otro lado, que son $\frac{n^2-n}{2}$, pero les sumamos $n$ de la diagonal, $\frac{n^2-n}{2}+n=\frac{n^2+n}{2}$.
		\item Cantidad de relaciones anti-simétricas: $2^n3^{\frac{n^2-n}{2}}$\\Las posibilidades de la diagonal son $2^n$, y fuera de la diagonal, los elementos $m_{ij}$ y $m_{ji}$, pueden ser ambos $0$, o uno $0$ y el otro $1$, por lo que son 3 posibilidades, y determinando la triangular superior ya se determina el otro lado, y son $\frac{n^2-n}{2}$ entradas.
	\end{itemize}}
\subsection{Dígrafos de relaciones}
\dfn{}{Sea $A$ un conjunto finito no vació, un grafo dirigido o dígrafo $G$ sobre $A$ esta formado por los elementos de $A$, llamados vértices o nodo de $G$, y un subconjunto $E$ de $A\times A$, conocido como las aristas o arcos de $G$. Si $a,b\in V$ y $(a,b)\in E$, entonces existe una arista de $a$ a $b$.\\El vértice $a$ es el origen o fuente de la arista, y $b$ es el termino, o vértice terminal, y decimos que $b$ es adyacente desde $a$, y que $a$ es adyacente hacia $b$. Ademas, si $a\neq b$, entonces $(a,b)\neq(b,a)$. Una arista de la forma $(a,a)$ es una lazo en $a$.}
\section{Relaciones de equivalencia y particiones}
\dfn{}{Dado un conjunto $A$ y un conjunto de indices $I$, sea $\emptyset\neq A_i \subseteq A\ \forall i\in I$. Entonces $\{A_i\}_{i\in I}$ es una partición de $A$ si $$A=\bigcup _{i\in I} A_{i} \ \ \text{y} \ \ A_{i} \cap A_{j} =\emptyset \ \forall i,j\in I\ /\ i\neq j$$
Cada subconjunto $A$, es una celda o bloque de la partición.}
\dfn{}{Sea $R$ una relación de equivalencia sobre un conjunto $A$. Para cualquier $x\in A$, la clase de equivalencia de $x$, que se denota con $[x]$, se define como $[x]=\{y\in A\ /\ yRx\}$.}
\thm{}{Si $R$ es una relación de equivalencia sobre un conjunto $A$, y $x,y\in A$, entonces
	\begin{itemize}
		\item $x\in [x]$.
		\item $xRy \Leftrightarrow [x]=[y]$.
		\item $[x]\neq[y]\Rightarrow [x]\cap[y]=\emptyset$ o $[a]\cap[b]\Rightarrow [a]=[b]$.
	\end{itemize}}
\thm{}{Si $A$ es un conjunto, entonces:
	\begin{itemize}
		\item Cualquier relación de equivalencia $R$ sobre $A$ induce una partición de $A$.
		\item Cualquier partición de $A$ da lugar a una relación de equivalencia $R$ sobre $A$.
	\end{itemize}}
\thm{}{Para cualquier conjunto $A$, existe una correspondencia uno a uno entre el conjunto de relaciones de equivalencia sobre $A$ y el conjunto de particiones de $A$.}
\dfn{}{Dada una relación de equivalencia $R$ en un conjunto $A$, se llama conjunto cociente de $A$ determinado por $R$ al conjunto formado por todas las clases de equivalencia. Se le representa por $A/R$. Es decir:$$A/R=\{[a] \ /\ a\in A\}$$}
\newpage\section{Conjuntos parcialmente ordenados}
Para analizar el concepto de orden, sea $A$ un conjunto y $R$ una relación sobre $A$, el par $(A,R)$ es un conjunto parcialmente ordenado, si la relación sobre $A$ es un orden parcial.
\dfn{Diagrama de Hasse}{En general, si $R$ es un orden parcial sobre un conjunto finito $A$, construimos un diagrama de Hasse para $R$ sobre $A$ trazando un segmento de $x$ hacia arriba, hacia $y$, si $x,y\in A$ son tales que $xRy$ y, lo que es mas importante, si no existe otro elemento $z\in A$ tal que $xRz$ y $zRy$. Si adoptamos el convenio de leer el diagrama de abajo hacia arriba, no es necesario dirigir las aristas.}
\dfn{}{Si $(A,R)$ es un conjunto parcialmente ordenado, decimos que $A$ es totalmente ordenado si $\forall x,y\in A$ ocurre que $xRy$ o $yRx$.\\
	En este caso, decimos que $R$ es un orden total.}
\dfn{}{Si $(A,R)$ es un conjunto parcialmente ordenado, entonces un elemento $x\in A$ es un elemento maximal de $A$ si $\forall a\in A,\ a\neq x\Rightarrow x\cancel{R}a$. Un elemento $y\in A$ es un elemento minimal de $A$ si $\forall b\in A,\ b\neq y\Rightarrow b\cancel{R}y$.}
\thm{}{Si $(A,R)$ es un conjunto parcialmente ordenado y $A$ es finito, entonces $A$ tiene un elemento maximal y uno minimal.}
\dfn{}{Si $(A,R)$ es un conjunto parcialmente ordenado, entonces decimos que $x\in A$ es un elemento mínimo si $xRa\ \forall a\in A$. El elemento $y\in A$ es un elemento máximo si $aRy\ \forall a\in A$.}
\thm{}{Si el conjunto parcialmente ordenado $(A,R)$ tiene un elemento máximo o mínimo, entonces ese elemento es único.}
\dfn{}{Sea $(A,R)$ un conjunto parcialmente ordenado con $B\subseteq A$. Un elemento $x\in A$ es una cota inferior de $B$ si $xRb\ \forall b\in B$. De manera similar, un elemento $y\in A$ es una cota superior de $B$ si $bRy\ \forall b\in B$.\\
	Un elemento $x'\in A$ es una máxima cota inferior o ínfimo de $B$ si es una cota inferior de $B$ y si para todas las demás cotas inferiores $x''$ de $B$ tenemos que $x''Rx'$. De forma análoga, $y'\in A$ es una mínima cota superior o supremo de $B$ si es una cota superior de $B$ y si $y'Ry''$ para todas las demás cotas superiores de $y''$ de $B$.}
\thm{}{Si $(A,R)$ es un conjunto parcialmente ordenado y $B\subseteq A$, entonces $B$ tiene a lo sumo un ínfimo y supremo.}
\dfn{}{El conjunto parcialmente ordenado $(A,R)$ es un retículo si para cualesquiera $x,y\in A$, los elementos $sup\{x,y\}$ e $inf\{x,y\}$ existen en $A$.}
\dfn{Cadena}{Sea $B\subseteq A$, $B$ es cadena si $\forall a,b\in B\ aRb$ o $bRa$.}
\dfn{Anticadena}{Sea $B\subseteq A$, $B$ es anticadena si $\begin{cases} a\cancel{R}b\\b\cancel{R}a\end{cases}$ con $a\neq b$.}
\thm{}{Sea $n$ el largo de la cadena mas larga $\Rightarrow\ A$ se puede particionar en $n$ anticadenas disjuntas.\\
	Sea $m$ la cantidad de elementos de la anticadena mas grande $\Rightarrow\ A$ se puede particionar en $m$ cadenas disjuntas.}
\newpage\chapter{Teoría de Grafos}
\section{Introducción}
\dfn{}{Sea $V$ un conjunto finito no vació, y sea $A\subseteq V\times V$. El par $(V,A)$ es un grafo sobre $V$, donde $V$ es el conjunto de vértices o nodos, y $A$ es su conjunto de aristas. Escribimos $G=(V,A)$ para denotar tal grafo.}
\cor{}{Para cualquier arista $(a,b)\in A$ se dice que:
	\begin{itemize}
		\item La arista $(a,b)$ es incidente con los vértices $a$ y $b$.
		\item $a,b$ son los extremos de $(a,b)$.
		\item $a$ es el origen de $(a,b)$.
		\item $b$ es el termino de $(a,b)$.
	\end{itemize}}
\dfn{}{Dados dos vértices $u,v\in V$ con $G=(V,A)$, decimos que son vértices adyacentes si $(u,v)\in A$, ademas se dice que $u$ y $v$ son adyacentes, y que $u$ es adyacente hacia $v$.}
\dfn{}{Dados $v\in V,\ a\in A$ con $G=(V,A)$, decimos que la arista $a$ es incidente al vértice $v$ si $\exists u\in V$ tal que $a=(u,v)$.}
\dfn{}{Una arista de la forma $(a,a)$ es un lazo.}
\dfn{}{Un vértice $a\in V$ esta aislado cuando no hay ningún vértice adyacente con $a$.}
\cor{}{Un vértice con un lazo no está aislado.}
\dfn{}{Dado $G=(V,A)$ grafo, llamamos:
	\begin{itemize}
		\item Orden de $G$ al número de vértices, $m=|V|$.
		\item Tamaño de $G$ al número de aristas, $n=|A|$.
	\end{itemize}}
\newpage\section{Tipos de caminos}
\dfn{Camino}{Sea $G=(V,A)$ un grafo, un camino (en $G$) es una sucesión finita y alternada de vértices y aristas, de la forma: $$c=v_0,a_1,v_1,a_2,v_2,\dots,v_{n-1},a_n,v_n$$donde $a_i=(v_{i-1},v_i)$ y $1\le i\le n$.\\Siendo $n$ el número de aristas, y la longitud del camino.}
\cor{}{Un camino puede repetir vértices y/o aristas.}
\dfn{}{Un camino $c=v_0,\dots,v_n$ esta cerrado cuando $v_0=v_n$. De otra forma, esta abierto.}
\dfn{}{Un recorrido es un camino sin repetición de aristas.}
\dfn{}{Un camino simple es un camino sin repetición de vértices (con la excepción posible de los extremos).}
\dfn{}{Un ciclo es un camino simple cerrado.}
\cor{}{Un ciclo de longitud 1 es un lazo.
	\begin{itemize}
		\item En un grafo dirigido, pueden existir ciclos de longitud 2.
		\item En un grafo no dirigido todos los ciclos tienen longitud $\ge 3$.
	\end{itemize}}
\begin{table}[htb]\centering
	\begin{tabular}{|c|c|c|c|}
		\hline
		Nombre          & Origen = Termino? & Vértices Repetidos? & Aristas Repetidas? \\ \hline
		Camino          &                   &                     &                    \\ \hline
		Camino Cerrado  & Si                &                     &                    \\ \hline
		Camino Simple   &                   & No                  & No                 \\ \hline
		Recorrido       &                   &                     & No                 \\ \hline
		Circuito        & Si                &                     & No                 \\ \hline
		Circuito Simple & Si                & No                  & No                 \\ \hline
	\end{tabular}
\end{table}
\cor{}{Se suele considerar como iguales los circuitos simples que pasan por las mismas aristas.}
\thm{}{Sean $a$ y $b$ dos vértices distintos de un grafo $G$, si existe un camino desde $a$ hasta $b$, entonces existe (otro) camino simple desde $a$ hasta $b$.}
\newpage\section{Grafos conexos y disconexos}
\dfn{}{Un grafo $G=(V,A)$ es conexo cuando $\forall a,b\in V,\ a\neq b$, existe un camino desde $a$ hasta $b$.\\
	Un grafo dirigido es conexo, cuando el grafo subyacente (olvidando la orientación de las aristas) es conexo.}
\dfn{}{Un grafo (dirigido o no) es disconexo cuando no es conexo.}
\dfn{Componentes Conexas}{Dado $G=(V,A)$ se considera la relación $(\sim)\subset V^2$ definida por $v\sim v'\Leftrightarrow$ existe un camino desde $v$ hasta $v'$.\\Es una relación de equivalencia. Las clases de equivalencia correspondiente son las componentes conexas del grafo $G$.}
\dfn{}{Cuando $G$ es finito, se escribe $K(G)$ al numero de componentes conexas de $G$.}
\cor{}{$G$ es conexo $\Leftrightarrow K(G)=1$.}
\cor{}{Las componentes conexas de un grafo dirigido se definen a partir del grafo no dirigido subyacente.}
\dfn{Multigrafo}{Un multigrafo es una terna $(V,A,ext)$ donde:
	\begin{itemize}
		\item $V$ es el conjunto de vértices.
		\item $A$ es el conjunto de aristas.
		\item $ext$: $A\to V^2$, $ext(a)=(x,y)$ quiere decir $x\xrightarrow{a} y$.
	\end{itemize}
	Es decir, de un vértice pueden salir múltiples aristas.}
\section{Subgrafos, grafos complementarios}
\dfn{}{Sea $G=(V,A)$ un grafo, un subgrafo de $G$ es un grafo $G'=(V',A')$ tal que $\begin{cases}V'\subset V\\A'\subset A\end{cases}$.}
\cor{}{$G'=(V',A')$ tiene que ser un grafo: $A'\subset V'^2$.}
\dfn{}{Sea $G=(V,A)$ un grafo, un subgrafo recubridor de $G$ es un subgrafo $G'=(V',A')\subset G$ tal que $V'=V$.}
\dfn{Grafo completo}{Un grafo $G=(V,A)$ es completo si es:
	\begin{itemize}
		\item No dirigido.
		\item No tiene lazos.
		\item $\forall a,b\in V,\ a\neq b,\ \{a,b\}\in A$.
	\end{itemize}
	De modo equivalente $\forall a,b\in V,\ \{a,b\}\in A\Leftrightarrow a\neq b$.\\
	Se escribe $K_n$ el grafo completo con $n$ vértices.}
\cor{}{$K_n$ tiene $n$ vértices y $\frac{n(n-1)}{2}$ aristas.}
\cor{}{Todo grafo $G=(V,A)$ sin lazos es un subgrafo recubridor de algún grafo completo $\hat{G}=(V,\hat{A})$ donde $\hat{A}=\{\{a,b\}\in V^2\ /\ a\neq b\}$.}
\dfn{}{Si $G=(V,A)$ es un grafo sin lazos, el grafo complementario de $G$ esta definido por $\overline{G}=(V,\overline{A})$ con $\overline{A}=\{\{a,b\}\in V^2\ /\ a\neq b\wedge \{a,b\}\notin A\}$.}
\cor{}{El complementario no tiene lazos.}
\cor{}{El complementario del complementario es el grafo inicial: $\overline{\overline{G}}=G$.}
\newpage\section{Isomorfismos de grafos}
\dfn{}{Un isomorfismo entre dos grafos $G_1=(V_1,A_1)$ y $G_2=(V_2,A_2)$ es una función $f:V_1\to V_2$ tal que:
	\begin{itemize}
		\item $f$ es biyectiva.
		\item $\forall a,b\in V_1,\ \{a,b\}\in A_1\Leftrightarrow\{f(a),f(b)\}\in A_2$.
	\end{itemize}
	Dos grafos $G_1$ y $G_2$ son isomorfos cuando existe un isomorfismo $f:G_1\xrightarrow{\sim} G_2$.}
\cor{}{Si $G_1$ es isomorfo con $G_2$, entonces $G_2$ es isomorfo con $G_1$.}
\cor{}{Si el grafo $G_1$ es isomorfo con el grafo $G_2$, y el grafo $G_2$ es isomorfo con el grafo $G_3$, entonces el grafo $G_1$ es isomorfo con $G_3$.}
\cor{}{De modo análogo se define la noción de isomorfismo para los grafos dirigidos y los multigrafos.}
\cor{}{Para cada $n\ge 3$, se define el grafo $C_n=(V_n,A_n)$ que es el ciclo de orden $n$, con $V_n=\{1,\dots,n\}$ y $A_n=\{\{1,2\},\{2,3\},\dots,\{n-1,n\},\{n,1\}\}$.}
\dfn{}{En un grafo $G$ sin lazos, un ciclo de orden $n\ (\ge 3)$ es un subgrafo de $G$ isomorfo a $C_n$.}
\dfn{Árboles}{Un árbol es un grafo finito sin lazos, conexo, y sin ciclos.}
\dfn{Bosque}{Un bosque es un grafo cuyos componentes conexos son arboles.}
\dfn{}{Sea $G$ un grafo, un árbol recubridor de $G$ es un subgrafo recubridor de $G$ que es un árbol.}
\dfn{}{Un bosque recubridor de $G$ es un bosque $B\subset G$ tal que $B$ es un subgrafo recubridor de $G$, cada árbol de $B$ es un árbol recubridor de una componente conexa de $G$.}
\thm{}{Un grafo es conexo si y sólo si tiene un árbol recubridor.}
\thm{}{Un grafo $G=(V,A)$ es un árbol si y solo si $\forall v,w\in V,\ v\neq w$, existe un único camino simple de $v$ a $w$.}
\dfn{}{En un grafo $G$ el grado de un vértice $v$ es el numero de aristas incidentes a $v$, y lo notamos como $gr_G(v)$.}
\cor{}{Si un vértice tiene un lazo, se cuenta 2 veces.}
\cor{}{En un árbol con al menos $2$ vértices, existe (al menos) un vértice de grado 1.}
\thm{}{Si $G=(V,A)$ es un árbol no vació, entonces: $|V|=|A|+1$.}
\thm{}{Sea $G=(V,A)$ un grafo finito sin lazos, las siguientes propiedades son equivalentes:
	\begin{itemize}
		\item $G$ es un árbol.
		\item $G$ es conexo, y si se elimina cualquier arista de $G$ se obtiene un bosque de $2$ árboles.
		\item $G$ no contiene ciclos, y $|V|=|E|+1$.
		\item $G$ es conexo, y $|V|=|E|+1$.
		\item $G$ no contiene ciclos, y cualquier arista suplementaria entre $2$ vértices de $V$ introduce un ciclo.
	\end{itemize}}
\newpage\section{Caminos eulerianos y hamiltonianos}
\dfn{}{Sea $G$ un grafo o multigrafo, para cualquier vértice $v$ de $G$, se llama grado de $v$ y se escribe $gr_G(v)$ (o $gr(v)$) el número de aristas incidentes con $v$, contando cada lazo (en $v$) $2$ veces.}
\cor{}{Cualquier grafo o multigrafo $G$ finito $G=(V,A)$, tenemos que $\sum_{v\in V} gr(v)=2|E|$.}
\cor{}{En cualquier grafo finito, la cantidad de vértices de grado impar es par.}
\dfn{}{Un grafo o multigrafo es regular cuando todos sus vértices tienen el mismo grado.\\
	Es $k$-regular cuando todos sus vértices tienen grado $k$.}
\cor{}{Si $G=(V,A)$ es $k$-regular, entonces $k|V|=2|A|$.}
\dfn{Caminos y circuitos eulerianos}{En un grafo o multigrafo $G$ finito no orientado, un recorrido euleriano es un recorrido que pasa por cada arista del grafo ($1$ vez cada una).
	\\De misma manera definimos los circuitos eulerianos, con la diferencia de que tiene que ser un circuito.}
\thm{}{Sea $G$ un grafo o multigrafo no dirigido $G=(V,A)$ no vació y sin vértices aislados.\\
	Entonces, $G$ tiene un circuito euleriano si y solo si $G$ es conexo y todos los vértices son de grado par.}
\cor{}{Sea $G$ un grafo o multigrafo no dirigido, sin vértices aislados, $G$ tiene un recorrido euleriano si y solo si $G$ es conexo y todos los vértices de $G$ tienen grado par, excepto por $2$.}
\dfn{Caminos y ciclos hamiltonianos}{Sea $G$ un grafo o multigrafo dirigido o no, un camino hamiltoniano un camino simple de $G$ que pasa por todos los vértices de $G$.\\
	Un ciclo hamiltoniano de $G$ es un ciclo de $G$ que pasa por todos los vértices de $G$.}
\cor{}{No hay relación entre los caminos y ciclos hamiltonianos con los recorridos y circuitos eulerianos.}
\cor{}{Cada grafo que tiene un ciclo hamiltoniano tiene un camino hamiltoniano (sacando cualquier arista).}
\dfn{}{Un grafo es hamiltoniano si tiene un ciclo hamiltoniano.\\
	Un grafo es semi-hamiltoniano si tiene un camino hamiltoniano pero ningún ciclo hamiltoniano.}
\cor{}{$K_n$ es hamiltoniano.}
\dfn{}{Un torneo es un grafo dirigido $G=(V,A)$ tal que para todos los vértices $x\neq y\in V$, tenemos que $(x\to y)\in A$ o (exclusivo) $(y\to x)\in A$.}
\dfn{Inclusión}{Un torneo es un grafo obtenido a partir de $K_n\ (n\ge1)$ orientando cada arista.}
\thm{}{Todo torneo tiene un camino hamiltoniano.}
\thm{}{Sea $G=(V,A)$ un grafo no dirigido sin lazos, si para todos $x\neq y\in V,$ tenemos que $gr(x)+gr(y)\ge n-1$ (con $n=|V|$), entonces existe un camino hamiltoniano.}
\cor{}{Sea $G$ un grafo no dirigido sin lazos con $|V|=n$, si $gr(x)\ge \frac{n-1}{2}$ para todo $x\in V$, entonces $G$ tiene un camino hamiltoniano.}
\thm{}{Sea $G$ un grafo no dirigido sin lazos, con $|V|=n$, si para todos $x\neq y\in V$, tenemos que $gr(x)+gr(y)\ge n$ entonces existe un ciclo hamiltoniano.}
\cor{}{Si $gr(x)\ge \frac{n}{2}$ para todo $x\in V$, entonces $G$ tiene un ciclo hamiltoniano.}
\section{Grafos planos}
\dfn{}{Un grafo o multigrafo $G$ es plano si se puede dibujar $G$ en el plano $(\mathbb{R}^2)$ que sus aristas se intersequen solo en los extremos que comparten.}
\cor{}{La definición se basa en la noción de representación de un grafo $G$ en el plano. Formalmente, una representación de $G$ en el plano es una función que:
	\begin{itemize}
		\item Asocia a cada vértice $v\in V$ a un punto $f(v)\in\mathbb{R}^2$.
		\item Asocia a cada arista $a\in A$ de extremos $v_1,v_2\in V$, una linea continua $f(a)\in\mathbb{R}^2$ que junta $f(v_1)$ y $f(v_2)$.
	\end{itemize}
	Dicha representación es una inmersión (encaje) cuando:
	\begin{itemize}
		\item $f$ es inyectiva sobre $V$.
		\item $f(a)$ es una linea inyectiva.
		\item $f(a)$ intersecta $f(a')$ (con $a\neq a')$ en los extremos.
	\end{itemize}
	Cuando tal inmersión existe se dice que $G$ es plano.}
\dfn{}{Un grafo $G=(V,A)$ es bipartito si existe una partición $V=V_1\cup V_2$ ($V_1\cap V_2=\emptyset,\ V_1\neq\emptyset,\ V_2\neq\emptyset$) tal que toda arista de la forma $\{v_1,v_2\}$ con $v_1\in V_1$, $v_2\in V_2$.}
\begin{center}
	\begin{tikzpicture}[thick,
			every node/.style={draw,circle},
			fsnode/.style={fill=myblue},
			ssnode/.style={fill=mygreen},
			every fit/.style={ellipse,draw,inner sep=-2pt,text width=2cm},
			->,shorten >= 3pt,shorten <= 3pt
		]

		% Vertices de V_1
		\begin{scope}[start chain=going below,node distance=7mm]
			\foreach \i in {1,2,...,5}
			\node[fsnode,on chain] (f\i) [label=left: \i] {};
		\end{scope}

		% Vertices de V_2
		\begin{scope}[xshift=4cm,yshift=-0.5cm,start chain=going below,node distance=7mm]
			\foreach \i in {6,7,...,9}
			\node[ssnode,on chain] (s\i) [label=right: \i] {};
		\end{scope}

		\node [myblue,fit=(f1) (f5),label=above:$V_1$] {};
		\node [mygreen,fit=(s6) (s9),label=above:$V_2$] {};

		% Aristas
		\draw (f1) -- (s6);
		\draw (s6) -- (f2);
		\draw (f2) -- (s7);
		\draw (s7) -- (f3);
		\draw (s8) -- (f3);
		\draw (f3) -- (s9);
		\draw (s9) -- (f5);
		\draw (f5) -- (s6);
	\end{tikzpicture}
\end{center}
\cor{}{Un grafo bipartito es completo cuando $\{v_1,v_2\}\in A$ para todos $v_1\in V_1,\ v_2\in V_2$.}
\cor{}{En un grafo bipartito, todos los ciclos tienen longitud par.}
\dfn{}{$K_{m,n}$ es un grafo bipartito completo con $|V_1|=m,\ |V_2|=n$, con $m+n$ vértices y $m\times n$ aristas.}
\dfn{}{Sea $G=(V,A)$ un grafo o multigrafo:
	\begin{itemize}
		\item Una subdivisión elemental de $G$ es un grafo $G'=(V',A')$ obtenido a partir de $G$, remplazando una arista $\{u,v\}\in A$ por $2$ aristas $\{u,w\},\{w,v\}\in E'$ ($w$ un nuevo vértice).
		\item Una subdivisión de $G$ es un grafo $G'$ obtenido mediante finitas subdivisiones elementales de $G$:$$G\to G_1\to G_2\to\dots\to G_n=G'$$
	\end{itemize}}
\cor{}{Si $G'$ es una subdivisión de $G$:
	\begin{enumerate}
		\item $K(G)=K(G')$. En particular: $G$ conexo $\Leftrightarrow\ G'$ conexo.
		\item $G$ plano $\Leftrightarrow\ G'$ plano.
	\end{enumerate}}
\cor{}{Si un grafo $G$ es plano, entonces todos sus subgrafos son planos también.\\
	Al contrario, si $G$ contiene un subgrafo no plano (por ejemplo: un subgrafo isomorfo a una subdivisión de $K_{3,3}$ o $K_5$) entonces $G$ no puede ser plano.}
\thm{Teorema de Kuratowski}{Un grafo no es plano si y solo si contiene un subgrafo $G'\subset G$ isomorfo a una subdivisión de $K_{3,3}$ o $K_5$.}
\cor{}{Cada inmersión de un grafo plano en el plano define finitas regiones (todas finitas, salvo una)}
\thm{}{Sea $G=(V,A)$ un grafo o multigrafo plano conexo con $|V|=v$ y $|A|=a$. Sea $r$ el numero de regiones en el plano determinadas por una inmersión (o representación) plana de $G$, una de estas regiones tiene un área infinita y se conoce como región infinita, entonces $v-a+r=2$.}
\cor{}{Mas generalmente, si $G$ es plano (conexo o no), $v-e+r=1+K(G)$.}
\dfn{}{Sea $G=(V,A)$ un grafo plano con $r\ge1$ regiones $R_1,\cdots,R_r$ a través de una inmersión dada en el plano.\\
	Dado $i\in[1,\cdots,r]$ el grado de $R_i$ es el numero de aristas que forman la frontera de $R_i$.}
\cor{}{$\sum_{i=1}^{r} gr(R_i)=2a\ (=2|A|)$.}
\cor{}{Si $G=(V,A)$ es un grafo simple, sin lazos, conexo, plano y no vació, entonces $3r\le 2e$ y $e\le 3v-6$.}
\end{document}

